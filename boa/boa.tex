\documentclass[article,A4,11pt]{llncs}%
\usepackage[utf8]{inputenc}
\usepackage{amsmath}
\usepackage{amssymb}
\usepackage{amsfonts}
\usepackage{mathrsfs}
\usepackage{graphicx}
\usepackage{times}
\usepackage{epsf}
\usepackage{bm}
\usepackage{cases}

\usepackage{url}
\usepackage{multicol}
\usepackage{tabularx}
%\usepackage{fullpage}
%\usepackage{mathrsfs}
%\usepackage{wrapfig}
%\usepackage{color}
%\usepackage{subfig}
%\usepackage{eso-pic}

\leftmargin=0.2cm
\oddsidemargin=1.2cm
\evensidemargin=0cm
\topmargin=0cm
\textwidth=15.5cm
\textheight=21.5cm
\pagestyle{plain}
\setlength{\columnsep}{20pt}


%\def\m{\mathbf{m}}
%\def\H{\mathbf{H}}
%\def\E{\mathbf{E}}
\newcommand{\vepsi}{{\varepsilon}}
\def\hnorm#1#2{\vert\,#1\,\vert_{#2}}
\newcommand{\R}{{\mathbb R}}
\newcommand{\Sph}{{\mathbb S}}
\def\x{\mathbf{x}}
\def\hvec{\overline{\mathbf{h}}}
\def\evec{\overline{\mathbf{e}}}

\DeclareMathAlphabet{\mathpzc}{OT1}{pzc}{m}{it}
%\leftmargin=0cm
%\oddsidemargin=1cm
%\textwidth=14cm
%\pagestyle{plain}

\newcommand{ \etal}{\mbox{\emph{et al. }}}

\newcommand\vect[1]{\mbf{#1}}
\newcommand{\mbf}[1]{\mbox{\boldmath$#1$}}
\newcommand{\RC}[1]{#1 $\times$ #1 $\times$ #1}
\def\um{$\mu$m}
\def\C{$^{\circ}\mathrm{C}$}

\def\clovek#1{\noindent\bgroup\vbox{\noindent#1}\egroup\vskip1em}

% DEFINITION OF CUSTOM FONT SIZE
\newcommand{\customfontA}{\fontsize{50}{55}\selectfont}
\newcommand{\customfontB}{\fontsize{14.4}{20}\selectfont}
\newcommand{\customfontC}{\fontsize{30}{35}\selectfont}

% TO INPUT BACKGROUND IMAGE

\newcommand\BackgroundPic{
\put(0,0){
\parbox[b][\paperheight]{\paperwidth}{%
\vfill
\centering
%\includegraphics[width=\paperwidth,height=\paperheight]{background_plzen.jpg}%
%\includegraphics[width=\paperwidth,height=\paperheight]{background_tmp.jpg}%
\vfill
}}}

% BEGIN DOCUMENT
\begin{document}

% inputting background image
%\AddToShipoutPicture{\BackgroundPic}

\vbox{}
\pagestyle{empty}

\newpage

\textwidth=15.5cm

%\ClearShipoutPicture

\newpage

\section*{}%

\vspace*{60mm}
%ISBN 978-80-7043-898-5\\ \\
This is a joint publication of the University of Nevada (Reno, USA),
University of West Bohemia (Pilsen, Czech Republic),
Czech Technical University (Prague, Czech Republic),
Institute of Thermomechanics (Prague, Czech Republic), and
FEMhub Inc (Reno, USA).\\

\noindent
FEMTEC 2013 \\
3rd European Seminar on Computing\\

\noindent
\begin{tabular}{ll}
Editors: & Pavel Solin (University of Nevada, Institute of Thermomechanics) \\
 & Pavel Karban (University of West Bohemia) \\
 & Jaroslav Kruis (Czech Technical University) \\
Publisher: & University of West Bohemia \\
 & Univerzitn\'{i} 8, 306 14 Plze\u{n}\\
 & Czech Republic\\
Printed by: & Dragon Print, s.r.o \\
 & Klatovsk\'{a} 24, 301 00 Plze\u{n}\\
 & Czech Republic\\
Year: & 2012\\
\end{tabular}

\subsection*{Contact Information}

Mailing address:\\
FEMTEC 2013 Conference\\
FEMhub Inc.\\
5490 Twin Creeks Dr.\\
Reno, NV 89523\\
U.S.A.

\noindent
E-mail: {\tt femtec2013@femhub.com}\\
Web page: {\tt http://femtec2013.femhub.com/}\\
Phone: 1-775-848-7892

\chapter*{\huge FEMTEC 2013}
\vspace{-5mm}
\normalsize
\begin{center}
3rd European Seminar on Computing,
Pilsen, Czech Republic,
June 25 - 29, 2012
\end{center}
\vspace{-3mm}

\section*{Main Thematic Areas}%

Multiphysics coupled problems; Higher-order computational methods
Computing with Python; GPU computing; Cloud computing.

\section*{Application Areas}%

Theoretical results as well as applications are welcome. Application areas include, but are not limited to: Computational electromagnetics, Civil engineering, Nuclear engineering, Mechanical engineering, Nonlinear dynamics, Fluid dynamics, Climate and weather modeling, Computational ecology, Wave propagation, Acoustics, Geophysics, Geomechanics and rock mechanics, Hydrology, Subsurface modeling, Biomechanics, Bioinformatics, Computational chemistry, Stochastic differential equations, Uncertainty quantification, and others.

\subsection*{Scientific Committee}%

%\hspace{4mm}

\begin{itemize}
\item Valmor de Almeida (Oak Ridge National Laboratory, Oak Ridge, USA)
\item Zdenek Bittnar (Faculty of Civil Engineering, CTU Prague)
\item Alain Bossavit (Laboratoire de Genie Electrique de Paris, France)
\item John Butcher (Auckland University, New Zealand)
\item Antonio DiCarlo (University Roma Tre, Rome, Italy)
\item Ivo Dolezel (Czech Technical University, Prague, Czech Republic)
\item Stefano Giani (University of Nottingham, UK)
\item Glen Hansen (Sandia National Laboratories, Albuquerque, USA)
\item Pavel Karban (University of West Bohemia, Pilsen, Czech Republic)
\item Darko Koracin (Desert Research Institute, Reno, USA)
\item Dmitri Kuzmin (University of Erlangen-Nuremberg, Germany)
\item Stephane Lanteri (INRIA, Sophia-Antipolis, France)
\item Jichun Li (University of Nevada, Las Vegas, USA)
\item Shengtai Li (Los Alamos National Laboratory, Los Alamos, USA)
\item Alberto Paoluzzi (University Roma Tre, Rome, Italy)
\item Jean Ragusa (Texas A\&M University, College Station, USA)
\item Francesca Rapetti (University of Nice, France)
\item Sascha Schnepp (Technical University of Darmstadt, Germany)
\item Stefan Turek (Technical University of Dortmund, Germany)
\end{itemize}

\subsection*{Organizing Committee}

\begin{itemize}
\item Pavel Solin (University of Nevada, Reno \& Institute of Thermomechanics, Prague)
\item Pavel Karban  (University of West Bohemia, Pilsen)
\item Jaroslav Kruis (Czech Technical University, Prague)
\end{itemize}

\newpage
{\ }

\tableofcontents

%%%%%%%%%%%%%%%%%%%%%%%%%%%%%%%%%%%%%%%%%%%%%%%%%%%%%%%%%%%%%%%%%%%%%%%%%%%%%%%%%%%%%%%%%%%%%%%%%%%%%%%%%%%%%%%%%%%%%%%%%%%%%%%%%%%%%%%%%%%%%%%%%%%%%%%%%
\part{Abstracts of Keynote Lectures}

\pagestyle{plain}
\part{List of Participants}
\title{Two Dimensional Numerical Modeling of Micro-shock Wave Creation in Nanosecond Plasma Actuators}
\tocauthor{M. Abdollahzadeh} \author{} \institute{}
\maketitle
\begin{center}
{\large M. Abdollahzadeh}\\
University of Beira Interior, Department of Electromechanical Engineering, Center for Aerospace Sciences and Technology, Portugal, Covilhã\\
{\tt mm.abdollahzadeh@yahoo.com}
\\ \vspace{4mm}
{\large J. Páscoa}\\
University of Beira Interior, Department of Electromechanical Engineering, Center for Aerospace Sciences and Technology, Portugal, Covilhã\\
{\tt pascoa@ubi.pt}
\\ \vspace{4mm}
{\large P. J. Oliveira}\\
University of Beira Interior, Department of Electromechanical Engineering, Portugal, Covilhã\\
{\tt pjpo@ubi.pt}
\end{center}

\section*{Abstract}
Surface dielectric barrier discharges (SDBDs) can modify the boundary layer of a flow and are studied as possible actuators for flow control. the air speed that can be achieved so far in SDBD is lower than 7-8 $ m*s^{-1} $. In this view, SDBDs are considered as a momentum source. 
The purpose of this study is to present a numerical modelling of a surface dielectric barrier discharge in air which uses a nanosecond voltage pulse generator. when a set of nanosecond voltage pulses with rise and decay time of the order of few nanoseconds is used, a significant amount of energy is transferred in a very short time from plasma to fluid which consequently lead to formation of micro shock weaves and therefore modifying the flow features.
 A two-dimensional fluid model of the DBD is used to describe the plasma dynamics. The model couples fluid discharge equations with compressible Navier–Stokes equations including momentum and thermal transfer from the plasma to the neutral
gas. The 2D fluid model of the discharge in air provides the space and time evolution of the charged particle densities, electric field and surface charges The model is numerically solved using implicit time integration technique . A validation of the model is presented in order to assess the capabilities of the developed computational code.

\bibliographystyle{plain}
\begin{thebibliography}{10}
\bibitem{Numerical analysis of plasma evolution on dielectric barrier discharge plasma actuator.}
{\sc H. Nishida and T. Abe}. {Numerical analysis of plasma evolution on dielectric barrier discharge plasma actuator.}. JOURNAL OF APPLIED PHYSICS 110, 013302, 2011.

\bibitem{Electrohydrodynamic force and aerodynamic flow acceleration in surface dielectric barrier}
{\sc J. P. Boeuf and L. C. Pitchford}. {Electrohydrodynamic force and aerodynamic flow acceleration in surface dielectric barrier}.  JOURNAL OF APPLIED PHYSICS 97, 103307, 2005.

\bibitem{Energy and force prediction for a nanosecond pulsed dielectric barrier discharge actuator}
{\sc Chin-Cheng Wang and Subrata Roy}. {Energy and force prediction for a nanosecond pulsed dielectric barrier discharge actuator}. JOURNAL OF APPLIED PHYSICS 111, 103302 (2012).
\end{thebibliography}

\title{An Accelerated-Time Simulation for Traffic Flow in a Smart City}
\tocauthor{Jos\'e Luis Gal\'an} \author{} \institute{}
\maketitle
\begin{center}
{\large Jos\'e Luis Gal\'an}\\
Universidad de M\'alaga\\
{\tt jl\_galan@uma.es}
\\ \vspace{4mm}
{\large Gabriel Aguilera}\\
Universidad de M\'alaga\\
{\tt gabri@ctima.uma.es}
\\ \vspace{4mm}
{\large Jos\'e Carlos Campos}\\
Universidad de M\'alaga\\
{\tt donjosecarlos@gmail.com}
\\ \vspace{4mm}
{\large Pedro Rodr\'{\i}guez}\\
Universidad de M\'alaga\\
{\tt prodriguez@uma.es}
\end{center}

\section*{Abstract}
Traffic control is nowadays one of the most important problems related with the urban development. New trends are based on the use of smart traffic lights and signals as a part of smart cities projects.
\\
Different cities are nowadays involved in the design and implementation of smart traffic control. Since the cost of the physical installation of such systems is very high, both in money and resources, accelerated-time simulations of traffic flow using smart traffic lights and signals are welcome.
\\
In this work we present a new model for accelerated-time simulations for traffic flow within this frame. The philosophy of this model is based on previous works of the authors, where accelerated-time simulations for car traffic in a motorway or in roundabouts, and baggage traffic in an airport were developed.
\\
The philosophy of the model combines ideas from cellular automata and neural networks theories, obtaining a mixed model.
\\
The implementation of the model is carried out using a CAS (Computer Algebra System) language. This fact allows the system to support very flexible mathematical conditions, including the use of ad-hoc distribution functions for the different events dealt in the accelerated-time simulations.
\\
A graphic interface, developed in JAVA, is used for the communication between the application, developed in the CAS Maxima, and the user. This interface also provides a friendly framework for both: entering input data and visualizing the obtained simulations together with different statistics.

\bibliographystyle{plain}
\begin{thebibliography}{10}
\bibitem{An accelerated-time simulation of car traffic on a motorway using a CAS}
{\sc Gabriel Aguilera and Jos\'e Luis Gal\'an and Jos\'e Manuel Garc\'{\i}a and Enrique M\'erida and Pedro Rodr\'{\i}guez}. {An accelerated-time simulation of car traffic on a motorway using a CAS}. Math. Comput. Simul. (2012), http://dx.doi.org/10.1016/j.matcom.2012.03.010.
\end{thebibliography}

\title{Mathematical and Numerical Study of the ROM-POD Sensitivity for a 2D Incompressible Fluid Flow}
\tocauthor{Nissrine Akkari} \author{} \institute{}
\maketitle
\begin{center}
{\large \underline{Nissrine Akkari}}\\
University of La Rochelle, LaSIE (Laboratoire des Sciences de l'Ingénieur pour l'Environnement)\\
{\tt nissrine.akkari@univ-lr.fr}
\\ \vspace{4mm}
{\large Aziz Hamdouni}\\
University of La Rochelle, LaSIE (Laboratoire des Sciences de l'Ingénieur pour l'environnement)\\
{\tt aziz.hamdouni@univ-lr.fr}
\\ \vspace{4mm}
{\large Erwan Liberge}\\
University of La Rochelle, LaSIE (Laboratoire des Sciences de l'Ingénieur pour l'environnement)\\
{\tt erwan.liberge@univ-lr.fr}
\\ \vspace{4mm}
{\large Mustapha Jazar}\\
University of Lebanon, LaMA (Laboratoire de Mathématiques et Applications)\\
{\tt mjazar@ul.edu.lb}
\end{center}

\section*{Abstract}
\vspace{-0.26cm}
We establish a mathematical and numerical study of the parametric sensitivity of the reduced order models (ROMs) by the Proper Orthogonal Decomposition (POD) method.
\vspace{-0.36cm} 
\section*{Theoretical and numerical results}
\vspace{-0.26cm}
$X=[L^{2}(\Omega)]^{2}$,  $V=\left\{v\in[H^{1}_{0}(\Omega)]^{2},\;\; div\;v=0\right\}$ where $\Omega$ is a bounded open set, connex and lipschitz of $\mathbb{R}^{2}$  and $V_h$ a subspace of $V$ of dimension $M$. $\lambda\in\mathbb{R}^{+*}$ denotes the viscosity parameter of the Navier-Stokes equations. We denote by $\Phi^{\lambda_0}$ a POD basis of dimension $M$ associated to the full solution $u^{h}_{\lambda_0}(t)\in V_h$ on a time interval $\textbf{T}=(0,T)$. $\hat{u}_{\lambda,\lambda_0}$ denotes the approximation of $u^{h}_{\lambda}$ in the reference POD subspace of dimension $N<<M$. We prove the following :
$$\left\|u^{h}_{\lambda}-\hat{u}_{\lambda,\lambda_0}\right\|^{2}_{L^{2}(\textbf{T},X)}\leq f^{\lambda_0}_{1}(N)+f^{\lambda_0}_{2}(N)\cfrac{\left|\lambda-\lambda_0\right|}{\lambda_0},$$
where $f^{\lambda_0}_{1}(N)$ is a fonction of the remainder of the POD eigenvalues sum associated to the full solution $u^{h}_{\lambda_0}$ and $f^{\lambda_0}_{2}(N)$ is the general term of a decreasing sequence for which $f^{2}_{\lambda_0}(0)<1/2$ and $f^{2}_{\lambda_0}(N)=0$ when $N=M$.\\
Numerical simulations of our result are done for a flow around a cylinder, in a driven cavity and in a ventilated one. We retrieve the same behavior of the error in function of the parametric variation as well as in function of the ROM's dimension $N$. We give also validations of this result for interpolation techniques of ROMs [Amsallem et al., 2009].         


\bibliographystyle{plain}
\begin{thebibliography}{10}
\bibitem{A method for interpolation on manifolds structural dynamics reduced-order models}
{\sc D. Amsallem and J. Cortial and K. Carlberg and C. Farhat}. {A method for interpolation on manifolds structural dynamics reduced-order models}. International journal for numerical methods in engineering (2009) 1241-1258.
\end{thebibliography}

\title{Smooth Particle Hydrodynamics of High Velocity Impact: Anti-Meteorite Protection of Space Apparatuses of Normal Impact of Aluminium Ball on the Shield Made of Two Aluminium Screens}
\tocauthor{Khaled Alhussan} \author{} \institute{}
\maketitle
\begin{center}
{\large Khaled Alhussan}\\
king Abdulaziz city for science and technology\\
{\tt alhussan@kacst.edu.sa}
\end{center}

\section*{Abstract}
SPH as a mesh-less (or meshfree) and particle method, was originally used for simulating astrophysical phenomena and later widely extended for applications to problems of continuum solid mechanics and fluid dynamics. SPH has been extensively studied and extended to dynamic response with material strength as well as dynamic fluid flows with large deformations. The problem of High-Velocity Impact of solid body arises in a study of asteroid danger, anti-meteoric protection of space ship hull, penetration problem, etc. This class of problem is characterized by a wide range of large gradients and large deformation combined with phase change.  Advantages of SPH consist in meshfree and adaptive nature of this method. Unlike the other meshfree methods, which are only used interpolation points, the SPH particles also carry material properties, and are allowed to move in light of the internal interactions and external forces. That is why the method is extensively used in simulating large deformation and impulsive loading events such as the explosion or high velocity impact. Smooth particle method is the best for the most complex cases of destruction and phase conversion.
High-velocity impact (HVI) is of interest for many problems of space physics and astrophysics:the study of meteoric craters, the origin of planetary atmospheres, and the consequences of large space object fall on the Earth etc. [1]. SPH method have received wide acceptance in HVI studies [1]. 
Possibilities of the designed SPH program [1] are demonstrated for the problem on anti-meteoroid protection of spacecrafts. The given problem is characterized by a particular set of parameters: a material of the shield, its thickness, a material of striker, its shape, the velocity of impact and a striker size. Additional parameters are the rotational velocity of the projectile and the incident angle of impact refer to the shield. For a shield made of two screens it is necessary to consider the standoff distance between the first shield and the second screen, near the wall, and also their comparative thicknesses. Shielding design: for protection of spacecrafts, shielding design is used to provide effective spacecraft protection from meteoroid and debris impacts. For the modelling purposes, normal impact of aluminum ball on the shield made of two aluminum screens, one should define the material and thickness of shield, and additionally for two-screen shields, the standoff distance between screens.


\bibliographystyle{plain}
\begin{thebibliography}{10}
\bibitem{Development of modified SPH approach for modeling of high-velocity impact.}
{\sc K.A. Alhussan and V.A. Babenko and I.M. Kozlov and A.S. Smetannikov.}. {Development of modified SPH approach for modeling of high-velocity impact.}. International Journal of Heat and Mass Transfer 55 (2012) 6340–6348..
\end{thebibliography}

\title{Algorithm for Interval Linear  Programming Involving Interval Constraints}
\tocauthor{Ibraheem Alolyan} \author{} \institute{}
\maketitle
\begin{center}
{\large Ibraheem Alolyan}\\
King Saud University\\
{\tt ialolyan05@yahoo.com}
\end{center}

\section*{Abstract}
In real optimization, we always come across problems with imprecise 
input values. Therefore, optimization problems are formulated 
under conditions of uncertainty. When realistic problems are formulated, 
a set of intervals may appear as coefficients in the objective 
function or the constraints. 
This paper considers linear programming problems with interval 
coefficients. For these problems, we cannot apply the technique of the 
classical linear programming directly. Therefore, a new method for 
solving linear programming problems with fuzzy parameters based on 
inequality relations  is investigated. 


        We use the useful method for ordering interval that was introduced in [2].    If we let   $\mathscr{I}$ be the set of all closed and bounded intervals on the real line $\mathbb R$, then the method is based on a measure function ($\mu-$function), that is defined from $\mathscr{I} \times \mathscr{I}$ to $\mathbb R$ to describe the relation between two intervals. 

Then we use the $\mu-$function to introduce the following method that will be used in this paper.
            \begin{proposition}
            \begin{enumerate}
            \item If $A$ and $B$ are real numbers, then $\mathscr{I}$ is the  ordinary inequality
            relation ``$\leq$" on the set of real numbers.
            \item $\mu (A,B) = 0 $ iff $A=B$.
            \item If $0 < \mu (A,B) \leq 1 $ then $ A \subset B$, (proper subset).
            \item If $1 < \mu (A,B) \leq 2 $ then $ A \bigcap B\neq \phi$; \\ Moreover,
            if  $1 < \mu (A,B) \leq 2 - \frac{2 \min\{r_A,r_B\}}{r_B+r_A}$, then  .
                  $$\begin{cases}
                   A \subset B   &\textrm{if  $r_B  \geq r_A$}\\
                   B \subset A,&\textrm{if $r_B  < r_A$}
                   \end{cases}$$
            \item $  \mu (A,B) > 2$ iff $ A \bigcap B = \phi$.
            \end{enumerate}
            \end{proposition}
To illustrate the efficiency of the proposed method, a numerical 
example is presented.

\bibliographystyle{plain}
\begin{thebibliography}{10}
\bibitem{J. Introduction Interval Computations}
{\sc Alefeld undefined and G. and Herzberger}. {J. Introduction Interval Computations}.   Academic Press, New Yourk. (1983)..

\bibitem{New Method for Comparing Closed Intervals}
{\sc Alolyan undefined and Ibraheem undefined}. {New Method for Comparing Closed Intervals}. Australian Journal of Mathematical Analysis and Applications, Volume 8, Issue 1 p: 1-6 (2011)..

\bibitem{Multiobjective programming in optimization of the interval objective function}
{\sc Ishihashi H and Tanaka M.}. {Multiobjective programming in optimization of the interval objective function}. European Journal of Operational Research (1990);48:219-25..

\bibitem{Rating and ranking multiple-aspect alternatives using fuzzy sets}
{\sc Baas SM and Kwakernaak H.}. {Rating and ranking multiple-aspect alternatives using fuzzy sets}. Automatica (1977);13:47-58..

\bibitem{Multiobjective programming in optimization of inteval objective functions - a generalized approach}
{\sc Chanas S. and Kuchta D.}. {Multiobjective programming in optimization of inteval objective functions - a generalized approach}. European Journal of Operational Research, Volume 94 Pages 594-598, (1996)..

\bibitem{Linear Programming with Fuzzy Variables}
{\sc Malek H.R. and Tata M. and Mashinchi M. and undefined undefined}. {Linear Programming with Fuzzy Variables}. Fuzzy Sets and Systems, 109, Pages 21-33, (2000)..
\end{thebibliography}

\title{a Paraxial Asymptotic Model for the Coupled Vlasov-Maxwel Problem in Electromagnetics}
\tocauthor{Franck Assous} \author{} \institute{}
\maketitle
\begin{center}
{\large \underline{Franck Assous}}\\
Ariel University Center, Israel\\
{\tt franckassous@netscape.net}
\\ \vspace{4mm}
{\large Joel Chaskalovic}\\
d'Alembert,  University Pierre and Marie Curie, Paris, France\\
{\tt joel.chaskalovic@upmc.fr,}
\end{center}

\section*{Abstract}
Charged particle beams and plasma physics problems are extensively used in Science and Technology. If we consider collisionless plasma or non-collisional beams, one of the most complete mathematical models is the time-dependent
Vlasov-Maxwell system of equations. 
However, the numerical solution of such models requires a large computational effort. Therefore, whenever possible, we have to take into account the
particularities of the physical problem to derive asymptotic approximate models
leading to cheaper simulations (see [1-3]). In this talk, we will consider the case of high energy short beams. A typical example is the transport of a bunch of highly relativistic
charged particles in the interior of a perfectly conducting hollow tube. This is usually modeled by the Vlasov equation, coupled with the Poisson or Maxwell equations. Numerical simulations are mostly performed using the particle-in-cell (PIC) method.\\


In this work, we will propose a new paraxial model that approximate the coupled  time-dependent Vlasov-Maxwell equations. Following [3-4] the model is derived by introducing a frame  which moves along the optical axis at the speed of light, so that the bunch of particles  is evolving slowly in this frame. Then, one considers a scaling of the equations which reflects the characteristics of the high energy short beam. This allows us to introduce a small parameter $\eta$ which denotes the ratio between the transverse characteristic velocity of the beam and the speed  of light. We finally use asymptotic expansion techniques to obtain a new paraxial model which is accurate up to fourth order in $\eta$. The simplicity of the obtained formulation allows to use a finite-difference or finite element discretization for the Maxwell equations. Hence, using a particle approximation for the Vlasov equation, a particle-in-cell technique can be easily developed. This approach promises to be very powerful in its ability to get an accurate and fast algorithm.


\bibliographystyle{plain}
\begin{thebibliography}{10}
\bibitem{The ARCTIC Charged Particle Beam Propagation Code}
{\sc M.A. Mostrom and D. Mitrovich and D.R. Welch}. {The ARCTIC Charged Particle Beam Propagation Code}. J. Comput. Physics, 128(2) (1996) 489-497.

\bibitem{On the paraxial approximation of the stationary Vlasov-Maxwell}
{\sc P. Degond and P.-A. Raviart }. {On the paraxial approximation of the stationary Vlasov-Maxwell}.  Math. Mod. Meth. Appl. Sci. 3(4) (1993) 513-562.

\bibitem{Paraxial approximation of ultrarelativistic intense beams}
{\sc G. Laval and S. Mas-Gallic and P.-A. Raviart}. {Paraxial approximation of ultrarelativistic intense beams}. Numer. Math. 69(1) (1994) 33-60.

\bibitem{A New Paraxial Asymptotic Model for the Relativistic Vlasov-Maxwell Equations}
{\sc F. Assous and J. Chaskalovic}. {A New Paraxial Asymptotic Model for the Relativistic Vlasov-Maxwell Equations}. to appear, http://dx.doi.org/10.1016/j.crme.2012.09.002 (2012).
\end{thebibliography}

\title{Some  Results on a Starlike  Log-harmonic Mappings of Order Alpha}
\tocauthor{Melike  Aydogan} \author{} \institute{}
\maketitle
\begin{center}
{\large Melike  Aydogan}\\
Isik University\\
{\tt melike.aydogan@isikun.edu.tr}
\end{center}

\section*{Abstract}
 Let H (D) be the linear space of all analytic functions defined on the open unit  disc $D ={z \in C : |z| < 1}$. A sense preserving  log-harmonic mapping is the solution of the non-linear  elliptic  partial differantial equation 
$$f_{z}=w(z)f_{z}(f_{z}/f)$$ 
where $w(z)\in H (D)$  is the second dilatation of f  such that $|w(z)|<1$ for all $z\in D$.\\
A  sense preserving log-harmonic mapping  is a solution of the non-linear  elliptic partial differential equation
\begin{equation}
\frac{f_{\overline{z}}}{\overline{f}}=w(z).\frac{f_{z}}{f}
\end{equation}
Where $w(z)$ the second dilatation of  f and  $w(z)\in H(D)$, $|w(z)| <1$ for every $z\in D$. It  has  been shown that  if f is non-vanishing  log-harmonic mapping,  then  f can be expressed as
\begin{equation} f(z)=h(z).\overline{g(z)}
\end{equation}
Where h(z) and  g(z) are analytic  in D  with  the  normalization $h(0)\neq0$, $g(0) = 1$. On the other  hand  if f vanishes at $z=0$, but it is not identically zero, then f admits  the following representation
\begin{equation} f(z)=z.z^{2\beta}h(z)\overline{g(z)}
\end{equation}
where $Re \beta>-\frac{1}{2} $, h(z) and g(z) are analytic  in the open disc D with the normalization $h(0)\neq0$ ,
$g(0) = 1$. $[1], [2]$.\\

In the present paper, we will give the extent of the idea, which was introduced by Z. Abdulhadi [1]. One of the interesting application of this extent idea is an investigation of the subclass of log-harmonic mappings for which starlike log-harmonic mappings of order alpha.


\bibliographystyle{plain}
\begin{thebibliography}{10}
\bibitem{Univalent Functions in H(D)}
{\sc  Z. Abdulhadi and D. Bshouty}. {Univalent Functions in H(D)}. Trans. Amer. Math. Soc., 305, pp.841-849, 1988..

\bibitem{One Pointed Univalent Log-harmonic Mappings}
{\sc Z. Abdulhadi and W. Hengartner}. {One Pointed Univalent Log-harmonic Mappings}.  J. Math. Anal. Apply. 203(2), pp. 333-351, 1996..

\bibitem{Univalent Functions}
{\sc A. W. Goodman}. {Univalent Functions}. Vol. 1, Mariner Publishing Company, Inc., Washington, New Jersey, 1983.
\end{thebibliography}

\title{A Certain Class of Starlike Log-harmonic Mappings}
\tocauthor{Melike Aydogan} \author{} \institute{}
\maketitle
\begin{center}
{\large \underline{Melike Aydogan}}\\
Isik University\\
{\tt melike.aydogan@isikun.edu.tr}
\\ \vspace{4mm}
{\large Yasar Polatoglu}\\
Istanbul Kultur University\\
{\tt y.polatoglu@iku.edu.tr}
\end{center}

\section*{Abstract}
\newcommand{\de}{\mathbb{D}}
Let H ($\de$) be the linear space of all analytic functions defined on the open unit  disc $\de$. 
A function $f: \de \rightarrow C$ is said to be log-harmonic in $\de$, if there is an $w(z)\in B$ such that $f$ is non-constant solution of the nonlinear elliptic partial differential equation; where $w(z)$ the second dilatation function, $w(z)\in \de$ is such that, $\left|w(z)\right|<1$ for all $z\in \de$. Univalent log-harmonic mappings have been studied extensively;  $[1]$, $[2]$, $[3]$, $[4]$, $[5]$. We note that the class of log-harmonic mappings is denoted by $S_{LH}$.\\
Denote by $ST^*_{LH}$ the set of all starlike log-harmonic mappings.
\begin{lemma}([7])Let $\phi(z)$ be regular in the unit disc $\de$, with $\phi(0)=0$. Then if $\left|\phi(z)\right|$ attains its maximum value on the disc $\left|z\right|=r$ at the point $z_{1}$, one has $z_{1}\phi'(z_{1})=k\phi(z_{1})$ for some $k\geq 1$.
\end{lemma}
In the present paper we investigate some properties of log-harmonic starlike mappings. For this aim we use the subordination principle and Lindelof Principle [6].

\bibliographystyle{plain}
\begin{thebibliography}{10}
\bibitem{Univalent functions in H (D)}
{\sc Z. Abdulhadi and D. Bshouty}. {Univalent functions in H (D)}. Trans. Amer. Math. Soc., 305 (1988), 841-849.

\bibitem{One pointed univalent logharmonic mappings}
{\sc Z. Abdulhadi and W. Hengartner}. {One pointed univalent logharmonic mappings}. J. Math. Anal. Apply. 203 (2) (1996), 333-351.

\bibitem{Starlike log-harmonic mappings of order alpha}
{\sc Z. Abdulhadi and Y. Abu Muhanna}. {Starlike log-harmonic mappings of order alpha}. JIPAM.Vol.7, Issue 4, Article 123 (2006).

\bibitem{Functions Star and Convex Univalent of Order alpha with Weight}
{\sc I. I. Barvin}. {Functions Star and Convex Univalent of Order alpha with Weight}. Doklady. Math., Vol 76. Issue 3 (2007) 848-850.

\bibitem{Univalent functions}
{\sc A. W. Goodman}. {Univalent functions}. Vol I, Mariner Publishing Company, Inc., Washington, New Jersey, 1983.

\bibitem{Starlike Majorants and Subordination}
{\sc Z. Lewandowski}. {Starlike Majorants and Subordination}. Annales Universitatis Marie-Curie Sklodowska, Section A, Vol XV (1961) 79-84.

\bibitem{Functions starlike and convex of order alpha}
{\sc I. S. Jack}. {Functions starlike and convex of order alpha}. J. London Math. Soc. (2) 3, (1971) no.2, 469-474.
\end{thebibliography}

\title{Optimal Control of Level Sets}
\tocauthor{Dmitri KuzminChristopher Basting} \author{} \institute{}
\maketitle
\begin{center}
{\large Dmitri Kuzmin}\\
University Erlangen-Nuremberg\\
{\tt kuzmin@math.fau.de}
\\ \vspace{4mm}
{\large \underline{Christopher Basting}}\\
University Erlangen-Nuremberg\\
{\tt christopher.basting@math.fau.de}
\end{center}

\section*{Abstract}
We present a new conservative level set method for numerical simulation of evolving  interfaces. A PDE-constrained optimization problem is formulated and solved in an iterative fashion. The proposed optimal control procedure constrains the level set function to satisfy a conservation law for the corresponding Heaviside function. The target value of the state variable is defined as the solution to the standard level set transport equation. The gradient of the control variable corrects the convective flux in the nonlinear state equation so as to enforce mass conservation while minimizing deviations from the target state. In addition to perfect mass conservation, the proposed methodology offers the possibility of maintaining the distance function property. The potential of the optimization-based approach is illustrated by numerical examples.

\bibliographystyle{plain}
\begin{thebibliography}{10}
\bibitem{An optimization-based approach to enforcing mass conservation in level set methods}
{\sc D. Kuzmin}. {An optimization-based approach to enforcing mass conservation in level set methods}. submitted to J. Comput. Appl. Math..

\bibitem{A minimization-based finite element formulation for interface-preserving level set reinitialization}
{\sc Ch. Basting and D. Kuzmin}. {A minimization-based finite element formulation for interface-preserving level set reinitialization}. submitted to Computing.
\end{thebibliography}

\title{A Hybrid Level Set / Front Tracking Approach}
\tocauthor{Steffen Basting} \author{} \institute{}
\maketitle
\begin{center}
{\large Steffen Basting}\\
University Erlangen-Nuremberg\\
{\tt basting@math.fau.de}
\end{center}

\section*{Abstract}
We present a hybrid level set / front tracking approach for the representation of sharp interfaces in finite element discretizations of fluid-structure interaction and two-phase flow problems. This hybrid approach makes use of an implicit representation of the interface by means of a level set function. The computational mesh is obtained from deforming a simplicial reference mesh such that the mesh is aligned to the implicitly described geometry providing an additional explicit representation of the interface while guaranteeing optimality of the mesh quality. The proposed method is based on a variational approach to optimal meshes and leads to a fully automated mesh optimization procedure which retains mesh connectivity and thus can be easily integrated in an existing mesh moving code. Due to the hybrid interface representation, the geometrical flexibility of conventional front tracking / mesh moving approaches is enhanced.

  We demonstrate and evaluate the proposed framework by applying it to example problems from two-phase flow applications and fluid-structure interaction problems.


\bibliographystyle{plain}
\begin{thebibliography}{10}
\bibitem{A hybrid level set - front tracking finite element approach for fluid-structure interaction and two-phase flow applications}
{\sc S. Basting and M. Weismann}. {A hybrid level set - front tracking finite element approach for fluid-structure interaction and two-phase flow applications}. submitted to: J. Comput. Phys. (10/2012).
\end{thebibliography}

\title{Algebraic Flux Correction and Hp-adaptivity for Hyperbolic Conservation Laws}
\tocauthor{Melanie Bittl} \author{} \institute{}
\maketitle
\begin{center}
{\large \underline{Melanie Bittl}}\\
University Erlangen-Nuremberg\\
{\tt bittl@math.fau.de}
\\ \vspace{4mm}
{\large Dmitri Kuzmin}\\
University Erlangen-Nuremberg\\
{\tt kuzmin@am.uni-erlangen.de}
\end{center}

\section*{Abstract}
This talk is concerned with the design of $hp$-adaptive algebraic flux correction schemes for hyperbolic conservation laws.
The proposed approach is based on a continuous Galerkin approximation with unconstrained high-order elements in smooth regions and constrained $P_1/Q_1$ elements in the neighborhood of steep fronts.
The local mesh size $h$ and polynomial degree $p$ are chosen by the reference solution approach 
whereby the polynomial degree $p$ can be increased in smooth elements only. These smooth elements are determined by 
a hierarchical smoothness indicator based on discontinous higher-order reconstructions.
 The discrete maximum principle for linear/bilinear finite elements is enforced using a linearized flux-corrected transport 
(FCT) algorithm. 
The algorithm is implemented in the open-source software package Hermes. The use of hierarchical data structures that support 
arbitrary level hanging nodes makes the extension of FCT to $hp$-FEM relatively straightforward. 
The method is presented in the context of a scalar transport equation and an extension to hyperbolic systems is discussed. 
A numerical study is performed for the compressible Euler equations in 2D.

\bibliographystyle{plain}
\begin{thebibliography}{10}
\bibitem{An hp-adaptive flux-corrected transport algorithm for continuous finite elements}
{\sc M. Bittl and D. Kuzmin}. {An hp-adaptive flux-corrected transport algorithm for continuous finite elements}. J.  Computing, 2012, http://dx.doi.org/10.1007/s00607-012-0223-y.

\bibitem{Higher-Order Finite Element Methods}
{\sc P. Solin and K. Segeth and I. Dolezel}. {Higher-Order Finite Element Methods}. Chapman and Hall / CRC Press, 2003.

\bibitem{Algebraic Flux Correction I}
{\sc D. Kuzmin}. {Algebraic Flux Correction I}. In:  D. Kuzmin, R. L\"ohner, S. Turek (eds)  Flux-Corrected Transport,  Springer 2012, 2nd edition.

\bibitem{Algebraic Flux Correction II}
{\sc D. Kuzmin and M. M\"oller and M. Gurris}. {Algebraic Flux Correction II}. In:  D. Kuzmin, R. L\"ohner, S. Turek (eds) Flux-Corrected Transport, Springer 2012, 2nd edition .

\bibitem{A parameter-free smoothness indicator for high-resolution finite  element schemes}
{\sc D. Kuzmin and F. Schieweck}. {A parameter-free smoothness indicator for high-resolution finite  element schemes}. Submitted to the CEJM Topical Issue "Numerical Methods for Large Scale Scientific Computing" in February 2012..
\end{thebibliography}

\title{The Distributed and Unified Numerics Environment (DUNE)}
\tocauthor{Markus Blatt} \author{} \institute{}
\maketitle
\begin{center}
{\large Markus Blatt}\\
Dr. Markus Blatt - HPC-Simulation-Software \& Services\\
{\tt markus@dr-blatt.de}
\end{center}

\section*{Abstract}
In this talk we present the Distributed and Unified Numerics Environment (DUNE). It is a software framework for the parallel solution of partial differential equations with grid-based methods. Unlike other frameworks it is not built 
around fixed mesh data structures, but represents a fine grained interface to several supported grids (simplicial, structured, unstructured,...). Algorithms, e.g. finite element discretizations, 
based on the interface can use different meshes without loosing performance due to generic programming techniques. This saves the scientist precious time.  

We present serveral parallel applications realized with DUNE and show their scalability on supercomputers. Special emphasis will be put on the performance of the parallel iterative solvers within DUNE. 

We conclude the talk with an overview of exemplary projects based upon DUNE, such as Dune-PDELab, the Open Porous Media Simulator  (OPM), and DuMux (DUNE for multi-phase, component, scale, physics, ..).

\bibliographystyle{plain}
\begin{thebibliography}{10}
\bibitem{A Generic Grid Interface for Parallel and Adaptive Scientific Computing. Part I Abstract Framework}
{\sc P. Bastian and M. Blatt and A. Dedner and C. Engwer and R. Klöfkorn and M. Ohlberger and O. Sander}. {A Generic Grid Interface for Parallel and Adaptive Scientific Computing. Part I: Abstract Framework}. Computing, 82(2-3), 2008, pp. 103-119.

\bibitem{A Generic Grid Interface for Parallel and Adaptive Scientific Computing. Part II Implementation and Tests in DUNE}
{\sc P. Bastian and M. Blatt and A. Dedner and C. Engwer and R. Klöfkorn and R. Kornhuber and M. Ohlberger and O. Sander.}. {A Generic Grid Interface for Parallel and Adaptive Scientific Computing. Part II: Implementation and Tests in DUNE}. Computing, 82(2-3), 2008, pp. 121-138.

\bibitem{The Iterative Solver Template Library}
{\sc M. Blatt and P. Bastian}. {The Iterative Solver Template Library}. In B. Kåström, E. Elmroth, J. Dongarra and J. Wasniewski, Applied Parallel Computing. State of the Art in Scientific Computing. Volume 4699 of Lecture Notes in Scientific Computing, pages 666-675. Springer, 2007.
\end{thebibliography}

\title{Two-dimensional Numerical Simulations of Normal Drop Impact on a Thin Liquid Film in Presence of a Boundary Layer}
\tocauthor{Paola Brambilla} \author{} \institute{}
\maketitle
\begin{center}
{\large \underline{Paola Brambilla}}\\
Politecnico di Milano\\
{\tt paola.brambilla@aero.polimi.it}
\\ \vspace{4mm}
{\large Andrea Cristina}\\
Politecnico di Milano\\
{\tt andrea1.cristina@mail.polimi.it}
\\ \vspace{4mm}
{\large Alberto Guardone}\\
Politecnico di Milano\\
{\tt alberto.guardone@polimi.it}
\end{center}

\section*{Abstract}
The present work proposes a two-dimensional numerical investigation of a single drop impact on a thin liquid film in a flow field characterized by the presence of a boundary layer \cite{Boundary layer flow of air past solid surfaces in the presence of rainfall}. The focus is on impacts in the splashing 
regime \cite{A numerical study on the mechanism of splashing}, which are characterized by the formation of a crown shape whose evolution is governed by inertial, surface tension and viscous effects.
Several simulations are performed in order to describe the post-impact dynamics of the liquid phase. In particular, we are interested in assessing the effects of the flow on the secondary droplets generating after the impact, in
order to estimate the fluid percentage which spreads over the free surface and the one which leaves the impact region. The volume of fluid with normal velocity higher than a known one is computed to describe quantitatively the phenomenon in exam.
Furthermore, we estimate the crown radius and height.
The numerical code used to perform the numerical simulations is part of the open-source suite for the computational fluid dynamics (CFD) OpenFOAM
which implements the Volume-Of-Fluid (VOF) method.

\bibliographystyle{plain}
\begin{thebibliography}{10}
\bibitem{Boundary layer flow of air past solid surfaces in the presence of rainfall}
{\sc J.A. Tsamopoulos and D.N. Smyrnaios and N.A. Pelekasis}. {Boundary layer flow of air past solid surfaces in the presence of rainfall}. J. of Fluid Mechanics, 425 (2000) 79-110.

\bibitem{A numerical study on the mechanism of splashing}
{\sc M. Rieber and A. Frohn}. {A numerical study on the mechanism of splashing}. Int. J. of Heat and Fluid Flow, 20 (1999) 455-461.
\end{thebibliography}

\title{Hartree Fock Calculations for Atoms and Small Molecules Using  a Three Dimensional  Finite Element Basis}
\tocauthor{Moritz Braun} \author{} \institute{}
\maketitle
\begin{center}
{\large Moritz Braun}\\
University of South Africa\\
{\tt moritz.braun@gmail.com}
\end{center}

\section*{Abstract}
Hartee Fock calculations for molecules have traditionally been done using basis sets 
of functions centered on  the molecules, such as Gaussian type or Slater type orbitals. While these methods are quite successfull they  also have their disadvantages. Firstly, the optimal choice of the basis set can become rather delicate and the Gaussian functions do not have the appropriate behaviour both at the nuclei as  well as at large distances.
Thus it is desirable, to perform Hartree Fock calculations with a different basis set, that does avoid the problems above.\\
In this contribution results of Hartree Fock calculations  for atoms and small molecules using the method of finite elements in three dimensions[1] using elements of 4th order are presented and the approach is  compared with that  of recent Hartree Fock calculations employing a linear finite element basis [2].  Details about the computational procedure employed to deal with the challenges of the resulting very large scale generalized eigen value problems are also discussed, in particular the use the factorization trick which allows an efficient representation of the Greens function[3].

\bibliographystyle{plain}
\begin{thebibliography}{10}
\bibitem{ Finite element calculations for systems with multiple Coulomb centers }
{\sc M. Braun}. { Finite element calculations for systems with multiple Coulomb centers }. Journal of Computational and Applied Mathematics 236 (2012) 4840–4845.

\bibitem{A divide and conquer real space finite-element Hartree-Fock method }
{\sc R. Alizadegan and K.J. Hsia and T.J. Martinez}. {A divide and conquer real space finite-element Hartree-Fock method }. J. Chem. Phys. 132 034101 (2010); doi: 10.1063/1.3290949.

\bibitem{Different approaches to the numerical solution of the 3D Poisson equation implemented in Python}
{\sc M. Braun}. {Different approaches to the numerical solution of the 3D Poisson equation implemented in Python}. Computing DOI 10.1007/s00607-013-0300-x.
\end{thebibliography}

\title{DPG: A Robust, Higher Order Adaptive Method for Convection-dominated Diffusion Problems}
\tocauthor{Jesse Chan} \author{} \institute{}
\maketitle
\begin{center}
{\large Jesse Chan}\\
Institute for Computational Engineering and Sciences at UT Austin\\
{\tt jchan@ices.utexas.edu}
\\ \vspace{4mm}
{\large Leszek  Demkowicz}\\
Institute for Computational Engineering and Sciences at UT Austin\\
{\tt leszek@ices.utexas.edu}
\end{center}

\section*{Abstract}
The Discontinuous Petrov-Galerkin (DPG) method is a new minimum-residual method, where the residual is measured not in standard norms, but in the dual norm on the space of test functions.  In choosing to minimize this specific choice of residual, the DPG method can also be interpreted as providing a set of optimal test functions for a given problem, and avoids some common problems found in standard minimum-residual and least-squares methods.  Additionally, the DPG method is stable for meshes of arbitrary element topology and order, and is thus well-suited for $hp$-adaptivity.  

Naive methods for convection-diffusion problems typically suffer from a lack of robustness in the diffusion parameter $\epsilon$, manifesting in the degradation of the solution on a fixed mesh as $\epsilon$ decreases.  DPG optimal test functions are determined by solving a local auxiliary projection problem with respect to a specific inner product, which determines a norm on the space of test functions.  By proper choice of this test norm, we will demonstrate that DPG provides a robust method for convection-dominated diffusion problems.  Coupled with a built-in error indicator, this provides an automatic $hp$-adaptive method for difficult singular perturbation problems.

Numerical examples confirming will be given for the convection-diffusion problem, a nonlinear Burgers' equation, and the compressible Navier-Stokes equations.  

\bibliographystyle{plain}
\begin{thebibliography}{10}
\bibitem{A class of discontinuous Petrov-Galerkin methods II Optimal test functions}
{\sc L. Demkowicz and J. Gopalakrishnan}. {A class of discontinuous Petrov-Galerkin methods II: Optimal test functions}. Num. Meth. for Partial Diff. Eq. 27 (2011), 70-105.

\bibitem{Robust DPG method for convection-dominated diffusion problems II A natural inflow condition}
{\sc J. Chan and N. Heuer and T. Bui-Thanh and and L. Demkowicz}. {Robust DPG method for convection-dominated diffusion problems II: A natural inflow condition}. Technical Report 21, ICES, June 2012. submitted to Comput. Math. Appl..

\bibitem{Robust DPG method for convection-dominated diffusion problems}
{\sc L. Demkowicz and N. Heuer}. {Robust DPG method for convection-dominated diffusion problems}. Technical Report 11-33, ICES, 2011.

\bibitem{Discontinuous Petrov-Galerkin method based on the optimal test space norm for one-dimensional transport problems}
{\sc A. Niemi and N. Collier and and V. Calo}. {Discontinuous Petrov-Galerkin method based on the optimal test space norm for one-dimensional transport problems}. Procedia CS 4 (2011), 1862-1869.

\bibitem{Adaptivity and variational stabilization for convection-diffusion equations}
{\sc W. Dahmen and A. Cohen and and G. Welper}. {Adaptivity and variational stabilization for convection-diffusion equations}. ESAIM: Mathematical Modelling and Numerical Analysis, 46 (5): 1247-1273, 2012.
\end{thebibliography}

\title{A Minimum Sobolev Norm Numerical Technique for PDEs}
\tocauthor{Shivkumar Chandrasekaran} \author{} \institute{}
\maketitle
\begin{center}
{\large \underline{Shivkumar Chandrasekaran}}\\
University of California, Santa Barbara\\
{\tt shiv@ece.ucsb.edu}
\\ \vspace{4mm}
{\large Hrushikesh Mhaskar}\\
Claremont Graduate University\\
{\tt hmhaska@gmail.com}
\end{center}

\section*{Abstract}
We present a method for the numerical solution of PDEs based on
finding solutions that minimize a certain Sobolev norm. Fairly
standard compactness arguments establish convergence. The method
prefers that the PDE is presented in first order form. A single short
Octave code is used to solve problems that range from first-order
Maxwell's equations to fourth-order bi-harmonic problems on
complicated geometries. The method is high-order convergent even on
complex curved geometries. Our method has its roots in generalized
Birkhoff interpolation. Let $x_i$ denote $N$ points in $R^d$. Let $f$
be an unknown function from $R^d$ to $R^q$. Given $N$ point-wise
(possibly vector-valued) linear observations of $f$: \begin{equation}
g(x_i) = \sum_{j\in \mathbb{N}^d}^{\|j\|_1\leq M} A_j(x_i) \,
\partial^{\|j\|_1}_j f (x_i), \label{eqn:obs} \end{equation} the
problem is to compute an approximation to $f$. The above problem is
well-posed for some special choices of the matrix coefficients $A_j$
and points $x_i$ in the sense that, as $N$ approaches infinity, only
the true solution can satisfy all the observations. In particular the
numerical solution of linear PDEs can be posed as generalized Birkhoff
interpolation problems. The classical approach to the above problem is
to expand the unknown solution as a finite linear combination of basis
functions, such that the constraints~(\ref{eqn:obs}) become a system
of square (or skinny) equations for the unknown coefficients. Then the
(least-squares) solution of these equations is taken to be the
computed solution for the unknown function. Our approach is almost the
same, except that we pick more expansion coefficients so that we
obtain a \textsl{fat} system of linear equations. As our solution we
pick the one that minimizes a certain Sobolev norm.  Assuming the true
solution satisfies all the interpolation constraints and has a finite
Sobolev norm, we can establish (see [1] for the special case of
classical interpolation) that there is a uniform bound on the Sobolev
norm of our computed solution independent of the number of
constraints. It then follows from standard compactness arguments that
our computed solution will converge to the true solution as the number
of interpolation conditions increase.


\bibliographystyle{plain}
\begin{thebibliography}{10}
\bibitem{Minimum Sobolev norm interpolation with trigonometric polynomials on the torus}
{\sc S. Chandrasekaran and H. Mhaskar and K. R. Jayaraman}. {Minimum Sobolev norm interpolation with trigonometric polynomials on the torus}. Accepted for publication in J. Comput. Phys. 2013.
\end{thebibliography}

\title{A Posteriori Error Analysis in Numerical Approximations of PDE's: A Pilot Study Using Data Mining Techniques}
\tocauthor{Joel Chaskalovic} \author{} \institute{}
\maketitle
\begin{center}
{\large \underline{Joel Chaskalovic}}\\
D'Alembert, University Pierre and Marie Curie\\
{\tt joel.chaskalovic@upmc.fr}
\\ \vspace{4mm}
{\large Franck Assous}\\
Ariel University Center\\
{\tt franckassous@netscape.net}
\end{center}

\section*{Abstract}
We introduced in [1] a new methodology based on data mining techniques for numerical approximation analysis. Our aim is to extend this methodology to a posteriori error analysis in numerical approximations of partial differential equations. Let us consider a real system (S) modeled by a set of partial differential equations (E). Generally, the solution of such a system is carried out by numerical approximations methods. Regarding the production of these approximations, it is essential to consider the sources of errors which spoil the accuracy of the description and the understanding of the real system (S). In this work, we define four types of errors : the modeling error, the approximation error, the parametrization error and the discretization error. Here, we focus our attention to a test case devoted to the discretization error defined by: For a given family of approximations methods, we consider two numerical methods of this family, says MN1 and MN2. The discretization error is defined as the error due to the difference of order between MN1 and MN2.
As an example, we considered numerical solutions to Vlasov Maxwell equations, computed by an asymptotic model [2], and numerically discretized by P1 and P2 finite elements. A priori, the Bramble-Hilbert theorem claims that the results obtained by the finite elements P2 will be more precise than those computed by finite elements P1. However, the estimations of the approximation error contain multiplicative constants, unknown or difficult to estimate. As a consequence, we looked for circumstances for which the finite element method P1 would be locally "equivalent" or more "accurate" than the finite element method P2. Then, we sought to qualify such situations by data mining techniques like segmentation by decision trees [3]. We showed that it may  be possible to heuristically precise the classical Bramble Hilbert's theorem, that gives a global error estimate, whereas our approach may give an local error estimate.

\bibliographystyle{plain}
\begin{thebibliography}{10}
\bibitem{Data mining techniques for scientific computing Application to asymptotic paraxial approximations to model ultra-relativistic particles}
{\sc F. Assous and J. Chaskalovic}. {Data mining techniques for scientific computing: Application to asymptotic paraxial approximations to model ultra-relativistic particles}. J. Comput. Phys., 230, pp. 4811-4827 (2011).

\bibitem{Paraxial approximation of ultrarelativistic intense beams}
{\sc G. Laval and S. Mas-Gallic and P.-A. Raviart}. {Paraxial approximation of ultrarelativistic intense beams}. Numer. Math., 69(1), 33-60 (1994).

\bibitem{Data Mining with Decision Trees Theory and Applications}
{\sc L. Rokach and O. Maimon}. {Data Mining with Decision Trees: Theory and Applications}. World Scientific Publishing Company , (2001)..
\end{thebibliography}

\title{Study of the Stochastic Inviscid Burgers Equation With the Joint Response-excitation PDF Equation}
\tocauthor{Heyrim Cho} \author{} \institute{}
\maketitle
\begin{center}
{\large \underline{Heyrim Cho}}\\
Brown university\\
{\tt heyrim\_cho@brown.edu}
\\ \vspace{4mm}
{\large Daniele Venturi}\\
Brown university\\
{\tt daniele\_venturi@brown.edu}
\\ \vspace{4mm}
{\large George Karniadakis}\\
Brown university\\
{\tt george\_karniadakis@brown.edu}
\end{center}

\section*{Abstract}
This paper aims to study the joint response-excitation PDF (REPDF) evolution equation of the stochastic inviscid Burgers equation by using the discontinuous Galerkin method. We initially focus on the validity of the PDF equation and the aspects of probability density solution when discontinuity occurs in the physical space. Then, applications of the Burgers equation involving colored random noise is simulated, which has been recently enabled by the REPDF approach. The discontinuous Galerkin method is demonstrated to be effective in resolving the local and discontinuous dynamics of the solution. Moreover, the method is combined with probabilistic collocation method and matrix exponential techniques to compute various types of random excitations. 

\bibliographystyle{plain}
\begin{thebibliography}{10}
\bibitem{Adaptive discontinuous Galerkin method for response-excitation PDF equations}
{\sc H. Cho and D. Venturi and G. E. Karniadakis}. {Adaptive discontinuous Galerkin method for response-excitation PDF equations}. SIAM J. Sci. Comput. (2013) submitted.

\bibitem{New evolution equations for the joint response-excitation probability density function of stochastic solutions to first-order nonlinear PDEs}
{\sc D. Venturi and G. E. Karniadakis}. {New evolution equations for the joint response-excitation probability density function of stochastic solutions to first-order nonlinear PDEs}. J. Comput. Phys. 231 (2012) 7450-7474.

\bibitem{A computable evolution equation for the probability density function of stochastic dynamical systems}
{\sc D. Venturi and T. P. Sapsis and H. Cho and G. E. Karniadakis}. {A computable evolution equation for the probability density function of stochastic dynamical systems}. Proc. Roy. Soc. A 468 (2012) 759–783.
\end{thebibliography}

\title{Time-dependent Karhunen-Loeve Decomposition Methods for SPDEs: Dynamically-Orthogonal and Bi-Orthogonal Conditions and Its Equivalence}
\tocauthor{Minseok Choi} \author{} \institute{}
\maketitle
\begin{center}
{\large \underline{Minseok Choi}}\\
Brown University\\
{\tt minseok\_choi@brown.edu}
\\ \vspace{4mm}
{\large Themistoklis Sapsis}\\
New York University\\
{\tt sapsis@cims.nyu.edu}
\\ \vspace{4mm}
{\large George Karniadakis}\\
Brown University\\
{\tt george\_karniadakis@brown.edu}
\end{center}

\section*{Abstract}
The Karhunen-Loeve decomposition method provides a low-dimensional
representation of the stochastic partial differential equations
(SPDEs) as it is optimal in the sense that it minizes the total mean squared error. Dynamically orthogonal (DO) field equations based on time-dependent Karhunen-Loeve decomposition has been developed. Then similar but slightly different time-dependent method called bi-orthogonal (BO) method has been proposed. We show the BO
evolution equations and compare it with the DO approach. We also show that BO and DO are equivalent in the sense that one method can be transformed dynamically through the matrix differential equation to another and vice-versa but BO is numerically more accurate than DO,
in particular involving a large number of modes in nonlinear
problems. Several examples are presented to illustrate the DO and BO and their equivalence.

\bibliographystyle{plain}
\begin{thebibliography}{10}
\bibitem{Dynamically orthogonal field equations for continuous stochastic dynamical systems}
{\sc T. Sapsis and P. Lermusiaux}. {Dynamically orthogonal field equations for continuous stochastic dynamical systems}. Physica D. 238(2009) 2347-2360.

\bibitem{M. Choi T. Sapsis G. Karniadakis}
{\sc Evolution Equations for Stochastic Modes in Polynomial Chaos Expansions: A Convergence Study for SPDEs}. {M. Choi, T. Sapsis, G. Karniadakis}. in review.
\end{thebibliography}

\title{Staggered Discontinuous Galerkin Methods for the Maxwell's Equations}
\tocauthor{Eric Chung} \author{} \institute{}
\maketitle
\begin{center}
{\large Eric Chung}\\
The Chinese University of Hong Kong\\
{\tt tschung@math.cuhk.edu.hk}
\end{center}

\section*{Abstract}
In this talk, a new type of staggered discontinuous Galerkin methods for the three dimensional Maxwell's equations will be presented.
The spatial discretization is based on staggered Cartesian grids so that many good properties are obtained. 
First of all, our method has the advantages that the numerical solution preserves the electromagnetic energy and automatically fulfills a discrete version of the Gauss law.
Moreover, the mass matrices are diagonal, thus time marching is explicit and is very efficient.
Our method is high order accurate and the optimal order of convergence is rigorously proved.
It is also very easy to implement due to its Cartesian structure 
and can be regarded as a generalization of the classical Yee's scheme as well as the quadrilateral edge finite elements. 
Furthermore, a superconvergence result, that is the convergence rate is one order higher at interpolation nodes, 
is proved.
Numerical results are shown to confirm our theoretical statements, and
applications to problems in unbounded domains with the use of PML are presented. 
A comparison of our staggered method and non-staggered method is carried out
and shows that our method has better accuracy and efficiency. 


\bibliographystyle{plain}
\begin{thebibliography}{10}
\bibitem{Staggered discontinuous Galerkin methods for the Maxwell's equations}
{\sc Eric T. Chung }. {Staggered discontinuous Galerkin methods for the Maxwell's equations}. J. Comput. Phys..
\end{thebibliography}

\title{Stochastic-Probabilistic Modeling of the Wind Turbine Power Output}
\tocauthor{Radian BeluIrina Ciobanescu Husanu} \author{} \institute{}
\maketitle
\begin{center}
{\large Radian Belu}\\
Drexel University\\
{\tt rb544@drexel.edu}
\\ \vspace{4mm}
{\large \underline{Irina Ciobanescu Husanu}}\\
Drexel University\\
{\tt inc22@drexel.edu}
\end{center}

\section*{Abstract}
Developing appropriate models for assessing and predicting the quality of power for any renewable energy source is important throughout the energy industry. Quality of power modeling is particularly important with regard to wind energy as the construction of new wind farms is growing rapidly compared with other renewable energy systems [1]. The intermittence in the wind resource prompted researchers to focus on probabilistic and/or stochastic models in order to assess the energy available from a wind power plant. A stochastic-probabilistic model is proposed to reproduce the power production of a wind energy conversion system (WECS) from any given wind measurement. The power quality of a wind turbine is determined by many factors but time-dependent variation in the wind velocity is arguably the most important. The atmospheric wind in which WECS operate is a complex process and fully understood [1, 2]. Wind turbulence cannot be controlled or reduced at will. The site topography and surface roughness have a direct impact on the turbulence intensity, making some sites less turbulent than others, mainly offshore sites. Our rising dependence on wind energy calls for a better understanding of such phenomena, as well as reliable models for wind turbine design, operation, and grid integration. However, the staggering level of complexity of turbulent effects makes modeling a complex task. The power grid reliability impacts could be significant when a large amount of variable wind generation is integrated with the electric power system. The widely used deterministic reliability assessment method is invalid when modeling intermittency of wind energy sources. The developed stochastic model also accounts for the inter-temporal and spatial dependencies of multi-area wind power production.

\bibliographystyle{plain}
\begin{thebibliography}{10}
\bibitem{Wind Energy Conversion and Analysis}
{\sc R. Belu}. {Wind Energy Conversion and Analysis}. Encyclopedia of Energy Engineering and Technology (online) (Eds. S. Anwar et al.) (2013) in press.

\bibitem{Small and Large Scale Fluctuations in Atmospheric Wind Speeds}
{\sc F. Bottcher and S. Barth and J. Peinke}. {Small and Large Scale Fluctuations in Atmospheric Wind Speeds}. Stochastic Environmental Research and Risk Assessment 21 (2007) 299 - 308.
\end{thebibliography}

\title{Integral Analysis and Simulation of Wind Turbine Rotors}
\tocauthor{Radian BeluIrina Ciobanescu Husanu} \author{} \institute{}
\maketitle
\begin{center}
{\large Radian Belu}\\
Drexel University\\
{\tt rb544@drexel.edu}
\\ \vspace{4mm}
{\large \underline{Irina Ciobanescu Husanu}}\\
Drexel University\\
{\tt inc22@drexel.edu}
\end{center}

\section*{Abstract}
Wind power is a fast-growing source of non-polluting, renewable energy with vast potential and wind energy is part of the solution to the world's energy demands. Wind turbines are designed to extract mechanical energy from the wind's kinetic energy through an aerodynamic rotor [1]. However, current wind technology must be improved before the potential of wind power can be fully realized. The wind turbines are classified based on their orientation of the rotation axis. Computational analysis and modeling for wind turbine is notably versatile, as it avoids the costs of experiments and physical prototypes, and it can easily simulate and manage real-time scenarios to seek the best and the most performant solutions [1, 2]. The following analysis and simulation is focusing on the vertical axis wind turbines. The model and analysis input/known variables are the wind velocity, the type of the wind turbine blades, and the radius of the vertical axis wind turbine. The fluid velocity and angle of attack on the turbine blades are subsequently determined from the analysis and simulation. From these variables the lift and drag forces (as well as the lift and drag coefficients) acting on the rotor blades for each of the rotor position are determined and computed. Later once the drag and lift forces are computed the torque and generated power can be computed to estimate and evaluate the turbine power coefficient. The flow is modeled based on a control volume approach, defined based on the variations in the wind velocity. The analysis can be extended to the global analysis of the flow in a wind farm.

\bibliographystyle{plain}
\begin{thebibliography}{10}
\bibitem{Wind Energy Conversion and Analysis }
{\sc R. Belu}. {Wind Energy Conversion and Analysis }. Encyclopedia of Energy Engineering and Technology (online) (Eds: S. Anwar et al.), 2013 (in press).

\bibitem{Double-multiple Streamtube Model for Darriesu Wind Turbine}
{\sc I. Paraschivoiu}. {Double-multiple Streamtube Model for Darriesu Wind Turbine}. NASA Conference Publication 2185, 981.
\end{thebibliography}

\title{GPU-Accelerated Algebraic Multigrid for Industrial Applications}
\tocauthor{Jonathan Cohen} \author{} \institute{}
\maketitle
\begin{center}
{\large Jonathan Cohen}\\
NVIDIA\\
{\tt jocohen@nvidia.com}
\end{center}

\section*{Abstract}
Many important industrial applications, such as automotive engineering, thermal simulations, and oil reservoir simulation, require the solution of large elliptic partial differential equations.  Typically, solving these equations requires the solution of one or more large sparse linear system of equations via an iterative solver.  In order to achieve acceptable convergence speeds, the iterative solver must be accelerated via some type of multilevel scheme.  Algebraic multigrid (AMG) is the method of choice for many of these applications due to its flexibility and high-performance.  A number of groups have looked at how to efficiently parallelize AMG algorithms to run on modern massively parallel architectures such as GPUs or on very large clusters.

NVIDIA has been developing a library of high-performance parallel sparse iterative linear solvers, with an emphasis on multilevel and multigrid methods.  In this presentation, I will provide an overview of the library’s design and outline many of the challenges we have faced in balancing numerical behavior against parallel scalability.  Our library has been integrated into ANSYS Fluent 14.5, and will be released as a fully supported feature in the upcoming Fluent 15.  I will describe the collaboration between ANSYS and NVIDIA, and present benchmarking results across a variety of test problems from CFD and other fields.  Finally, I will talk about our future plans and discuss some of the open research problems in the area of algebraic multigrid on massively parallel processors.

\bibliographystyle{plain}
\begin{thebibliography}{10}
\bibitem{Exposing Fine-Grained Parallelism in Algebraic Multigrid Methods}
{\sc W. Bell and S. Dalton and L. Olson}. {Exposing Fine-Grained Parallelism in Algebraic Multigrid Methods}. SIAM J. Sci. Comput., 34(4), C123–C152. 2012.

\bibitem{Parallel unsmoothed aggregation algebraic multigrid algorithms on GPUs}
{\sc J. Brannick and Y. Chen and X. Hu and L. Zikatanov}. {Parallel unsmoothed aggregation algebraic multigrid algorithms on GPUs}. http://arxiv.org/abs/1302.2547. 2013.

\bibitem{NVAMG REFERENCE MANUAL}
{\sc NVIDIA Corporation}. {NVAMG REFERENCE MANUAL}. V2.0, 2013.

\bibitem{Reducing complexity in parallel algebraic multigrid preconditioners}
{\sc H. De Sterck and U. M. Yang and J.J. Hey}. {Reducing complexity in parallel algebraic multigrid preconditioners}. SIAM Journal on Matrix Analysis and Applications, 27, (2006), pp. 1019-1039..
\end{thebibliography}

\title{Hybrid Clouds as Hydro-Meteorological Simulation Enablers}
\tocauthor{A. Quarati, A. Clematis, E. Danovaro,  A. Galizia, D. D'Agostino} \author{} \institute{}
\maketitle
\begin{center}
{\large A. Quarati, A. Clematis, E. Danovaro,  A. Galizia, D. D'Agostino}\\
CNR-IMATI\\
{\tt \{quarati,clematis,danovaro,galizia,dago\}@ge.imati.cnr.it}
\end{center}

\section*{Abstract}
Extreme precipitation and flooding events are among the greatest risks to human life and property, with significant societal and economic implications. To cope with these issues a Hydro Meteorological (HM) research major objective is to enable the acceleration and the integration of advances in the everyday forecasts thus improving the environment protection. Actually, predicting weather and climate and their impacts is a crucial task both for HM research groups as well for civil protection departments. Moreover preventing hazards such as floods and landslides needs to address manifold issues that involve not only HM scientists but requires a strong connection and collaboration with the ICT community to explore new technological solutions and approaches. Run prediction systems (e.g. WRF, RIBS, etc) in a timely and efficient way, both for research and even more for possible operational applications, usually requires the use of HPC resources that are both costly and not always easily accessible. In particular the Weather Research and Forecasting (WRF) model is a numerical weather prediction and atmospheric simulation system developed to advance the understanding and prediction of mesoscale weather and accelerate the transfer of research advances into operations. WRF is considered the reference model, and reflects flexible, state-of-the-art, portable code that is efficient in computing environments ranging from massively-parallel supercomputers to laptops. The computational capacity of the underline system may actually represent a limitation in the every-day-life work of each HM scientist. 
A key factor of Cloud computing is represented by its on-demand, pay-per-use approach towards virtualized and distributed ICT solutions, in contrast to the creation and the maintenance of expensive, tightly pre-configured IT infrastructures, needed to grant analogous services at the same level of business continuity. Hybrid Clouds integrating internal (protected) and external (commercial) service, couple the scalability offered by public Clouds with the greater control supplied by private ones. A Cloud broker, acting as an intermediary between users and providers of public Cloud services -  may support HM customers in the selection of the most suitable computational platform depending their simulation objectives, optionally adding the provisioning of dedicated services with higher levels of quality. Fostered by the research activities of the FP7 EU Distributed Research Infrastructure for Hydro-Meteorology project (2012-2015) [1], in the paper we analyze the performance behavior of a brokering tool for Hybrid Clouds in adequately responding to the operational constraints raised by different instances of  some HM models, that, as in the case of WRF, may have various degree of urgency requirements. 


\bibliographystyle{plain}
\begin{thebibliography}{10}
\bibitem{DRIHM Distributed Research Infrastructure for Hydro-Meteorology}
{\sc Clematis A. and D'Agostino D. and Danovaro E. and Galizia A. and Quarati A. and Parodi A. and Rebora N. and Bedrina T. and Kranzlmueller D. and Schiffers M. and Jagers B. and Harpham Q. and Cros P.H.}. {DRIHM: Distributed Research Infrastructure for Hydro-Meteorology}. In: 7th International Conference on System of Systems Engineering (SOSE2012) Proceedings, pp. 149-154, IEEE Computer Society, 2012.
\end{thebibliography}

\title{Heterogeneous Architectures for Computational Intensive Applications: A Cost-effectiveness Analysis.}
\tocauthor{Emanuele Danovaro, Alfonso Quarati} \author{} \institute{}
\maketitle
\begin{center}
{\large Emanuele Danovaro, Alfonso Quarati}\\
CNR-IMATI\\
{\tt {danovaro, quarati}@ge.imati.cnr.it}
\\ \vspace{4mm}
{\large Andrea Clematis, Antonella Galizia, Daniele D'Agostino}\\
CNR-IMATI\\
{\tt {clematis, galizia, dago}@ge.imati.cnr.it}
\end{center}

\section*{Abstract}
Current workstations can offer really amazing raw computational power: up to 10 TFlops on a single machine equipped with multiple CPUs and accellerators as the Intel Xeon Phi or GPU devices, which means less than half a dollar for a GFlop. 
Such result can only be achieved with a massive parallelism of the computational devices, but unfortunately not every application are able to fully exploit them. 
On the contrary, the traditional HPC architectures based on clusters can have a cost per GFlop an order of magnitude higher, but offer a well-known architecture that is easier to exploit, so real achieved performances can be closer to the theoretical maximum.\\
It is well known that the different elements of the whole software stack, from parallel algorithms design to programming paradigm and development tools are other key factors in developing application that can exploit current heterogeneous architectures. In fact they can have a huge impact in development time and cost, and in computational efficiency of the produced parallel application.\\
In this paper we analyze the performances of some widely used, computational intensive, applications, like FFT, convolution, Navier-Stokes solutor,  comparing the performance of amulti-core workstation and a  a small cluster, with or without the contribution of GPUs.
We aim to provide clear measure of the benefit of a heterogeneous architecture, in term of time and cost, with a stress on the technology adopted at different levels of the software stack for the application parallelization.


\bibliographystyle{plain}
\begin{thebibliography}{10}
\bibitem{A view of the parallel computing landscape}
{\sc K. Asanovic and R. Bodik and J. Demmel and T. Keaveny and K. Keutzer and J. Kubiatowicz and N. Morgan and D. Patterson and K. Sen and J. Wawrzynek and D. Wessel and K. Yelick}. {A view of the parallel computing landscape}. Commun. ACM 52, 10 (October 2009), 56-67.
\end{thebibliography}

\title{A Parallel Isosurface Extraction Component for Visualization Pipelines Executing on GPU Clusters}
\tocauthor{Daniele D'Agostino} \author{} \institute{}
\maketitle
\begin{center}
{\large Daniele D'Agostino}\\
CNR-IMATI\\
{\tt dagostino@ge.imati.cnr.it}
\end{center}

\section*{Abstract}
Isosurface extraction is an important operation in many e-Science applications which allows for a variety of queries on 3D data. Given a threshold value (or \emph{isovalue}), the isosurface enables inspection of, for example, tissue features and shapes of organs in medical analysis, shape of ~--~and interaction between~--~molecules in bioinformatics applications, local rainfall from weather radar measurements, and so on. Because of the importance of the operation, it is essential to have a high-performance implementation available that is scalable in the input datasets that can be handled, that produces high quality results, and that can be plugged into different visualization pipelines of any type of (complex) applications. Marching Cubes is the most widely used algorithm for isosurface extraction from volumetric datasets. It has been widely studied and many efficient implementations, both sequential and parallel, have been presented in the literature. Of these, several implementations are aimed at CUDA architectures: they provide good performance but have two important drawbacks. First, the implementations focus on direct rendering of the resulting isosurface, but do not allow proper storage for subsequent processing. Second, the implementations are designed for execution on a single Graphics Processing Unit (GPU). 

In this contribution the development of an effective parallel software component that, apart from direct rendering, allows both for efficient storage of isosurfaces, and that can exploit multiple CUDA devices concurrently, is presented. 
The developed software component has the following characteristics:
\begin{itemize}
\item[-] The programming interface (API) is identical to the original algorithm;
\item[-] The quality and size of the produced isosurface is identical to that of the original algorithm; to this end a smart vertex interpolation strategy which is not adopted in related work is exploited;
\item[-] Very large datasets can be processed, due to a combination of data buffering and optimized use of device memory;
\item[-] Speedups of close to 1490 are obtained on a moderate-size GPU cluster.
\end{itemize}

\bibliographystyle{plain}
\begin{thebibliography}{10}
\bibitem{A survey of the marching cubes algorithm}
{\sc T. S. Newman and H. Yi}. {A survey of the marching cubes algorithm}. Computers and Graphics, vol. 30, no. 5, pp. 854-879, 2006.
\end{thebibliography}

\title{Efficient Image Processing Elaboration on Parallel Resources}
\tocauthor{A. Galizia, D. D'Agostino A. Clematis} \author{} \institute{}
\maketitle
\begin{center}
{\large A. Galizia, D. D'Agostino A. Clematis}\\
CNR-IMATI\\
{\tt \{galizia,dago,clematis\}@ge.imati.cnr.it}
\end{center}

\section*{Abstract}
Scientific image processing is an important topic of interest for a broad scientific community since it is a means of gaining understanding and insight into the data for a growing number of applications. Furthermore the technological evolution permits large data acquisition through the use of sophisticated instruments or by complex multidisciplinary applications, resulting in datasets that are growing at an extremely rapid pace. However the consequence is that the processing of such images needs huge computational power, and therefore it makes necessary to move towards parallel computing, with the consequent need to develop a parallel version of the image processing algorithms/operations. 
In this work we describe PIMA(GE) Lib, the Parallel IMAGE processing GEnoa Library, developed with the purpose of providing robust and efficient implementations of the most common low level image processing operations, according to the classification provided in Image Algebra [1]. The library is based on the data parallel paradigm, and a important feature is the possibility to compose the operations in workflows optimized as regards the data movements among the parallel processes, allowing complex applications with variable granularity. The adoption of automatic optimization policies, that are transparent to the user, permit to obtain interesting values of performance and scalability.  The first version of the library has been developed using MPI library for the cluster computing environment. 
Moreover, a new study has been done to explore the use of heterogeneous parallel architectures, including Graphics Processing Unit (GPU) as General Purpose devices, in particular considering the CUDA environment. The aim is to introduce a  new level of parallelism in PIMA(GE) Lib, merging the optimization policies used on traditional architectures with the GPU implementation, in order to fully exploits the computational power of clusters equipped with these devices.  In this work the strategy adopted and early promising results, are presented. 


\bibliographystyle{plain}
\begin{thebibliography}{10}
\bibitem{Handbook of ComputerVision Algorithms in Image Algebra}
{\sc G. Ritter and J.Wilson }. {Handbook of ComputerVision Algorithms in Image Algebra}. 2nd edition. 2001. CRC Press.
\end{thebibliography}

\title{Multivariate Integration Algorithms and Applications on GPUs}
\tocauthor{Elise Helene J. de Doncker} \author{} \institute{}
\maketitle
\begin{center}
{\large Elise Helene J. de Doncker}\\
Western Michigan University\\
{\tt elise.dedoncker@wmich.edu}
\\ \vspace{4mm}
{\large Rida Assaf}\\
Western Michigan University\\
{\tt rida.assaf@wmich.edu}
\end{center}

\section*{Abstract}
We focus on algorithms and software solutions for numerical multivariate integration, by GPU computing using CUDA. These methods target mathematical modeling and simulation problems, which give rise to computational-intensive multivariate integrals.

The problems arise in such areas as: computational geometry with applications to computer graphics [5]; behavioral psychology and biometrics, for the analysis of discriminal sensory perceptions [2]; and computational finance, for the modeling of cash flows. 

Furthermore, in medical physics, we compute the dose function for radiation therapy in oncology, as discrete values on a 3D grid, where the resolution analysis is based on the results of an inverse Fourier transform performed numerically [3]. In high-energy physics we obtain corrections to the interaction cross section for the collision of elementary particles, by the computation of Feynman loop integrals [1]. 

A basic cubature or integration rule is a weighted sum of function values to approximate the integral of a function $f(\vec{x})$ over a specific domain ${\mathcal D},$ so that
$\int_{\mathcal D} f(\vec{x}) dx \approx \sum_{j=1}^N f(\vec{x}_j) w_j,$ where the sum can be considered as a scalar product $\vec{v}\cdot \vec{w}$ with $v_j =  f(\vec{x}_j), ~1\le j\le N.$ 

For high dimensions or an erratic function behavior, Monte Carlo type sampling is advised, by evaluating the function at a (pseudo-)random point sequence. Other sampling methods generate points on a regular grid or lattice, and "Quasi-Monte Carlo" methods use a randomized grid. 

The cubature rule is mapped to a GPU. Its evaluation within the CUDA kernel may be associated with an MPI (Message Passing Interface) process running on a cluster node (CPU), for example as part of a program layered over the MPI parallel environment. 
We will give results and analyze parallel performance on the new NVIDIA Tesla Kepler (K20) GPUs, which have the "dynamic parallelism" feature allowing GPU threads to automatically spawn new threads, and the "Hyper-Q" capability enabling multiple CPU cores to use CUDA cores on the same GPU device simultaneously [4].


\bibliographystyle{plain}
\begin{thebibliography}{10}
\bibitem{Quadpack Computation of Feynman Loop Integrals}
{\sc E. de~Doncker and J. Fujimoto and N. Hamaguchi and T. Ishikawa and Y. Kurihara and Y. Shimizu and F. Yuasa}. {Quadpack Computation of Feynman Loop Integrals}. J. Comp. Science (JoCS) 3, 3 (2011), 102-112.

\bibitem{Dimensional Recursion for Multivariate Adaptive Integration}
{\sc E. de Doncker and and K. Kaugars}. {Dimensional Recursion for Multivariate Adaptive Integration}. Procedia Computer Science 1 (2010), 117-124.

\bibitem{A Fast Integration Method and its Application in a Medical Physics Problem}
{\sc S. Li and E. de Doncker and K. Kaugars and and H. Li }. {A Fast Integration Method and its Application in a Medical Physics Problem}. Springer Lecture Notes in Computer Science (LNCS) 3984 (2006), 789-797.

\bibitem{Tesla K-Series GPU Accel}
{\sc url NVIDIA}. {Tesla K-Series GPU Accel}. www.nvidia.com/content/tesla/pdf/Tesla-KSeries-Overview-LR.pdf.

\bibitem{The Expected Volume of a Tetrahedron whose Vertices are Chosen at Random in the Interior of a Cube}
{\sc A. Zinani}. {The Expected Volume of a Tetrahedron whose Vertices are Chosen at Random in the Interior of a Cube}. Monatsh. Math. 139, 341-348 (2003).
\end{thebibliography}

\title{GNU Octave, a Free High-level Environment for Scientific Computing}
\tocauthor{Carlo de Falco} \author{} \institute{}
\maketitle
\begin{center}
{\large Carlo de Falco}\\
MOX - Department of Mathematics - Politecnico di Milano\\
{\tt carlo.defalco@polimi.it}
\\ \vspace{4mm}
{\large Carlo de Falco}\\
MOX - Department of Mathematics - Politecnico di Milano\\
{\tt carlo.defalco@polimi.it}
\end{center}

\section*{Abstract}
GNU Octave is a Free high-level environment for
 numerical computations. 
It provides capabilities for the numerical solution of a wide range of linear and 
nonlinear problems which may be accessed via an
 interpreted scripting language, mostly 
compatible with that of Matlab, or via a 
C++ API. 
It also provides extensive graphics capabilities for data visualization and manipulation. 
In this talk we present an overview of some of the most advanced, yet least known, of the features of GNU Octave, which may be of interest to the Scientific Computing community. 
We also discuss the history, the current status and some future perspectives in the development of the GNU Octave project.

\bibliographystyle{plain}
\begin{thebibliography}{10}
\bibitem{GNU Octave Manual Version 3}
{\sc John W. Eaton and David Bateman and Søren Hauberg}. {GNU Octave Manual Version 3}. Network Theory Ltd.
\end{thebibliography}

\title{Model of System Disk-Shaft Mutually Fixed by Induction Shrink Fit}
\tocauthor{Ivo Dolezel} \author{} \institute{}
\maketitle
\begin{center}
{\large \underline{Ivo Dolezel}}\\
Czech Technical University in Prague, Faculty of Electrical Engineering\\
{\tt dolezel@fel.cvut.cz}
\\ \vspace{4mm}
{\large Vaclav Kotlan}\\
University of West Bohemia, Faculty of Electrical Engineering\\
{\tt vkotlan@kte.zcu.cz}
\\ \vspace{4mm}
{\large Bohus  Ulrych}\\
University of West Bohemia, Faculty of Electrical Engineering\\
{\tt ulrych@kte.zcu.cz}
\end{center}

\section*{Abstract}
A complete model of system disk-shaft mutually fixed by induction shrink fit is presented. The model consists of two parts. The first part is purely mechanical and its goal is to find an appropriate interference between both above parts sufficient for transfer of a given torque. The design of the disk is based on the condition that mechanical stresses in it (in any admissible operation regime) must not exceed the yield stress of its material. The second part deals with the numerical modeling of the process of induction heating of the disk before its putting on the shaft. This process is characterized by a nonlinear and nonstationary interaction of three physical fields (magnetic field, temperature field and field of thermoelastic displacements). Its model is solved in a monolithic formulation using the finite element technique. The methodology is illustrated by a typical example. The disk represents an active wheel of a gas turbine and its shape approximately corresponds to a disk of constant internal stress when operated on the prescribed revolutions. The paper contains a lot of results that are detailedly discussed.

\bibliographystyle{plain}
\begin{thebibliography}{10}
\bibitem{On Shrink Fit Analysis and Design}
{\sc P. Pedersen}. {On Shrink Fit Analysis and Design}. Computational Mechanics 37 (2006), No. 2, 121-130.

\bibitem{Analysis on the Process of Induction Heating for Shrink-Fit Chuck}
{\sc W. Shulin and Z. Bo and Z. Weizhan and Y. Zhije and L. Gang}. {Analysis on the Process of Induction Heating for Shrink-Fit Chuck}. Advanced Materials Research 383-390 (2012), 2850-2855.

\bibitem{Numerical Modeling of the Thermal-Stressed State of Cooled Pulsed Solenoids of Electrophysical Apparatus}
{\sc M. G. Pantelyat}. {Numerical Modeling of the Thermal-Stressed State of Cooled Pulsed Solenoids of Electrophysical Apparatus}. Int. Appl. Mech. 35 (1999), 420-425.
\end{thebibliography}

\title{Numerical Modeling of Induction Surface Hardening of Gear Wheels}
\tocauthor{Jerzy BarglikIvo Dolezel} \author{} \institute{}
\maketitle
\begin{center}
{\large Jerzy Barglik}\\
Silesian University of Technology, Faculty of Material Science and Metallurgy, Katowice, Poland\\
{\tt jerzy.barglik@polsl.pl}
\\ \vspace{4mm}
{\large Albert  Smalcerz}\\
Silesian University of Technology, Faculty of Material Science and Metallurgy, Katowice, Poland\\
{\tt albert.smalcerz@polsl.pl}
\\ \vspace{4mm}
{\large Roman Przylucki}\\
Silesian University of Technology, Faculty of Material Science and Metallurgy, Katowice, Poland\\
{\tt roman.przylucki@polsl.pl}
\\ \vspace{4mm}
{\large \underline{Ivo Dolezel}}\\
Czech Technical University, Faculty of Electrical Engineering, Praha, Czech Republic\\
{\tt dolezel@fel.cvut.cz}
\end{center}

\section*{Abstract}
Numerical analysis of surface induction hardening of gear wheels is presented. The process requires reaching the required temperature profile along the surface of particular teeth before cooling. From the physical viewpoint the problem is charcterized by the interaction of electromagnetic and temperature fields strongly influencing one another. The corresponding mathematical model consists of two relevant second-order nonlinear and nonstationary partial differential equations whose coefficients are temperature-dependent functions. This 3D model is then solved numerically in the hard-coupled formulation using code FLUX 3D supplemented with own procedures and scripts. Attention is paid to the numerical aspects of the task such as the convergence of solution and its stability. Performed is also the sensitivity analysis, i.e., finding the influence of the material parameters (magnetic permeability, electric and thermal conductivities, and heat capacity) on the results of computation.


\bibliographystyle{plain}
\begin{thebibliography}{10}
\bibitem{Induction hardening of complex geometry and geared parts}
{\sc F. Biasutti and Ch. Krause and S. Lupi}. {Induction hardening of complex geometry and geared parts}. Heat Processing, 2012, No. 03, pp. 60–70.

\bibitem{Experimental investigation of process parameters influence on contour induction hardening of gears}
{\sc F. Biasutti and Ch. Krause}. {Experimental investigation of process parameters influence on contour induction hardening of gears}. Proc. HES-10 (Heating by Electromagnetic Sources), May 18–21, 2010, Padua, Italy, pp. 189-199.
\end{thebibliography}

\title{Numerical Modeling of Induction Heating of Metal in Cold Crucible}
\tocauthor{Jan  KynclIvo Dolezel} \author{} \institute{}
\maketitle
\begin{center}
{\large Jan  Kyncl}\\
Czech Technical University in Prague\\
{\tt kyncl@fel.cvut.cz}
\\ \vspace{4mm}
{\large Ladislav  Musil}\\
Czech Technical University in Prague\\
{\tt musill@fel.cvut.cz}
\\ \vspace{4mm}
{\large Lubomir Musalek}\\
Czech Technical University in Prague\\
{\tt musallub@fel.cvut.cz}
\\ \vspace{4mm}
{\large Jiri Doubek}\\
Czech Technical University in Prague\\
{\tt doubekjir@fel.cvut.cz}
\\ \vspace{4mm}
{\large \underline{Ivo Dolezel}}\\
Czech Technical University in Prague\\
{\tt dolezel@fel.cvut.cz}
\end{center}

\section*{Abstract}
Numerical analysis of heating metals in a cold crucible furnace is modeled. This way of heating is applied in case of melting highly reactive metals where its direct contact with the crucible wall could lead to its contamination. In the course of heatig still solid material the continuous mathematical model of the process consists of two nonlinear and nonstationary partial differential equations describing the time evolutions of magnetic and temperature fields in the system (however, after melting, a free level of the molten metal in the form of a meniscus must also be taken into account in the model). Its numerical solution is realized by a combination of commercial and academic codes. The fields are modeled in 2D arrangement, but selected equivalent parameters occurring in the coefficients of both equations are determined on the basis of 3D modeling of selected parts of the system. The methodology is illustrated by an example of a typical heating process. The results are discussed. 

\bibliographystyle{plain}
\begin{thebibliography}{10}
\bibitem{Numerical Studies of the Melting Process in the Induction Furnace with Cold Crucible}
{\sc A. Umbrasko and E. Baake and B. Nacke and A. Jakovics}. {Numerical Studies of the Melting Process in the Induction Furnace with Cold Crucible}. COMPEL 	27 (2008), No. 2, pp. 359--368.

\bibitem{Three-Dimensional Moving Simulation of Levitation Melting Method}
{\sc M. Enokizono and T. Todaka and K. Yokoji and Y. Wada and I. Matsumoto}. {Three-Dimensional Moving Simulation of Levitation Melting Method}. IEEE Trans. Magn. 31 (1995), No. 3, pp. 1869--1872.
\end{thebibliography}

\title{Efficient Algorithm for Incompressible Two-phase Flows Involving Large Density Ratios}
\tocauthor{Suchuan Dong} \author{} \institute{}
\maketitle
\begin{center}
{\large Suchuan Dong}\\
Purdue University\\
{\tt sdong@math.purdue.edu}
\end{center}

\section*{Abstract}
We present an efficient phase field-based  algorithm for simulations of incompressible two-phase flows described by the coupled system of  Navier-Stokes and Cahn-Hilliard equations. 
The scheme has several attractive characteristics: 
(i) It is suitable for large density ratios; 
(ii) It involves only constant (time-independent) 
coefficient matrices for all flow variables, 
which can be pre-computed during pre-processing, 
so it effectively overcomes the performance 
bottleneck induced by variable coefficient matrices associated with the variable density and
variable viscosity; 
(iii) It  completely de-couples 
the computations of the velocity, pressure, 
and the phase field function. 
Strategy for high-order spectral-element type spatial discretizations to overcome the difficulty associated with the large spatial 
order of the Cahn-Hilliard equation 
is also discussed. 
With numerical simulations we demonstrate that 
the current algorithm, together 
with the Navier-Stokes/Cahn-Hilliard phase field 
approach, is an efficient and effective method 
for studying two-phase flows 
involving large density ratios, moving contact 
lines, and interfacial topology changes. 


\bibliographystyle{plain}
\begin{thebibliography}{10}
\bibitem{A time-stepping scheme involving constant coefficient matrices for phase field simulations of two-phase incompressible flows with large density ratios}
{\sc S. Dong and J. Shen}. {A time-stepping scheme involving constant coefficient matrices for phase field simulations of two-phase incompressible flows with large density ratios}. Journal of Computational Physics, 231 (2012), 5788-5804.

\bibitem{On imposing dynamic contact-angle boundary conditions for wall-bounded liquid-gas flows}
{\sc S. Dong}. {On imposing dynamic contact-angle boundary conditions for wall-bounded liquid-gas flows}. Computer Methods in Applied Mechanics and Engineering, vol 247-248 (2012), 179-200.
\end{thebibliography}

\title{Stochastic Control Theory and the Brain Cancer}
\tocauthor{Mahmoud El-Borai} \author{} \institute{}
\maketitle
\begin{center}
{\large \underline{Mahmoud El-Borai}}\\
Professor of mathematics Faculty of Science Alexandria University\\
{\tt \tt m\_m\_elborai@yahoo.com}
\\ \vspace{4mm}
{\large Khairia El-Nadi}\\
Professor of mathematics Faculty of Science Alexandria University\\
{\tt \tt khairia\_el\_said@hotmail.com}
\end{center}

\section*{Abstract}

 In this paper a stochastic mathematical model is presented that describes the concentration of tumor cells of the brain. The treatment of the brain cancer is interpreted as a stochastic  control problem. Evolution of the disease is characterized by a stochastic parabolic partial differential equation that describes the growth of a tumor brain.
While biomedical research concentrates on the development of new drugs and experimental and clinical determinations of their treatment schedules, the analysis of mathematical models can assist in testing various treatment strategies and searching for optimal ones.
Using the considered mathematical model, we try to solve some medical problems in brain caner.
Keywords: Optimal control, tumor cells,brain cancer, stochastic parabolic partial differential equations.
AMS Subject Classifications: 92B05, 37C45.


\bibliographystyle{plain}
\begin{thebibliography}{10}
\bibitem{BSDEs with stochastic Lipschitz condition and quadratic PDEs in Hilbert spaces}
{\sc Philippe Briand and Fulvia Confortola}. {BSDEs with stochastic Lipschitz condition and quadratic PDEs in Hilbert spaces}. Stochastic Processes and their Applications 118, 818-838, (2008.
\end{thebibliography}

\title{An Inverse Fractional Abstract Cauchy Problem With Nonlocal Conditions}
\tocauthor{Mahmoud El-Borai} \author{} \institute{}
\maketitle
\begin{center}
{\large \underline{Mahmoud El-Borai}}\\
Professor of Mathematics Faculty of Science Alexandria University Egypt Egypt\\
{\tt \tt m\_m\_elborai@yahoo.com}
\\ \vspace{4mm}
{\large Khairia El-Nadi}\\
Professor of Mathematics Faculty of Science Alexandria University  Egypt\\
{\tt \tt khairia\_el\_said@hotmail.com}
\end{center}

\section*{Abstract}



This note is devoted to the study of an inverse Cauchy problem in a Hilbert space $H$ for the abstract fractional differential equation of the form:  $$\frac{d^{\alpha}u(t)}{dt^\alpha}~=~A~u(t)~+~f(t)~g(t),$$with the nonlocal initial condition:$$u(0) = u_0 + \sum_{k=1}^p~c_k u(t_k),$$and the over determination condition:$$(u(t), v) = w (t),$$where (.,.) is the inner product in $H$, $f$is a real unknown function $w$ is a given real function, $u_0$, $v$are given elements in $H$, $g$ is a given abstract function with valuesin $H$, $0 < \alpha \leq 1 $, $u$ is unknown, and $A$ is alinear closed operator defined on a dense subset of $H$.It is supposed that $A$ generates a bounded semigroup. An application isgiven to study an inverse problem in a suitable Sobolev space forgeneral fractional parabolic partial differential equations with  unknown source functions and with nonlocal initial conditions.
Keywords and phrases: Fractional abstract differential equations, nonlocal initial conditions, inverse Cauchy problem 
2000 Mathematics Subject Classifications: 45D05, 47D09, 35A05, 34G20, 77D09, 47G10

\bibliographystyle{plain}
\begin{thebibliography}{10}
\bibitem{On some fractional differential  equations in- the Hilbert-space }
{\sc Mahmoud M. El-Borai}. {On some fractional differential  equations in- the Hilbert-space, }. ournal of Discrete-and--Continuous Dynamical Systems, Series A, 2005, 233-241. .
\end{thebibliography}

\title{A New Strategy for Relationship Modelling of Complex Systems Using Self-Evolving Semantic Networks}
\tocauthor{Omid Reza Esmaeili Motlagh} \author{} \institute{}
\maketitle
\begin{center}
{\large \underline{Omid Reza Esmaeili Motlagh}}\\
University Teknikal Malaysia Melaka\\
{\tt omid@utem.edu.my}
\\ \vspace{4mm}
{\large Sai Hong Tang}\\
University Putra Malaysia\\
{\tt saihong@eng.upm.edu.my}
\\ \vspace{4mm}
{\large Seyed Mahdi  Homayouni}\\
Lenjan Branch,  Islamic Azad University of Iran\\
{\tt homayouni.mahdi@gmail.com}
\end{center}

\section*{Abstract}
Mathematical modelling is complex and bulky. Therefore, neural and fuzzy computation is replacing mathematical models. Relationship modelling is described as discovering causal relationships which exist among factors within a system. Relationship between factors A and B is mathematically presented as B=F(A), where function F describes the causation from A to B. However, it is suggested that F could be replaced with a weight W in a neural model of the system where A and B are associated with the influencing and influenced neurons, respectively. Using semantic networks such as fuzzy cognitive map (FCM), graph nodes A and B could be associated with the grey scales of their actual values rather than binary values as in neural networks. Accordingly, the graph edges or the matrix of weights in FCM could be tuned to realistically mimic causal relationships among graph nodes representing all system variables including A and B. There are many applications of semantic networks in data mining, computational geometry, pattern recognition, and forecast models. This article proposes a new philosophy about automatic construction of FCMs for relationship modelling of natural and real-life systems. The main contribution is the development of FCMs rather based on physically measurable factors than qualitative abstract models seen in the literature. The term natural activation is introduced in contrast to neural activation. In terms of FCM inference, three issues are discussed including weight training, uniqueness of weights, and interpretation of weights when it comes to modelling a particular real-life system. Interpretation of weights and linking them with physical counterparts is beyond the scope of the existing abstract conceptual schema in the literature. The result obtained from a real-life example case has been included followed by the future line of the research.

\bibliographystyle{plain}
\begin{thebibliography}{10}
\bibitem{Fuzzy Cognitive Map}
{\sc O. Motlagh}. {Fuzzy Cognitive Map}. J. Neural Computing \& Applications 21(5) (2010) 1007-1015.
\end{thebibliography}

\title{FIVER: A HIGHER-ORDER EMBEDDED BOUNDARY METHOD FOR MULTI-MATERIAL COMPRESSIBLE FLOW AND FLOW-STRUCTURE INTERACTION PROBLEMS}
\tocauthor{Charbel Farhat} \author{} \institute{}
\maketitle
\begin{center}
{\large \underline{Charbel Farhat}}\\
Department of Aeronautics and Astronautics Department of Mechanical Engineering Institute for Computational and Mathematical Engineering Stanford University, USA\\
{\tt cfarhat@stanford.edu}
\\ \vspace{4mm}
{\large Alex Main}\\
Institute for Computational and Mathematical Engineering Stanford University, USA\\
{\tt alexmain@stanford.edu}
\end{center}

\section*{Abstract}
FIVER is a robust finite volume method for the solution of high-speed compressible flows in highly nonlinear multi-material domains involving arbitrary equations of state and large density jumps. The global domain of interest can include a moving or deformable solid subdomain that furthermore may undergo topological changes due to, for example, crack propagation. The key components of FIVER, include: (1) the definition of a discrete surrogate material interface, (2) the computation first of a reliable inviscid approximation of the fluid state vector on each side of a discrete material interface via the construction and solution of a local, exact, two-phase Riemann problem, (3) the algebraic solution of this auxiliary problem when the equation of state allows it, (4) the solution of this two-phase Riemann problem using sparse grid tabulations otherwise, (5) a ghost fluid scheme for approximating next the diffusive and source terms, (6) a systematic procedure for populating the ghost or inactive fluid grid points that guarantees under specified conditions the desired order of spatial accuracy, and (7) an energy conserving algorithm for enforcing the equilibrium transmission condition at a fluid-structure interface and therefore properly communicating with a finite element structural analyzer. All of the aforementioned FIVER components accommodate both explicit and implicit time-integration schemes. After motivating and reviewing FIVER, this talk will focus on demonstrating its potential for the solution of large-scale, realistic multi-phase fluid and fluid-structure interaction problems with the massively parallel simulation of the underwater implosion of an aluminum cylinder, the flapping of a real pair of wings made of mylar for the generation of thrust and lift, and the pull-up and roll maneuvers of a complete, flexible, aircraft configuration. The first application is characterized by ultrahigh compressions, shock waves, large density jumps at the fluid material interfaces, self-contact, and the initiation and propagation of cracks in the structure. The second application is characterized by a turbulent viscous flow past highly flexible wings undergoing large displacements, rotations, and deformations. The third application features fluid-structure-control coupling. The lecture will also discuss correlations with experimental data, and CPU performance issues on multi-core systems.

\bibliographystyle{plain}
\begin{thebibliography}{10}
\bibitem{Robust and provably second-order explicit-explicit and implicit-explicit staggered time-integrators for highly nonlinear fluid-structure interaction problems}
{\sc Farhat  and C.  and Rallu  and A.  and Wang  and K. and Belytschko and T. }. {Robust and provably second-order explicit-explicit and implicit-explicit staggered time-integrators for highly nonlinear fluid-structure interaction problems}. International Journal for Numerical Methods in Engineering, 84, 73-107, 2010..

\bibitem{A higher-order generalized ghost fluid method for the poor for the three-dimensional two-phase flow computation of underwater implosions}
{\sc Farhat  and C.  and Rallu  and A. and Shankaran and S. }. {A higher-order generalized ghost fluid method for the poor for the three-dimensional two-phase flow computation of underwater implosions}. Journal of Computational Physics, 227, 7674-7700, 2008..

\bibitem{A systematic approach for constructing higher-order immersed boundary and ghost fluid methods for fluid and fluid-structure interaction problems. }
{\sc Zeng  and X. and Farhat and C. }. {A systematic approach for constructing higher-order immersed boundary and ghost fluid methods for fluid and fluid-structure interaction problems. }. Journal of Computational Physics, 231, 2892–2923, 2012..

\bibitem{Algorithms for interface treatment and load computation in embedded boundary methods for fluid and fluid-structure interaction problems.}
{\sc Wang  and K.  and Rallu  and A.  and Gerbeau  and J.-F. and Farhat and C. }. {Algorithms for interface treatment and load computation in embedded boundary methods for fluid and fluid-structure interaction problems.}. International Journal for Numerical Methods in Fluids, 67, 1175-1206, 2011..
\end{thebibliography}

\title{An Approximate Formula for the Maximal SIF for an Infinite Array of Longitudinal Coplanar Internal Surface Cracks in an Autofrettaged Cylindrical Pressure Vessel}
\tocauthor{Hadi  Fekrmandi} \author{} \institute{}
\maketitle
\begin{center}
{\large Hadi  Fekrmandi}\\
Department of Mechanical and Materials Engineering, Florida International University\\
{\tt hfekr001@fiu.edu}
\\ \vspace{4mm}
{\large Cesar  Levy}\\
Department of Mechanical and Materials Engineering, Florida International University\\
{\tt levyez@fiu.edu}
\\ \vspace{4mm}
{\large Qin Ma}\\
Edward F. Cross School of Engineering, Walla Walla University\\
{\tt Qin.Ma@wallawalla.edu}
\\ \vspace{4mm}
{\large Mordechai Perl}\\
Department of Mechanical Engineering, Ben Gurion University of the Negev\\
{\tt merpr01@bgumail.bgu.ac.il}
\end{center}

\section*{Abstract}
In this study the three dimensional mode I stress intensity factors (SIF) were numerically determined for a longitudinal array of internal surface cracks emanating from the bore of an autofrettaged cylindrical pressure vessel. To obtain the cylinder’s fatigue life span, the effective SIFs must be predicted. 
A three dimensional linear elastic finite element method is employed in the analysis and a sub-modeling technique incorporates singular elements at the crack front to evaluate the SIFs. The cylinder is assumed to be infinitely long to preclude end effects and planes of symmetry are exploited whenever possible. The pressure is assumed to operate in the bore and on the crack faces as well. The autofrettage residual stress field, wherein the Bauschinger effect is accounted for, is simulated by an equivalent thermal load. The effect of dimensionless crack parameters such as crack depth-to-wall thickness, crack ellipticity, crack spacing ratio and level of autofrettage on the maximum combined SIF, Keff, is found and the behavior of the Kmax is investigated as a function of these parameters. 
    To reduce the computational effort and develop a more efficient and generalized function to estimate the maximum stress intensity factors for intermediate crack parameters, a curve fitting technique is employed to obtain the maximum SIFs as a function of dimensionless crack parameters and the accuracy of SIFs that are calculated using this method are verified by the proposed FEM analysis. 

\bibliographystyle{plain}
\begin{thebibliography}{10}
\bibitem{Stress Intensity Factors for Partially Autofrettaged Pressurized Thick-Walled Cylinders Containing Closely and Densely Packed Cracks}
{\sc C. Levy and Q. Ma and M. Perl}. {Stress Intensity Factors for Partially Autofrettaged Pressurized Thick-Walled Cylinders Containing Closely and Densely Packed Cracks}. Journal of Pressure Vessel Technology 2010, 132, 051203–051219.
\end{thebibliography}

\title{Modeling Adversarial Intent With a POMDP}
\tocauthor{Mike  Fowler} \author{} \institute{}
\maketitle
\begin{center}
{\large Mike  Fowler}\\
Clarkson University\\
{\tt fowlermj@clarkson.edu}
\end{center}

\section*{Abstract}
With the advent of modern cyber warfare techniques, the field of cyber security has been forced to adapt to a changing climate of malicious activity. Complicated multi-stage attacks have become the new norm and have created a perpetual state of uncertainty for network administrators, who struggle to adapt their defenses to these new and ever-changing threats. In this work, we model a malicious attacker and a network admin with a Partially Observable Markov Decision Process (POMDP) in an attempt to optimize the strategies of the network admin.  Within this framework, we examine some ways to extract adversarial intent and strategy based on observations of the network.  We attempt to seamlessly incorporate this information into our POMDP model. In particular, we look to do so in an optimal way that considers reward from both current network state and reward earned from gaining information about the adversary.

\bibliographystyle{plain}
\begin{thebibliography}{10}
\bibitem{Planning and Acting in Partially Observable Stochastic Domains}
{\sc Anthony Cassandra and Leslie Pack Kaelbing and Michael Littman}. {Planning and Acting in Partially Observable Stochastic Domains}. Artificial Intelligence, 101:99--134, 1998.

\bibitem{A Game Theoretic Analysis of Intrusion Detection in Access Control Systems}
{\sc Tansu Alpcan and Basar Tamer}. {A Game Theoretic Analysis of Intrusion Detection in Access Control Systems}. Proceedings of the 43rd IEEE on Decision and Control, pages 1568-1573, 2004.

\bibitem{Measuring IDS-Estimated Attack Impacts for Rational Incident Response A Decision Theoretic Approach}
{\sc Zonghua Zhang}. {Measuring IDS-Estimated Attack Impacts for Rational Incident Response: A Decision Theoretic Approach}. Computers and Security, Volume 28 (Issue 7), October 2009.
\end{thebibliography}

\title{Estimating Radial Railway Network Improvement With a CAS}
\tocauthor{Eugenio Roanes--LozanoJos\'e Luis Gal\'an--Garc\'{\i}a} \author{} \institute{}
\maketitle
\begin{center}
{\large Eugenio Roanes--Lozano}\\
Algebra Dept., Universidad Complutense de Madrid, Spain\\
{\tt eroanes@mat.ucm.es}
\\ \vspace{4mm}
{\large Alberto Garc\'{\i}a--\'Alvarez}\\
Deputy Director for Renfe (Spanish Railways) Passengers Services, Spain\\
{\tt albertoga@renfe.es}
\\ \vspace{4mm}
{\large \underline{Jos\'e Luis Gal\'an--Garc\'{\i}a}}\\
Applied Mathematics Dept., Universidad de M\'alaga, Spain\\
{\tt jl\_galan@uma.es}
\\ \vspace{4mm}
{\large Luis Mesa}\\
Spanish Railways Foundation, Spain\\
{\tt observatoriodelferrocarril@ffe.es}
\end{center}

\section*{Abstract}
The Spanish railway network is very complex, with two different track gauges: the \textit{Iberian} gauge ($1667 mm$) and the \textit{international gauge} ($1435 mm$), the latter used in the high speed network. Only China has nowadays a longer high speed railway network. All new lines have been built with double track and top technologies
($\geq 300 km/h$ track design, \textit{ERTMS} traffic management system, $25000 KV$ AC electrification, etc.). But there are controversial opinions among experts regarding how the network should grow. A possibility could be to build very high speed trunks followed by \textit{not so high speed} antennas (the latter not reserved to high speed trains). We have two research lines focused on the comparison of different alternatives. On one hand, we have developed a computer package that is able to calculate precise timings, consumptions, costs, emissions, best routes, etc., for each piece of \textit{Renfe}'s (main railway operator) rolling stock running on \textit{Adif}'s (infrastructure company) lines [1]. On the other hand, we have developed what we have called \textit{isochrone circle graphs} and a \textit{geometric index} for radial railway networks improvement estimation [2]. These graphs were inspired by \textit{isochrone diagrams}, but also take into consideration the population served by each line. This latter article was illustrated with a sketch constructed with a Dynamic Geometry System and used sliders to change the input parameters (timing to each peripheric destination and population of these destinations). Although very comfortable to use, altering the number of destinations considered required to construct a complete new sketch. We have therefore begun from scratch and have designed and implemented a complete new package in the CAS \textit{Maple} that takes as input the lists of destinations, timings and populations and builds the corresponding \textit{isochrone circle graphs} and performs all the corresponding calculations. Moreover, symbolic computations (like obtaining local extrema) can be now computed.


\bibliographystyle{plain}
\begin{thebibliography}{10}
\bibitem{Optimal Route Finding and Rolling--Stock Selection for the Spanish Railways}
{\sc A. Hernando and E. Roanes--Lozano and A. Garc\'{\i}a--\'Alvarez and L. Mesa and I. Gonz\'alez--Franco}. {Optimal Route Finding and Rolling--Stock Selection for the Spanish Railways}. Comp. in Sci. \& Eng. (2012) 14/4, 82--89,  http://dx.doi.org/10.1109/MCSE.2012.80.

\bibitem{A geometric approach to the estimation of radial railway network improvement}
{\sc E. Roanes--Lozano and A. Garc\'{\i}a--\'Alvarez and A. Hernando}. {A geometric approach to the estimation of radial railway network improvement}. Rev. R. Acad. Cienc. Exactas F\'is. Nat., Ser. A Mat. RACSAM (2012) 106/1, 35--46, http://dx.doi.org/10.1007/s13398-011-0050-6.
\end{thebibliography}

\title{Data Assimilation for Hydrodynamical Modeling of San Quintin Bay, B.C., Mexico}
\tocauthor{Mariangel Garcia} \author{} \institute{}
\maketitle
\begin{center}
{\large Mariangel Garcia}\\
Computational Science Research Center - San Diego State University\\
{\tt mgarcia@sciences.sdsu.edu}
\end{center}

\section*{Abstract}
San Quintin Bay (SQB), B.C., forms an interesting ecosystem in which has been develop aquaculture for more than 30 years. The most important biological process for cultivating organisms is the reproductive cycle, which is principally regulated by the amount and quality of the food as well as by temperature and salinity. Furthermore, the hydrodynamic state has to be known with high accuracy for efficient ecological monitoring of the bay to satisfy this need. A regional numerical model using Delft3d is presented to provide information on the water velocities and water elevation around SQB.  In this study data assimilation aims to incorporate measured observation into dynamical system model in order to produce accurate estimates of all the current state variables of the system. To pursuit this goal, the open-source software environment OpenDA for sensitivity analysis and simultaneous parameter optimization is used. The purpose of this study, is to give short-term operational predictions of the hydrodynamical state of the SQB. A limited set of costal measurement observed data can be used to explore how filtering methods can be used to combine model output with observed data.


\bibliographystyle{plain}
\begin{thebibliography}{10}
\bibitem{Simulating the Hydrodynamics of San Quintin Bay Bcfa Mexico}
{\sc I. Ramirez and R. Blanco and R. Vazquez and G. Ramirez}. {Simulating the Hydrodynamics of San Quintin Bay, Bcfa, Mexico}. ECM11: Eleventh International Conference of Estuarine and Coastal Modeling.

\bibitem{OpenDA open source generic data assimilation environment and its application in process models. }
{\sc El Serafy GY and Verlaan M and Hummel S and Weerts A and Dhondia J}. {OpenDA open source generic data assimilation environment and its application in process models. }. Geophysical Research Abstracts 12:EGU2010-9346-2.
\end{thebibliography}

\title{Higher Order Multiscale Finite Element for Solving Problems With Heterogeneous Coefficients}
\tocauthor{Victor Ginting} \author{} \institute{}
\maketitle
\begin{center}
{\large Victor Ginting}\\
University of Wyoming\\
{\tt vginting@uwyo.edu}
\\ \vspace{4mm}
{\large Lawrence  Bush}\\
University of Wyoming\\
{\tt lbush4@uwyo.edu}
\end{center}

\section*{Abstract}
In this presentation, we discuss a numerical multiscale approach for solving elliptic boundary value problems with heterogeneous coefficients. Our interest comes from flow in porous media applications and we assume that there is no scale separation in the spatial variables. To obtain the solution of these multiscale problems on a coarse grid, we first construct a set of multiscale basis functions that take into account the heterogeneity information. With this representation, we set a Galerkin finite element approximation that satisfies a variational formulation. It has been long established that the Galerkin FEM type solution does not satisfy local conservation property. We propose a postprocessing technique that gives a conservative flux from the multiscale solution. Finally, we give several examples when the multiscale method is used to solve multiphase flow and transport problems. 

\bibliographystyle{plain}
\begin{thebibliography}{10}
\bibitem{Operator Splitting Multiscale Finite Volume Element Method for Two-Phase Flow with Capillary Pressure}
{\sc F. Furtado and V. Ginting and F. Pereira and M. Presho}. {Operator Splitting Multiscale Finite Volume Element Method for Two-Phase Flow with Capillary Pressure}. Transport in Porous Media, 90 (2011), no. 3, pp. 927-947.
\end{thebibliography}

\title{Discrete Network Approximation for Dirichlet-to-Neumann Map for High Contrast Problems}
\tocauthor{Yuliya Gorb} \author{} \institute{}
\maketitle
\begin{center}
{\large Yuliya Gorb}\\
University of Houston\\
{\tt yagorb@gmail.com}
\end{center}

\section*{Abstract}
The motivation for the problem described in this talk is driven by the applications in which processes occur in strongly heterogeneous media with large variations in material properties, e.g. subsurface flows in natural porous formations, electrical conduction in composite materials, and medical and geophysical imaging (impedance tomography). 
Mentioned feature is referred to as a high contrast in material properties and characterized by parameter $\epsilon=\max_{x\in\Omega}\sigma(x)/\min_{x\in\Omega}\sigma(x)\gg 1,$ where $\sigma$ is conductivity of the underlying highly heterogeneous medium. 
Modeling and numerical simulation of such media pose significant mathematical and computational challenges. Indeed, the fields inside strongly heterogeneous high contrast media exhibit singular behavior that is laborious to capture analytically, or the convergence rates of iterative methods, when solving the problem numerically, deteriorate as the variations in problem parameters increase. 


To that end, we consider a high contrast heterogeneous medium with particles that are close-to-touching, and study the Dirichlet-to-Neumann map describing the corresponding domain decomposition preconditioner. Note that developing domain decomposition preconditioners for multiscale flows in high-contrast media with a condition number that is independent of the contrast in medium constituents has been actively studied in the last couple of years. In this talk, we propose a discrete network approximation for the Dirichlet-to-Neumann map whose construction is demonstrated and justified. Application of the constructed approximation for solving flows in high-contrast heterogeneous media by a conjugate gradient type iterative method is discussed.


\bibliographystyle{plain}
\begin{thebibliography}{10}
\bibitem{Discrete network approximation for highly-packed composites with irregular geometry in three dimensions}
{\sc L. Berlyand and Y. Gorb and A. Novikov}. {Discrete network approximation for highly-packed composites with irregular geometry in three dimensions}. Multiscale methods in science and engineering, 2157, Lect. Notes Comput. Sci. Eng., 44, Springer, Berlin, 2005.

\bibitem{Domain Decomposition Preconditioners for Multiscale Flows in High-Contrast Media}
{\sc Y. Efendiev and J. Galvis}. {Domain Decomposition Preconditioners for Multiscale Flows in High-Contrast Media}. SIAM MMS, 8:4, (2010), pp. 1461--1483.

\bibitem{Domain Decomposition Preconditioners for Multiscale Flows in High-Contrast Media Reduced Dimension Coarse Spaces}
{\sc Y. Efendiev and J. Galvis}. {Domain Decomposition Preconditioners for Multiscale Flows in High-Contrast Media: Reduced Dimension Coarse Spaces}. SIAM MMS, 8:5, (2010), pp.1621--1644.
\end{thebibliography}

\title{BLAZE-DEM: A GPU Based Polyhedral DEM Particle Transport Code}
\tocauthor{nicolin govender} \author{} \institute{}
\maketitle
\begin{center}
{\large nicolin govender}\\
CSIR, University of Pretoria\\
{\tt govender.nicolin@gmail.com}
\end{center}

\section*{Abstract}
This paper introduces the BLAZE-DEM code that is based on the Discrete Element Method (DEM) [1] and specifically targeted for Graphical Processing Unit (GPU) platforms.

BLAZE-DEM uses actual polyhedral particle representations as opposed to multi-sphere[2] approaches , which approximate polyhedral geometries. 

The modeling of real particle shapes is critical for realistically simulating complex interaction phenomena in granular assemblies [3].  

BLAZE-DEM primarily concerns itself with simulating the flow of granular materials for a variety of different geometries [4]. 

The use of computational modeling tools is essential in  evaluation of various designs and processes as computational power increases [5]. However current DEM simulations are only able model a few hundred to thousands of particles in real time without the use of expensive and power consuming clusters [6]. 

The dramatic increase in GPU computing power has enabled the computational simulation of physical systems that was not possible a few years ago via CUDA API[8].

The DEM method involves computing the interactions of all particles that are in contact to determine the net force and subsequently its evolution in phase space, which is an N Body problem. Solving this problem for all N Bodies is too computational expensive and cannot be done in real time. This paper will discuss methods and algorithms that substantially reduce the computational run-time of such simulations. An example is the spatial partitioning and hashing algorithm that allows just the nearest neighbors (NN) which are most likely to be in contact to be determined thus reducing the required computations. 

The computational strategy employed in BLAZE-DEM which makes use of multiple kernels on the GPU via CUDA will be discussed as well as the software design that allows for a variety of particle shapes and geometries. 

The BLAZE-DEM code which is still in development and achieves 166 FPS for 65536 8 Faced polyhedra compared to the 300 FPS for spheres obtained by [7]. The planned features in physics and computation will also be discussed.

\bibliographystyle{plain}
\begin{thebibliography}{10}
\bibitem{A discrete numerical model for granular assemblies.}
{\sc P. Cundall and A. Strack.}. {A discrete numerical model for granular assemblies.}. 1979, Geotechnique 29, 47.

\bibitem{Investigation of adequacy of multi-sphere approximation of elliptical particles for DEM simulations.}
{\sc D. Markauskas. }. {Investigation of adequacy of multi-sphere approximation of elliptical particles for DEM simulations.}. Granular Matter (2010) 12 , 107-123.

\bibitem{iscrete element modelling of non-spherical granular flow in rectangular hopper.}
{\sc H. Tao and B. Jin and et.al. }. {iscrete element modelling of non-spherical granular flow in rectangular hopper.}. 2010, Chemical Engineering and Processing 49 151–158.

\bibitem{Large-scale powder mixer simulations using massively parallel GPU architectures.}
{\sc A. Radeke. }. {Large-scale powder mixer simulations using massively parallel GPU architectures.}.  2010,Chemical Engineering Science 65 6435-6442.

\bibitem{The beginning of a new era in design calibrated discrete element modelling. }
{\sc A. Grima.}. {The beginning of a new era in design: calibrated discrete element modelling. }. 2012, University of Wollongong DEM Solutions Ltd, Edinburgh, UK.

\bibitem{Experimental validation of polyhedral discrete element model.}
{\sc S. Mack and P. Langston and et.al. }. {Experimental validation of polyhedral discrete element model.}. 2011, Powder Technology 214 431-442.

\bibitem{www.nvidia.com}
{\sc 2012 Nvidia.}. {www.nvidia.com}. (December 2012 Cuda 5.0).

\bibitem{Particle simulation using CUDA.}
{\sc S. Green.}. {Particle simulation using CUDA.}. 2012, presentation packaged with CUDA Toolkit 5.0.
\end{thebibliography}

\title{Numerical Simulation of the Turbulent Flow Around a Turbo-sail}
\tocauthor{Ouahiba Guerri} \author{} \institute{}
\maketitle
\begin{center}
{\large \underline{Ouahiba Guerri}}\\
LaSIE. Université de La Rohelle\\
{\tt oguerri@univ-lr.fr}
\\ \vspace{4mm}
{\large Erwan Liberge}\\
LaSIE. Université de La Rohelle\\
{\tt erwan.liberge@univ-lr.fr}
\\ \vspace{4mm}
{\large Aziz Hamdouni}\\
LaSIE. Université de La Rohelle\\
{\tt aziz.hamdouni@univ-lr.fr}
\end{center}

\section*{Abstract}
The computation of a turbulent flow around a turbo sail is presented in this paper.  The incompressible Navier-Stokes equations are solved for a Reynolds number based on the chord of the profile of $10^5$.
Several configurations are considered. First, the profile is at a zero incidence. These simulations are performed for a profile without grid, ignoring the fluid area inside the turbo sail. Different turbulence models are applied: 
the $k-\varepsilon/V2-f$ model formulated for low Reynolds numbers, developed by Laurence et al. (1) and Billard and Laurence (2) (these two formulations are compared), 
the SSG Rij turbulence model with wall law,
the dynamic LES model with inlet conditions defined by the Synthetic Eddy Method (3).
Then the profile is set at an angle of attack of $15^\circ$ and computations are performed with the $k-\varepsilon/V2-f$ model of Billard and Laurence (2). Finally, the case of a turbo-sail with grid and suction is studied. For this configuration, computations are performed with the Rij SSG turbulence model.
All computations are performed with code\_saturne, an open source code developed by EDF. The results presented in this paper are the lift (CL) and drag (Cd) coefficients and the contours of the velocity around the profile. 

\bibliographystyle{plain}
\begin{thebibliography}{10}
\bibitem{1. A Robust Formulation of the v2-f Model}
{\sc 1. D.R. Laurence and J.C. Uribe and S.V. Utyuzhnikov}. {1. A Robust Formulation of the v2-f Model}. 1. Flow, Turbulence and Combustion 73 (2004) 169–185.

\bibitem{2. A robust k-e / v2-k elliptic blending turbulence model applied to near-wall}
{\sc 2. F. Billard and D. Laurence}. {2. A robust k-e / v2-k elliptic blending turbulence model applied to near-wall}. 2. International Journal of Heat and Fluid Flow, 33 (2012) 45–58.

\bibitem{3. Reconstruction of turbulent fluctuations for hybrid RANS/LES simulations using a Synthetic-Eddy Method}
{\sc 3. N. Jarrin and R. Prosser and J.-C. Uribe and S. Benhamadouche and D. Laurence}. {3. Reconstruction of turbulent fluctuations for hybrid RANS/LES simulations using a Synthetic-Eddy Method}. 3.  International Journal of Heat and Fluid Flow 30 (2009) 435–442.
\end{thebibliography}

\title{Two-grid Method for a 3D Two-phase Mixed-domain Non-isothermal Model of PEM Fuel Cell}
\tocauthor{Mingyan He} \author{} \institute{}
\maketitle
\begin{center}
{\large \underline{Mingyan He}}\\
Department of Mathematics,Tongji University\\
{\tt hemingyan1985@yahoo.com.cn}
\\ \vspace{4mm}
{\large Cheng Wang}\\
Department of Mathematics,Tongji University\\
{\tt wangcheng@tongji.edu.cn}
\\ \vspace{4mm}
{\large Pengtao Sun}\\
Department of Mathematical Sciences,University of Nevada Las Vegas\\
{\tt pengtao.sun@unlv.edu}
\\ \vspace{4mm}
{\large Ziping Huang}\\
Department of Mathematics, Tongji University\\
{\tt huangziping@tongji.edu.cn}
\\ \vspace{4mm}
{\large Yuzhou Sun}\\
Department of Mathematical Sciences,University of Nevada Las Vegas\\
{\tt suny615@gmail.com}
\end{center}

\section*{Abstract}
In this paper, a three-dimensional (3D), multiphysics, two-phase transport mixed-domain non-isothermal model and its effective two-grid numerical methods are systematically studied for a full proton exchange membrane fuel cell (PEMFC). The present multifluid-type two-phase model fully incorporates both anode and cathode sides, properly accounts for the following three water phases: water vapor, liquid water and water in membrane phase, and enables numerical investigations for water management issues with the existence of condensation/evaporation. An efficient and fast numerical method is crucial to be studied and implemented for the two-phase model. By employing a combined finite element-upwind finite volume method, appropriate linearization schemes and two-grid method, we eventually design a series of efficient numerical methods for the newly reformulated two-phase mixed-domain fuel cell non-isothermal model. To ensure the correctness of the attained numerical solutions, we carry out a series of convergence tests and mass balance tests to verify the numerical efficiency and accuracy of the present numerical methods. Numerical experiments demonstrate that our methods are effective and accurate to deal with the simulation with saving lots of compute time.

\bibliographystyle{plain}
\begin{thebibliography}{10}
\bibitem{Efficient Numerical Methods for an Anisotropic Nonisothermal Two-phase Transport Model of Proton Exchange Membrane Fuel Cell}
{\sc P. Sun}. {Efficient Numerical Methods for an Anisotropic, Nonisothermal, Two-phase Transport Model of Proton Exchange Membrane Fuel Cell}. Acta Applicandae Mathematicae 118(2012) 251-279.

\bibitem{Modeling and Numerical Studies for a 3D Two-Phase Mixed-Domain Model of PEM Fuel Cell}
{\sc Mingyan He and Ziping Huang and Pengtao Sun and Cheng Wang}. {Modeling and Numerical Studies for a 3D Two-Phase Mixed-Domain Model of PEM Fuel Cell}. J. Electrochem. Soc.160(2013) F324-F336.
\end{thebibliography}

\title{Weighted Least-Square Finite Element Methods for PIV Data Assimilation}
\tocauthor{Jeffrey Heys} \author{} \institute{}
\maketitle
\begin{center}
{\large Jeffrey Heys}\\
Montana State University\\
{\tt jeffrey.heys@coe.montana.edu}
\end{center}

\section*{Abstract}
The assimilation of noisy experimental data into a computer simulation that is based on the solution of differential equations is an ongoing challenge in many scientific fields. The need for data assimilation exists in many fields including meteorological modeling and particle imaging velocimetry (PIV). Existing approaches for data assimilation rely heavily on smoothing and interpolation of the experimental data, which can have a crippling impact on the data. Moreover, they tend to use a large ensemble of approximate solutions that, in many large-scale simulations, is simply not practical.

Recently, echocardiologists have developed methods for introducing microbubbles into circulating blood that can be resolved using ultrasound. The location of the microbubbles, combined with the high temporal resolution of ultrasound, allows the local blood velocity to be determined using PIV, but the high temporal resolution requirement also limits the ultrasound scans (and, hence, the velocity field data) to two dimensions. One goal of using the FDA-approved microbubbles is to use the blood velocity data to calculate physiologically important information such as pressure gradients and energy loss for the blood flow, but these calculations require a full three-dimensional velocity field. What is needed for this problem and many others is a numerical approach that can approximate the solution of a system of differential equations and assimilate an arbitrary quantity of noisy experimental data at arbitrary points within the domain to give a “most likely” approximate solution that is properly influenced by the experimental data.

We have developed a novel data assimilation strategy for combining, in a very flexible and consistent manner, a numerical approximation to the Navier-Stokes equations with noisy, two-dimensional PIV data. The data assimilation and approximate solution of the Navier-Stokes equations are based on least-squares finite element methods (LSFEMs), and this approach is uniquely well suited for assimilating noisy PIV data into approximation processes in a very general way. For cases where the PIV data are known more accurately, the data can be weighted so that the approximate solution closely matches the data and, in cases of noisy data, the data can use a lower weight so that need not be matched as closely. The LSFEM approach for assimilating PIV data is demonstrated on a number of test problems, including a moving flap and pulsatile flow into the left ventricle.

\bibliographystyle{plain}
\begin{thebibliography}{10}
\bibitem{New results in linear prediction and filter theory}
{\sc R. Kalman and R. Bucy}. {New results in linear prediction and filter theory}. Trans. ASME J. Basic Engrg., 1961, 83D: 85-108.

\bibitem{Weighted Least-Squares Finite Elements for Particle Imaging Velocimetry Analysis}
{\sc J.J. Heys and T.A. Manteuffel and S.F. McCormick and M. Milano and M. Belohlavek}. {Weighted Least-Squares Finite Elements for Particle Imaging Velocimetry Analysis}.  J. Comp. Physics, 2010, 229(1): 107-118.

\bibitem{Weighted Least-Square Finite Element Method for Cardiac Blood Flow Simulation with Echocardiographic Data}
{\sc F. Wei and J. Westerdale and M. Belohlavek and and J.J. Heys}. {“Weighted Least-Square Finite Element Method for Cardiac Blood Flow Simulation with Echocardiographic Data}. Comp. Math. Methods Medicine, 2012, n. 371315.
\end{thebibliography}

\title{A Genetic Algorithm for Optimization of Scheudling of Container Handling Equipment}
\tocauthor{Seyed Mahdi Homayouni} \author{} \institute{}
\maketitle
\begin{center}
{\large \underline{Seyed Mahdi Homayouni}}\\
Department of Industrial Engineering, Lenjan Branch, Islamic Azad University, Zarrinshahr, Isfahan, Iran.\\
{\tt homayouni@iauln.ac.ir}
\\ \vspace{4mm}
{\large Sai Hong Tang}\\
Mechanical and Manufacturing Engineering Department, Universiti Putra Malaysia (UPM), Selangor, Malaysia\\
{\tt saihong@eng.upm.edu.my}
\\ \vspace{4mm}
{\large O.  Motlagh}\\
Faculty of Manufacturing Engineering, Universiti Teknikal Malaysia Melaka (UTeM), 76100 Melaka, Malaysia\\
{\tt motlagh7@gmail.com}
\end{center}

\section*{Abstract}
In automated container terminals (ACTs), containers are unloaded from (loaded to) the vessels, by using quay cranes (QCs). Commonly, containers are stacked in storage yards before they delivered to final customers. Moreover, automated guided vehicles (AGVs) are used to connect the QCs to the storage yard. Previous researches proposed split-platform automated storage/retrieval system (SP-AS/RS) for the ACTs. In this paper, a genetic algorithm (GA) is developed to optimize the integrated scheduling of aforementioned container handling equipment with the objective function of minimizing total delays in tasks of QCs, in addition to total travel time of AGVs and platforms of the SP-AS/RS. The GA determines sequence of loading/unloading tasks; and a heuristic rule assigns the AGVs to the tasks. It is shown that the proposed GA finds near optimal solutions in a relatively low computational time. Moreover, it is shown that the proposed GA outperforms the previous algorithms.

\bibliographystyle{plain}
\begin{thebibliography}{10}
\bibitem{Integrated scheduling of SP-AS/RS and handling equipment in automated container terminals}
{\sc S. M. Homayouni and M. R. Vasili and S. M. Kazemi and and S. H. Tang.}. {Integrated scheduling of SP-AS/RS and handling equipment in automated container terminals}. in 42nd International Conference on Computers and Industrial Engineering (CIE42), 2012. Cape Town, South Africa..

\bibitem{Using simulated annealing algorithm for optimization of quay cranes and automated guided vehicles scheduling}
{\sc S. M. Homayouni and S. H. Tang and N. Ismail and and M. K. A. Ariffin}. {Using simulated annealing algorithm for optimization of quay cranes and automated guided vehicles scheduling}. Int. J. Phys. Sci. 6(2011) 6286-6294..

\bibitem{Integrated scheduling of handling equipment at automated container terminals}
{\sc H. Y. K. Lau and Y. Zhao}. {Integrated scheduling of handling equipment at automated container terminals}. Int. J. Prod. Econ. 112(2008) 665-682..

\bibitem{A statistical model for expected cycle time of SP-AS/RS an application of Monte Carlo simulation}
{\sc M. R. Vasili and S. H. Tang and S. M. Homayouni and and N. Ismail}. {A statistical model for expected cycle time of SP-AS/RS: an application of Monte Carlo simulation}. Appl. Artif. Intell, 22(2008) 824 - 840..
\end{thebibliography}

\title{Designing Better Buildings With Computational Fluid Dynamics Analysis}
\tocauthor{Neihad Hussen Al-Khalidy} \author{} \institute{}
\maketitle
\begin{center}
{\large Neihad Hussen Al-Khalidy}\\
SLR Consulting, 2 Lincoln Street, Lane Cove, NSW 2066, Australia\\
{\tt nal-khalidy@slrconsulting.com}
\end{center}

\section*{Abstract}
This paper introduces the role of Computational Fluid Dynamics (CFD) as a sophisticated thermofluid modelling method to evaluate the indoor and outdoor environment of buildings. The paper provides insight into using CFD to optimise the design of sustainable buildings, diagnose air flow problems, promote innovative natural or mixed mode ventilation design and solve other complex wind driven rain and condensation problems. The topic of the first case study is the dispersion of pollutant within and then downstream of building complexes utilising CFD simulation [1].  The pollutant sources are located at the roof of a proposed building. The design of the proposed building suggests locating stacks and vertical ducts through the roof of a number of selected areas.  The fluid flow was modelled by partial differential equations describing the conservation of mass, momentum and species concentration in three Cartesian coordinate directions for steady state conditions. The flow characteristics are seen to be captured well by the CFD model. The pollutant concentrations were predicted at the chest level and at a range of elevations during near calm wind and windy conditions.  Recommendations were provided to ensure compliance with Occupational Health \& Safety (OH\&S) commission limits at human height levels and living areas. The topic of the second case study is the CFD assessment of combined internal-external airflow in building projects. In the past, partially due to the limitations of commercial CFD software and processing power, it has been the norm to use CFD analysis on the interior airflow only. However, with advances in processing power, commercial CFD can be used to produce a combined internal-external flow analysis and handle large numerical models [2]. This case study provides a procedure to produce internal-external flow CFD analysis on the examples of a number of successful building projects in Australia. The software package utilised in the CFD analysis is the commercially available code Fluent. Accuracy of the CFD model is compared against wind tunnel data. Some of the challenges facing CFD for modelling built environment are discussed in the paper. Topics include mesh development, treatment of turbulence, selection of numerical schemes and boundary conditions approximation within the constraints of the design environment. The paper demonstrates that CFD can be used as a useful design tool to manage risks, optimise opportunities, promote innovative design and enhance the cost effectiveness of building project solutions while satisfying all buildings code regulations.


\bibliographystyle{plain}
\begin{thebibliography}{10}
\bibitem{Computational Fluid Dynamics Simulations of Plume Dispersions in City Canyons }
{\sc N. Al-Khalidy}. {Computational Fluid Dynamics Simulations of Plume Dispersions in City Canyons }. Proceeding of the Fifth International Conference on Engineering Computational Technology, Las Palmas de Gran Canaria, Spain, 2006.

\bibitem{Shopping Sensation Page 49 (http//www.ancr.com.au/Pplaza.pdf)}
{\sc N. Al-Khalidy}. {Shopping Sensation, Page 49 (http://www.ancr.com.au/Pplaza.pdf)}. Australian National Construction Review, 2005.
\end{thebibliography}

\title{HMM-Based Adaptive Power Distribution by the BS in WiMAX Network}
\tocauthor{RAKESH KUMAR JHA} \author{} \institute{}
\maketitle
\begin{center}
{\large RAKESH KUMAR JHA}\\
SVNIT SURAT\\
{\tt jharakesh.45@gmail.com}
\end{center}

\section*{Abstract}
In this research proposal, I propose an adaptive mechanism of BS on the basis of the HMM
model to assist the Power Saving (PS) in WiMAX network. Recent studies have shown that the power
consumption of ICT is approximately 4 % of the annual energy production. More importantly, this
number is expected to grow drastically in the coming years. WiMAX has great chance to deploy in all
over world with its great features. In this chapter, we propose an adaptive mechanism of BS on the basis
of the HMM model to assist the Power Saving (PS) in WiMAX network. Currently the transmitted data
volume in communication networks doubles every five years. Moreover, the WWRF (Wireless World
Research Forum) has a vision of 7 trillion wireless devices serving 7 billion users by 2015.In case of
power optimization we have analyzed with three case study i.e on the basis of traffic loads then on the
basis of proportional algorithm and finally with HMM based adaptive power distribution. We have found
that HMM-Based Adaptive Power Distribution by the BS in WiMAX Network will play a significant role
in WiMAX network power optimization in next generation network.

\bibliographystyle{plain}
\begin{thebibliography}{10}
\bibitem{Oana Iosif Elena-Roxana Cirstea}
{\sc Rakesh Kumar Jha}. {Oana Iosif, Elena-Roxana Cirstea,}. Performance Analysis of uplink resource allocation in WIMAX”,Communications, 8th International Conference.

\bibitem{Reducing Power Consumption of Subscriber Stations in WiMAX Networks}
{\sc Umut Akyol and undefined undefined}. {Reducing Power Consumption of Subscriber Stations in WiMAX Networks}. 2010 Second International Conference on Evolving Internet, 2010, pp- 205.

\bibitem{WiMAX Network Optimization -Analyzing Effects of Adaptive Modulation and Coding Schemes Used in Conjunction with ARQ and HARQ}
{\sc Abdul Qadir Ansari}. {WiMAX Network Optimization -Analyzing Effects of Adaptive Modulation and Coding Schemes Used in Conjunction with ARQ and HARQ”}. Seventh Annual Communications Networks and Services Research Conference, 2009, pp- 6-13.

\bibitem{hammad Fundamentals of WiMAX Understanding Broadband Wireless Networking}
{\sc Jeffrey G. Andrews}. {hammad, “Fundamentals of WiMAX Understanding Broadband Wireless Networking}. TMH, 2007.

\bibitem{Location Based WiMAX Network Optimization Power Consumption with Traffic Load}
{\sc Rakesh Kumar Jha}. {Location Based WiMAX Network Optimization: Power Consumption with Traffic Load}. ACWR,2012.

\bibitem{Hidden Markov Model A Tutorial for the course Computational Intelligence}
{\sc Barbara Resch }. {Hidden Markov Model, “A Tutorial for the course Computational Intelligence}. Signal Processing and Speech Communication Laboratory, Inffeldasse 16c/II.
\end{thebibliography}

\title{A Mathematical Model for Bandwidth Attack Based on Game Theory for WiMAX Network}
\tocauthor{RAKESH KUMAR JHA} \author{} \institute{}
\maketitle
\begin{center}
{\large RAKESH KUMAR JHA}\\
SVNIT SURAT\\
{\tt jharakesh.45@gmail.com}
\end{center}

\section*{Abstract}
We are in the phase to deploy 4G i.e. WiMAX/LTE network in upcoming year of NGN (Next generation Networks). As we know that a significant advancement work has been done in wireless network since last few years but as far as security concerns, challenges also increases with technology development.In this paper we have introduced game-theory formation for Bandwidth attack in wireless communication networks. In our case study we have considered WiMAX networks.In our proposed algorithm found that there is possibility of bandwidth attack on the basis of Game Theory. This attack is one type of Denial of Service (DoS) attacks. The attacker has knowledge about traffic pattern of network i.e. the DL/UL mapping of client with BS. Since BS has allocated particular IP to the client. The entire process of communication between BS and client is in three phase. In first phase BS performs the operation of ranging. Once the ranging has been done, the client is able to send request to server from BS (UL) and server responds the particular application from BS (DL) to clients. In this process bandwidth is required and this bandwidth has been assigned by BS to all clients. We examine the Bandwidth attack by attacker (Un-Authorized) client on defender (valid client) with game theory. Additionally, we have analyzed how attacker client wins the game (spoof the band-width) and how defender will protect the band-width by saddle point or Nash equilibrium.

\bibliographystyle{plain}
\begin{thebibliography}{10}
\bibitem{A Game-Theoretic Framework for Bandwidth Attacks and Statistical Defenses}
{\sc Snyder undefined and M.E undefined and Sundaram undefined and R. }. {A Game-Theoretic Framework for Bandwidth Attacks and Statistical Defenses,}. Thakur, M., A Game-Theoretic Framework for Bandwidth Attacks and Statistical Defenses, “Local Computer Networks, 2007. LCN 2007, 32nd IEEE Conference , 2007, pp- 556-566 .

\bibitem{Game Theory in Wireless Networks A Tutorial}
{\sc Mark Felegyhazi and Jean-Pierre Hubaux}. {Game Theory in Wireless Networks: A Tutorial}. http://www.liafa.jussieu.fr/~zielonka/Enseignement/MPRI/2007/FelegyhaziHubaux.pdf.

\bibitem{Lecture 3 Game Theory}
{\sc Amit Aggrawal}. {Lecture 3, “Game Theory”}. IITD, August 2002 .
\end{thebibliography}

\title{Error Analysis and Blockwise Adaptivity in Time of a Coupled PDE-ODE System}
\tocauthor{August  Johansson} \author{} \institute{}
\maketitle
\begin{center}
{\large August  Johansson}\\
Department of Mathematics, UC Berkeley\\
{\tt august@math.berkeley.edu}
\end{center}

\section*{Abstract}
In [1], the authors describe and test an adaptive algorithm for evolution problems that employs a sequence of ``blocks'' consisting of fixed, though non-uniform, space meshes. This approach offers the advantages of adaptive mesh refinement but with reduced overhead costs associated with load balancing, re-meshing, matrix reassembly, and the solution of adjoint problems used to estimate discretization error and the effects of mesh changes.\\

A major issue with a block-adaptive approach is determining block discretizations from coarse scale solution information that achieve the desired accuracy. We describe several strategies to achieve this goal using adjoint-based {\em a posteriori} error estimates and we demonstrate the behavior of the proposed algorithms as well as several technical issues in a set of examples.\\

As a particular example, we will in this talk consider a novel method for solving a reaction-diffusion partial differential equation (PDE) coupled to a set of nonlinear ordinary differential equations (ODEs). The model describes the electrical activity of the heart and is computationally challenging due to the multiple scales in time and space.\\

For this system, the {\em a posteriori} error estimates distinguish the errors in time and space for the PDE and the ODEs separately and include errors due to the transfer of the solutions between the equations. In addition, since the ODEs in many applications are defined on a much smaller spatial scale than what can be resolved by the finite element discretization for the PDE, we propose different models and take the error of these into account. The error indicators derived are used in the block adaptive procedure. The results is to be presented in the forthcoming paper.

\bibliographystyle{plain}
\begin{thebibliography}{10}
\bibitem{Blockwise Adaptivity for Time Dependent Problems Based on Coarse Scale Adjoint Solutions}
{\sc V. Carey and D. Estep and V. Ginting and A. Johansson and M. Larson and S. Tavener}. {Blockwise Adaptivity for Time Dependent Problems Based on Coarse Scale Adjoint Solutions}. SIAM J. Sci. Comput. 32 (2010) 2121-2145.
\end{thebibliography}

\title{Towards Crowdsourcing for Enterprise Innovation on Social Media Technology: The Social Technical Perspective}
\tocauthor{Suwan Juntiwasarakij} \author{} \institute{}
\maketitle
\begin{center}
{\large Suwan Juntiwasarakij}\\
Computer Science, Faculty of Science, King Mongkut's Institute of Technology Ladkrabang, Bangkok, Thailand\\
{\tt kjsuwan@kmitl.ac.th}
\end{center}

\section*{Abstract}
Innovation is an important tool for a firm to compete in the ever-growing free global market. Up to date social media has been widely adopted in everyday life due to its communicability, ease of use, popularity, the network effect, and accessibility. In this regard, social media is potentially a knowledge management tool applicable to companies employing social media for organizational innovation (Bughin et al., 2008; Hinchcliffe, 2007a; 2007b; Hoover, 2007; Levy, 2009; McAfee, 2006; Scarff, 2006; Spanbauer, 2006). However, this research area has been underdeveloped and moreover not existed under social technical perspective in enterprise context.

This research aimed at studying the use of social media for internally organizational innovation in order to forge an effective framework. Drawn from two large, multinational companies’ social media platform systems through theoretical lens such as knowledge management theories, social capital theory, complex adaptive systems, and social technical systems, the findings showed that the implementation of successful, sustainable enterprise social media platforms recognizes emerging socio-psychological layers comprised of social capital formation, organic arrangement, vernacular collaboration protocol, and elastic governance. 

This research offered an enterprise social media framework in support of innovation and practical recommendations for users, designers, managers, executives, enterprises, and prospective adopters. The implication and future application of the framework will lead an adopting firm to sustainable competitiveness and innovation in global market and IT industry. This research is partly funded by the National Science and Technology Development Agency (NSTDA) of Thailand.


\bibliographystyle{plain}
\begin{thebibliography}{10}
\bibitem{Enterprise 2.0 The dawn of emergent collaboration}
{\sc A McAfee}. {Enterprise 2.0: The dawn of emergent collaboration}. MIT Sloan management review, 47(3), 21.

\bibitem{The next step in open innovation}
{\sc J Bughin and M Chui and B Johnson}. {The next step in open innovation}. The McKinsey Quarterly, June, 22-29.

\bibitem{Leveraging the convergence of IT and the next generation of the Web}
{\sc D Hinchcliffe}. {Leveraging the convergence of IT and the next generation of the Web}. Retrieved August 22, 2010: http://blogs.zdnet.com/Hinchcliffe/?p=101.

\bibitem{More results on use of WEB 2.0 in business emerge}
{\sc D Hinchcliffe}. {More results on use of WEB 2.0 in business emerge}. Retrieved August 22, 2010: http://blogs.zdnet.com/Hinchcliffe/?p=103.

\bibitem{Most Business Tech Pros Wary About WEB 2.0 Tools in Business}
{\sc N Hoover}. {Most Business Tech Pros Wary About WEB 2.0 Tools in Business}. Retrieved August 22, 2010, from Information Weeks: http://km-consulting.blogspot.com/2007/03/km-as-disciplinehas-been-disrupted-by.html.

\bibitem{WEB 2.0 implications on knowledge management}
{\sc M Levy}. {WEB 2.0 implications on knowledge management}. Journal of Knowledge Management, 13(1), 120-134.

\bibitem{Enterprise 2.0 The dawn of emergent collaboration}
{\sc A McAfee}. {Enterprise 2.0: The dawn of emergent collaboration}. MIT Sloan management review, 47(3), 21.

\bibitem{Advanced knowledge sharing with Intranet 2.0}
{\sc A Scarff}. {Advanced knowledge sharing with Intranet 2.0}. Knowledge Management Review, 9(4).

\bibitem{Knowledge management 2.0 new focused lightweight applications rewrite the rule about KM}
{\sc S Spanbauer}. {Knowledge management 2.0: new focused, lightweight applications rewrite the rule about KM}. CIO, 20(5), 1.
\end{thebibliography}

\title{A High Resolution Interface Capturing Method With Discrete Conservation Equations for Compressible Multi-fluid Flows.}
\tocauthor{Bora Kalpakli} \author{} \institute{}
\maketitle
\begin{center}
{\large \underline{Bora Kalpakli}}\\
ROKETSAN Missile Industry\\
{\tt bkalpakli@roketsan.com.tr}
\\ \vspace{4mm}
{\large Hakan I.  Tarman}\\
Middle East Technical University\\
{\tt tarman@metu.edu.tr}
\\ \vspace{4mm}
{\large Yusuf Ozyoruk}\\
Middle East Technical University\\
{\tt yusuf.ozyoruk@ae.metu.edu.tr}
\end{center}

\section*{Abstract}
This paper reports on a novel method for compressible multi-fluid interface and mixture flow problems. Method is based on high resolution discrete conservation equations utilizing HLLC Riemann solver which is more robust and faster than iterative Riemann solvers. The solution algorithm is based on a Godunov type finite volume scheme. Godunov fluxes are calculated for each phase $\Sigma_i$ and for each possible interface $\left(\Sigma_i,\Sigma_j\right)$ for $j=0,..,N$ on the cell faces. Possible interfaces configurations are calculated by using a high resolution differencing method for volume fractions. Method is based on Normalized Variable Diagram approach and similar to high resolution schemes of Leonard \cite{The ULTIMATE conservative difference scheme applied to unsteady one-dimensional advection.} and Ubbink \cite{Numerical prediction of two fluid systems with sharp interfaces}. A new boundedness criteria based on wave propagation in phase continuums in cell volume is used in addition to availability criteria utilized  in volume fraction differencing schemes for incompressible flows. Our scheme provides  much higher interface resolution than the discrete  approach of \cite{Discrete equations for physical and numerical compressible multiphase mixtures} which is based on gradient reconstruction for determining interface configurations. This new method provides higher accuracy and robustness than available interface capturing methods for compressible multi-phase flows while less computationally expensive compared to geo-reconstruct interface tracking methods on multi-dimensional unstructured solution grids. Our scheme can be applied to all multi-phase flow regimes in the same simulation including interface problems, mixture flows and particulate flows.

\bibliographystyle{plain}
\begin{thebibliography}{10}
\bibitem{The ULTIMATE conservative difference scheme applied to unsteady one-dimensional advection.}
{\sc B.P. Leonard}. {The ULTIMATE conservative difference scheme applied to unsteady one-dimensional advection.}. Comp. Math. in Appl. Mech. and Eng. 88(1991) 1182-1197.

\bibitem{Numerical prediction of two fluid systems with sharp interfaces}
{\sc O. Ubbink}. {Numerical prediction of two fluid systems with sharp interfaces}. PhD thesis submitted to University of London (1997).

\bibitem{Discrete equations for physical and numerical compressible multiphase mixtures}
{\sc R. Abgrall and R. Saurel}. {Discrete equations for physical and numerical compressible multiphase mixtures}. J. Comput. Phys. 186 (2003) 361-396.
\end{thebibliography}

\title{Application of Sparse Tensors for Optimizing Multi-Dimensional VLSI Electromagnetic Analysis}
\tocauthor{Sandeep Koranne} \author{} \institute{}
\maketitle
\begin{center}
{\large Sandeep Koranne}\\
Mentor Graphics Corporation\\
{\tt sandeep.koranne@gmail.com}
\end{center}

\section*{Abstract}
Electromagnetic analysis of VLSI layout results in sparse matrices 
which have to be solved to calculate the capacitance of nodes. 
Modern semiconductor manufacturing involves many photo-lithography and chemical processes which 
induce \emph{in-die} process variation~$[1,2]$. Each effect is pre-characterized as a \emph{corner}. 
Each corner is defined statistically, eg., CWORST (variation which is most 
pessimistic for capacitance), CBEST (most optimistic), RCWORST (most pessimistic for $R\times C$), RCBEST etc. 
For reliability, the circuit needs to be analyzed and validated at each corner of 
this process variability. Similarly, the circuit needs to be extracted at different operating temperatures. 
Each of these configurations of process variations and
 temperatures is termed a \emph{corner}. Recently, mask-alignment variability has also become a concern as sub-wavelength 
lithography is achieved using double patterning. This effect can also be treated as a dimension in the multi-dimensional analysis.
Traditional electromagnetic analysis to extract parasitics (including resistance, ground-capacitance, 
coupling-capacitance, and inductance) have solved this multi-dimensional problem sequentially (one corner at a time). 
In this paper we apply \emph{sparse tensor} techniques~$[3]$ to solve multiple dimensions at the same time. 
We have represented the analysis problem using 3-tensor (for process variation), 4-tensor (for process and temperature variation) 
and 5-tensor (mask alignment). 
Our proposed system analyzes VLSI layout data containing millions of parasitic nodes with 
negligible cost for each additional corners. We present implementation details of 
isomorphic sparse tensors (where the non-zero structure of the tensor $R^{I\times J \times K}$ is 
determined only by $(i,j)$) as well as non-isomorphic tensors where even if $A[i,j,k]$ is zero, $A[i,j,k+1]$ can be non-zero.



\bibliographystyle{plain}
\begin{thebibliography}{10}
\bibitem{A capacitance solver for incremental variation-aware extraction}
{\sc T.~El-Moselhy undefined and I.~Elfadel undefined and and L.~Daniel.}. {A capacitance solver for incremental variation-aware extraction}.  ICCAD 2008. IEEE/ACM   International Conference on CAD, pages 662 --669, nov. 2008.

\bibitem{Enablement of Variation-Aware Timing Treatment of Parasitic   Resistance and Capacitance}
{\sc N.~Lu and J.~H. McCullen.}. {Enablement of Variation-Aware Timing: Treatment of Parasitic   Resistance and Capacitance}. ISQED '07. 8th International   Symposium on Quality Electronic Design, pages 743 --748, march 2007.

\bibitem{Tensor decompositions and applications}
{\sc T.~G. Kolda and B.~W. Bader.}. {Tensor decompositions and applications}. SIAM Review 51(3):455--500, September 2009.
\end{thebibliography}

\title{Computational Comparison of Various FEM Adaptivity Approaches}
\tocauthor{Lukáš Korous} \author{} \institute{}
\maketitle
\begin{center}
{\large Lukáš Korous}\\
Faculty of Electrical Engineering, University of West Bohemia\\
{\tt korous@rice.zcu.cz}
\\ \vspace{4mm}
{\large Pavel Kůs}\\
Faculty of Electrical Engineering, University of West Bohemia\\
{\tt pkus@rice.zcu.cz}
\\ \vspace{4mm}
{\large Pavel Karban}\\
Faculty of Electrical Engineering, University of West Bohemia\\
{\tt karban@kte.zcu.cz}
\\ \vspace{4mm}
{\large Pavel Šolín}\\
University of Nevada, Reno\\
{\tt solin@unr.edu}
\\ \vspace{4mm}
{\large František Mach}\\
Faculty of Electrical Engineering, University of West Bohemia\\
{\tt fmach@kte.zcu.cz}
\end{center}

\section*{Abstract}
Mesh adaptivity algorithms for the Finite Element Method are crucial for solving problems with no  a-priori information about its solution. They can significantly influence the results. On one hand one can get unnecessarily good results and pay for it by waiting unbearable amount of time, on the other hand by decreasing the mesh resolution to obtain the results faster, one can lose substantial details of the solution at hand. In this work, we focus on a different aspect of the algorithms - the computational aspect - for a number of test problems we will compare how various techniques compare to each other in terms of CPU time, number of instructions executed by the processor(s), number of iterations in the case of nonlinear problems, memory utilization, parallelization aspects, etc.We compare h-adaptivity, p-adaptivity, hp-adaptivity with various polynomial orders, and we also compare some approaches to handle weakly- and strongly- coupled problems (projections, multimesh). Some heuristics (precalculated integrals, inexact Jacobian) are also presented. All this work was done within the framework of the open source Finite Element library Hermes2D, for problems of various size and complexity. The underlying data structures and algorithms are described to support the conclusions.

\bibliographystyle{plain}
\begin{thebibliography}{10}
\bibitem{rbitrary-level hanging nodes and automatic adaptivity in the hp-FEM}
{\sc Solin P and Cerveny J and Dolezel I}. {rbitrary-level hanging nodes and automatic adaptivity in the hp-FEM}. Math Comput Simul 77:117–132.

\bibitem{Higher-order finite element methods}
{\sc Solin P and Segeth K and Dolezel I}. {Higher-order finite element methods}. Chapman \& Hall/CRC Press, 2003.
\end{thebibliography}

\title{Numerical Solution of Acoustic Transient Phenomena and Calculating the Coefficients of Acoustic Diffuser}
\tocauthor{Lukáš Koudela} \author{} \institute{}
\maketitle
\begin{center}
{\large \underline{Lukáš Koudela}}\\
University of West Bohemia\\
{\tt koudela@kte.zcu.cz}
\\ \vspace{4mm}
{\large Pavel Karban, Jindřich Jansa, David Pánek}\\
University of West Bohemia\\
{\tt \{karban, jansaj, panek50\}@kte.zcu.cz}
\\ \vspace{4mm}
{\large Oldřich Tureček, Ladislav Zuzjak, Martin Schlosser, Jan Altman}\\
University of West Bohemia\\
{\tt \{turecek, zuzjak, schlossi, altmanj\}@ket.zcu.cz}
\end{center}

\section*{Abstract}
The numerical modeling of the acoustic transient field is carried out. There are more goals followed. The first one is to create a computational model of an acoustic diffusor and obtain the time evolution of reflected acoustic waves. The second objective is to obtain with the help of discrete Fourier transform and according to \cite{Acoustic Absorbers and Diffusers} the diffusity and scattering coefficients describing the properties of the diffusor. 

The mathematical model is described by the partial differential wave equation with appropriate boundary conditions \cite{Modeling of Loudspeaker Using hp-Adaptive Methods}. The numerical solution is performed by a fully adaptive higher-order finite element method developed by the hp-FEM group \cite{Hermes - Higher-Order Modular Finite Element System} and used by the Agros2D application \cite{Agros2D - Multiplatform C++ Application for the Solution of PDEs} \cite{Numerical Solution of Coupled Problems Using Code Agros2D}. The results of computations of time dependence of acoustic pressure are compared with the results obtained by the commercial code Comsol Multiphysics \cite{Version 4.2} and verified by measurement. 

The experimental measurement of the acoustic diffuser is realized in the anechoic chamber to avoid the unwanted reflections from the surrounding environment.

\bibliographystyle{plain}
\begin{thebibliography}{10}
\bibitem{Acoustic Absorbers and Diffusers}
{\sc  J. T. Cox and P. D’Antonio}. {Acoustic Absorbers and Diffusers}. Theory, Design and Application, Taylor \& Francis Group, 2009.

\bibitem{Modeling of Loudspeaker Using hp-Adaptive Methods}
{\sc L. Koudela and P. Karban and O. Tureček and L. Zuzjak}. {Modeling of Loudspeaker Using hp-Adaptive Methods}. Computing, Springer Vienna, ISSN 0010-485X, 2013.

\bibitem{Hermes - Higher-Order Modular Finite Element System}
{\sc P. Solin et al}. {Hermes - Higher-Order Modular Finite Element System}. http://hpfem.org.

\bibitem{Agros2D - Multiplatform C++ Application for the Solution of PDEs}
{\sc P. Karban et al}. {Agros2D - Multiplatform C++ Application for the Solution of PDEs}. http://agros2d.org.

\bibitem{Numerical Solution of Coupled Problems Using Code Agros2D}
{\sc P. Karban and F. Mach and P. Kůs and D. Pánek and I. Doležel}. {Numerical Solution of Coupled Problems Using Code Agros2D}. Computing, Springer Vienna, ISSN 0010-485X, 2013.

\bibitem{Version 4.2}
{\sc Comsol Multiphysics}. {Version 4.2}. http://www.comsol.com.
\end{thebibliography}

\title{The Discontinuous Galerkin Galerkin Trefftz Method}
\tocauthor{Fritz Kretzschmar} \author{} \institute{}
\maketitle
\begin{center}
{\large Fritz Kretzschmar}\\
Graduate School of Computational Engineering,  Technische Universitaet Darmstadt\\
{\tt kretzschmar@gsc.tu-darmstadt.de}
\end{center}

\section*{Abstract}
In this work we present a novel Discontinuous Galerkin Finite Element Method \cite{Triangular mesh methods for the neutron transport equation} for wave propagation problems. The method employs space--time Trefftz-type basis functions \cite{Survey of trefftz-type element formulations,Ein Gegenst"uck zum Ritzschen Verfahren} that satisfy the underlying partial differential equations exactly in an element--wise fashion. The developed Trefftz-type basis functions can be of arbitrary high order; and the approximated solution in the whole space--time domain of interest $\Omega$ shows spectral convergence under $p$ enrichment for sufficiently smooth initializations. Formulating the approximation in terms of a space--time Trefftz-type basis makes high order time integration an inherent property of the method and clearly sets it apart from methods which employ a high order approximation in space only. \\\\To date we have investigated one-dimensional electromagnetic wave propagation problems for different settings. In problems like this a wave is propagating in a given direction $x$ with one-component fields $E\equiv E_y$ and $H\equiv H_z$ inside a space-time domain of interest $\Omega$. The underlying equations are non-static one-dimensional Maxwell's equations in a coordinate-independent divergence form\begin{align} \label{eqn:1D-Maxwell-compact-form}  \mathbf{\nabla}^{\mathrm{T}} \cdot \mathrm{\eta}_\mu \cdot \mathbf{F} \,=\, 0 \quad \mathrm{and} \quad \mathbf{\nabla}^{\mathrm{T}} \cdot \mathrm{\eta}_\epsilon \cdot  \mathbf{F} \,=\, 0.\end{align} Here we used the material operators $\mathrm{\eta}_\epsilon $ and $\mathrm{\eta}_\mu$ as well as the differential operator $\mathbf{\nabla}=(\partial_t,\partial_x)^{\mathrm{T}}$; the $E$ and $H$ fields are combined into one field vector $\mathbf{F} \equiv (E,H)^{\mathrm{T}}$. On this set of equations we apply Trefftz-type basis functions only. By definition, Trefftz basis functions satisfy the underlying differential equation, and the relevant interface boundary conditions, exactly, in a local sense. In the one-dimensional case transport polynomials form a Trefftz basis\begin{align}\label{eqn:transport-basis}  \mathbf{u}^{p,\pm} \, = \,   \left(\begin{array}{c}  u^{E,p,\pm} \\u^{H,p,\pm}  \end{array}\right) \, = \,  \left(\begin{array}{c}  \pm \big( x \mp v \, t \big)^p \\ \frac{1}{Z} \big( x \mp v \, t \big)^p  \end{array}\right).\end{align}  Here $Z = \sqrt{\mu  / \epsilon}$ and $v = 1 / \sqrt{\mu \epsilon}$ are the intrinsic impedance and the free space phase velocity, respectively; $\pm$ denotes the propagation direction of the simulated wave. By using these Trefftz-type basis functions we obtain spectral convergence of the approximation error in the whole space-time domain of interest even for partially filled elements. \\\\

\textbf{Acknowledgement} This is joint work with S. Schnepp (ETH Zurich), I. Tsukerman (U Akron), and T. Weiland (TU Darmstadt).



\bibliographystyle{plain}
\begin{thebibliography}{10}
\bibitem{Triangular mesh methods for the neutron transport equation}
{\sc W. Reed and T. Hill}. {Triangular mesh methods for the neutron transport equation}. Los Alamos Scientific Laboratory Report, 1973.

\bibitem{Survey of trefftz-type element formulations}
{\sc J. Jirousek and A. Zielinski}. {Survey of trefftz-type element formulations}. Comput Struct, 63(2):225 -- 242, 1997.

\bibitem{Ein Gegenst"uck zum Ritzschen Verfahren}
{\sc E. Trefftz}. {Ein Gegenst\"uck zum Ritzschen Verfahren}. Int. Kongress f\"ur Technische Mechanik, Z\"urich, 1926.
\end{thebibliography}

\title{Optimization of an Actuator With Non-Linear Materials Using Evolutionary Algorithms and Higher-Order Finite Element Modeling}
\tocauthor{Petr Krop\'{i}k} \author{} \institute{}
\maketitle
\begin{center}
{\large Petr Krop\'{i}k}\\
University of West Bohemia, Pilsen, Czech Republic\\
{\tt pkropik@kte.zcu.cz}
\\ \vspace{4mm}
{\large Lenka \v{S}roubov\'{a}}\\
University of West Bohemia, Pilsen, Czech Republic\\
{\tt lsroubov@kte.zcu.cz}
\\ \vspace{4mm}
{\large Roman Hamar}\\
University of West Bohemia, Pilsen, Czech Republic\\
{\tt hamar@kte.zcu.cz}
\end{center}

\section*{Abstract}
An electromagnetic actuator with nonlinear structural parts (permanent-magnet core, steel shell) is modeled and optimized. The first aim is to obtain the maximum possible force acting on the core at its smallest possible dimensions. The second aim is to avoid too hard collision of the core with the front end of the shell, i.e. the static characteristic of the device should be as flat as possible. The analysis of the electromagnetic field (using FEM) is based on the in-house FEM application Agros2D [1], which has implemented curvilinear elements and \textit{hp}-adaptivity. Optimization scripts are implemented in the Python language (using in-house scripts and specialized libraries (inspyred, [2]). The multi-objective optimization is used, applying the constrained Non-dominated Sorting Genetic Algorithm-II with elitism [3]. The solution accuracy, convergence and solution time is compared to the results evaluated in previuos research [4][5]. The methodology is illustrated by computations of several different electromagnetic actuator configurations whose results are discussed, illustrated and explained using graphs of results, Pareto fronts [6] etc.

\bibliographystyle{plain}
\begin{thebibliography}{10}
\bibitem{Agros2d}
{\sc P. Karban et al}. {Agros2d}. URL http://agros2d.org.

\bibitem{inspyred Bio-inspired algorithms in python}
{\sc Group of authors}. {inspyred: Bio-inspired algorithms in python}. URL http://inspyred.github.com.

\bibitem{Fast and elitist multiobjective genetic algorithm NSGA-II}
{\sc K. Deb and A. Pratap and S. Agarwal and T.A. Meyarivan}. {Fast and elitist multiobjective genetic algorithm: NSGA-II}. IEEE Transactions On Evolutionary Computation 6(2), 2002.

\bibitem{Higher-Order Finite Element Modeling and Optimization of Actuator with Non-Linear Materials}
{\sc P. Krop\'{i}k and L. \v{S}roubov\'{a} and R. Hamar}. {Higher-Order Finite Element Modeling and Optimization of Actuator with Non-Linear Materials}. ESCO 2012, Pilsen, Czech Republic, 2012.

\bibitem{Optimization of actuator with permanent magnet core from viewpoint of total magnetic force}
{\sc P. Krop\'{i}k and L. \v{S}roubov\'{a} and R. Hamar}. {Optimization of actuator with permanent magnet core from viewpoint of total magnetic force}. In: XXXV Miedzynarodowa konferencja z podstaw elektrotechniki i teorii obwodow, Gliwice, 2012.

\bibitem{Multiobjective Shape Design in Electricity and Magnetism}
{\sc {P. {Di Barba}}}. {Multiobjective Shape Design in Electricity and Magnetism}. Lecture Notes in Electrical Engineering 47, Springer, 2009.

\bibitem{The finite element method in magnetics}
{\sc M. Kuczmann and A. Ivanyi}. {The finite element method in magnetics}. Akademiai Kiado, Budapest, 2008.

\bibitem{Actuators Basics and Applications}
{\sc H. Janocha}. {Actuators, Basics and Applications}. Springer, New York, 2004.
\end{thebibliography}

\title{Optimization of Electrical Properties of Parallel Plate Antena for EMC Testing}
\tocauthor{Zdeněk Kubík} \author{} \institute{}
\maketitle
\begin{center}
{\large \underline{Zdeněk Kubík}}\\
University of West Bohemia\\
{\tt zdekubik@kae.zcu.cz}
\\ \vspace{4mm}
{\large Denys Nikolayev}\\
University of West Bohemia\\
{\tt d@deniq.com}
\\ \vspace{4mm}
{\large Miroslav Hromádka}\\
University of West Bohemia\\
{\tt mhromadk@kee.zcu.cz }
\\ \vspace{4mm}
{\large Pavel Karban}\\
University of West Bohemia\\
{\tt karban@kte.zcu.cz}
\\ \vspace{4mm}
{\large Jiří Skála}\\
University of West Bohemia\\
{\tt skalaj@kae.zcu.cz}
\end{center}

\section*{Abstract}
Parallel plate antennas (or improved TEM-cell, strip line antenna, etc.) are used in electromagnetic compatibility for a susceptibility testing. The well-designed antenna represents transmission line with transverse electromagnetic (TEM) wave propagation and the electromagnetic field between plates is uniform [1]. In the area with the uniform field are placed tested devices. This paper deals with optimization of electric properties of small parallel plate antenna, which was designed for testing of telecommunication devices. 
The optimization is based at numerical solution, where a fully adaptive high-order finite elements method was used. The results were compared with the results obtained by using the commercial software and the results of the original parallel plate antenna were verified by experimental measurement.

\bibliographystyle{plain}
\begin{thebibliography}{10}
\bibitem{Generation of Standard EM Fields Using TEM Transmission Cells}
{\sc Myron N.Crawford}. {Generation of Standard EM Fields Using TEM Transmission Cells}. IEEE Transactions on Electromagnetic Compatibility, Vol. EMC-16, No. 4, pp. 189-195, 1974.
\end{thebibliography}

\title{An Application of the Schwarz Domain Decomposition Method on the Nonlinear Richards Equation Problem}
\tocauthor{Michal Kuraz} \author{} \institute{}
\maketitle
\begin{center}
{\large \underline{Michal Kuraz}}\\
Czech University of Life Sciences Prague, Faculty of Environmental Sciences, Department of Water Resources and Environmental Modeling, Czech Republic\\
{\tt kuraz@fzp.czu.cz}
\\ \vspace{4mm}
{\large Petr  Mayer}\\
Czech Technical University in Prague, Faculty of Civil Engineering, Department of Mathematics, Czech Republic\\
{\tt petr.mayer@fsv.cvut.cz}
\end{center}

\section*{Abstract}
Modeling of the transport processes in a vadose zone is an important role for predicting the reactions of soil biotopes to
anthropogenic activity, e.g. modeling the contaminant transport, effect of soil water regime to changes in soil structure and
composition, etc. The water flow is governed by Richards equation, the constitutive laws are typically supplied by van
Genuchten's model, that could be understood as a pore size distribution function. Certain material with dominantly uniform
pore sizes (e.g. coarse-grained material) could exhibit ranges of the constitutive function values within several orders of magnitude,
possibly beyond the computer real numbers length. And thus the numerical approximation of Richards equation often requires
solving the systems of equations that cannot be solved on a computer arithmetic. An appropriate decomposition of the domain
into sub-domains that cover only a limited range of the constitutive function values, and that will adaptively change reflecting
the time progress of the model, will enable an effective and reliable solution of this problem.

This problem has been already partially treated in Kuraz et. al. [1]. This paper is focused to improving the performance of the nonlinear solver by locating the areas with  abrupt changes in the solution.


\bibliographystyle{plain}
\begin{thebibliography}{10}
\bibitem{Domain decomposition adaptivity for the Richards equation model}
{\sc Michal Kuraz and Petr Mayer and Vojtech Havlicek and Pavel Pech}. {Domain decomposition adaptivity for the Richards equation model}. Computing, In Press.
\end{thebibliography}

\title{Solution of Transient Partial Differential Equations Using Time-adaptive Methods in General-purpose Software}
\tocauthor{Pavel  Kůs} \author{} \institute{}
\maketitle
\begin{center}
{\large Pavel  Kůs}\\
Department of Theory of Electrical Engineering, University of West Bohemia, Czech Republic\\
{\tt pkus@rice.zcu.cz}
\\ \vspace{4mm}
{\large Pavel Karban}\\
Department of Theory of Electrical Engineering, University of West Bohemia, Czech Republic\\
{\tt karban@kte.zcu.cz}
\\ \vspace{4mm}
{\large Lukáš Korous}\\
Department of Theory of Electrical Engineering, University of West Bohemia, Czech Republic\\
{\tt korous@rice.zcu.cz}
\\ \vspace{4mm}
{\large František Mach}\\
Department of Theory of Electrical Engineering, University of West Bohemia, Czech Republic\\
{\tt fmach@kte.zcu.cz}
\end{center}

\section*{Abstract}
Time adaptivity plays an essential role in the solution of time-dependent partial differential equations. There are many problems that exhibit rapid changes of behavior in some parts of the time domain and do not change significantly otherwise. Solving such problems using fixed-step methods would be very ineffective, since the step would have to be very small to capture fast changes. We compare several methods based on a-posteriori error estimates obtained by using two methods of different accuracy. As for most adaptive methods, a tolerance has to be provided. This is a major inconvenience for a software user, since a reasonable value of the tolerance may be difficult to determine prior to the calculation. Our goal is to design a method, that would use approximately a prescribed number of steps and change the length of steps in such a way, that the final solution is as accurate as possible. This makes the method useful for general-purpose engineering software, like Agros2D. In this paper we present such methods and compare them in terms of CPU running time.

\bibliographystyle{plain}
\begin{thebibliography}{10}
\bibitem{Higher-Order Finite Element Methods}
{\sc P. Šolín and K. Segeth and I. Doležel}. {Higher-Order Finite Element Methods}. CRC Press, Boca Raton, 2003.

\bibitem{Numerical solution of coupled problems using code Agros2D}
{\sc P. Karban and F. Mach and P. Kůs and D. Pánek and I. Doležel}. {Numerical solution of coupled problems using code Agros2D}. Computing, accepted 2013.

\bibitem{Adaptive backward difference formula-Discontinuous Galerkin finite element method for the solution of conservation laws}
{\sc V. Dolejší and P. Kůs}. {Adaptive backward difference formula-Discontinuous Galerkin finite element method for the solution of conservation laws}. Int J Numer Methods Eng  73(12), 2008, pp. 1739-1766.
\end{thebibliography}

\title{Hierarchical Slope Limiting in  Discontinuous Galerkin Methods}
\tocauthor{Dmitri Kuzmin} \author{} \institute{}
\maketitle
\begin{center}
{\large Dmitri Kuzmin}\\
University Erlangen-Nuremberg\\
{\tt kuzmin@am.uni-erlangen.de}
\end{center}

\section*{Abstract}
This talk is concerned with the design of multiscale discontinuous Galerkin (DG) methods for hyperbolic conservation laws. A discontinuous (P0) or continuous (P1/Q1) coarse-scale approximation is enriched with discontinuous basis functions of higher order. Unresolvable fine-scale features are eliminated using a hierarchical slope limiting procedure or by solving an inequality-constrained least squares problem. The bounds for each derivative at a mesh vertex are defined using the average values in surrounding elements. An unconditionally stable implicit algorithm is developed for steady state computations. The nonlinear system is decomposed into a global problem for the coarse scale components and small local problems for the fine scale components. The interscale transfer operators provide a two-way iterative coupling between the solutions to the global and local problems. A new approach to slope limiting for systems of conservation laws is proposed. Numerical studies are performed for scalar convection problems and for the compressible Euler equations in 2D.


\bibliographystyle{plain}
\begin{thebibliography}{10}
\bibitem{ A vertex-based hierarchical slope limiter for p-adaptive discontinuous Galerkin methods}
{\sc D. Kuzmin}. { A vertex-based hierarchical slope limiter for p-adaptive discontinuous Galerkin methods}.  J.  Comput.  Appl.  Math.  233  (2010) 3077-3085.

\bibitem{A multiscale discontinuous Galerkin  method with the computational structure of a continuous Galerkin method}
{\sc  T.J.R. Hughes and G. Scovazzi and P.B. Bochev and A. Buffa}. {A multiscale discontinuous Galerkin  method with the computational structure of a continuous Galerkin method}. Comp.  Meth.  Appl.  Mech.  Engrg.  195  (2006) 2761-2787.
\end{thebibliography}

\title{Control of Sound Radiation From Vibrating Structures With a Piezoelectric Shunt Damping: A Coupled FE/BE Formulation}
\tocauthor{Walid Larbi} \author{} \institute{}
\maketitle
\begin{center}
{\large Walid Larbi}\\
Conservatoire National des Arts et M\'{e}tiers, Paris, France\\
{\tt walid.larbi@cnam.fr}
\end{center}

\section*{Abstract}
During the last two decades there has been an accelerating level of interest in the control of noise radiation and sound transmission from vibrating structures by active piezoelectric techniques in the low frequency range. In this context, resonant shunt damping techniques have been recently used for interior structural-acoustic problems [1, 2]. The present work concerns the extension of this technique to external vibroacoustic problems using an integrated finite-element/boundary-element method (FEM/BEM) for the numerical resolution of the fully coupled electro-mechanical-acoustic system.\\  

First, a finite element formulation of an elastic structure with surface-mounted piezoelectric patches and subjected to pressure load due to the presence of an external fluid is derived from a variational principle involving structural displacement, electrical voltage of piezoelectric elements and acoustic pressure at the fluid-structure interface. This formulation, with only one couple of electric variables per patch, is well adapted to practical applications since realistic electrical boundary conditions, such that equipotentiality on the electrodes and prescribed global electric charges, naturally appear. The global charge/voltage variables are intrinsically adapted to include any external electrical circuit into the electromechanical problem and to simulate the effect of resistive or resonant shunt techniques.\\

In the second part of this work, the direct boundary element method is used for modeling the scattering/radiation of sound by the structure submerged in acoustic domain. The BEM is derived from Helmholtz integral equation involving the surface pressure and normal acoustic velocity at the boundary of the acoustic domain. A compatible mesh at the fluid-structure interface is considered. The coupled FE-BE model is obtained by using a compatible mesh at the interface. The present coupling procedure is quite general and suitable for modeling any three-dimensional geometry for bounded and especially unbounded structural-acoustic radiation problems.\\

Finally, the efficiency of the proposed coupling methodology is demonstrated on two examples. First, the vibration reduction of an elastic plate backed by a closed acoustic cavity is analyzed. For this example, a complete FE method developed by the authors in [1] is compared to the present FEM/BEM approach.  The second example is the simulation of the attenuation of the sound field emitted from a submerged plate in an acoustic domain by means of a piezoelectric shunt system. 


\bibliographystyle{plain}
\begin{thebibliography}{10}
\bibitem{Finite element formulation of smart piezoelectric composite plates coupled with acoustic fluid}
{\sc W. Larbi and J.-F. De\"{u} and R. Ohayon}. {Finite element formulation of smart piezoelectric composite plates coupled with acoustic fluid}. Composite Structures 94 (2) ( 2012), 501-509.

\bibitem{Structural-acoustic vibration reduction using switched shunt piezoelectric patches a finite element analysis}
{\sc W. Larbi and J.-F. De\"{u} and M. Ciminello and R. Ohayon}. {Structural-acoustic vibration reduction using switched shunt piezoelectric patches: a finite element analysis}. Journal of Vibration and Acoustics 132(5) (2010), 051006 (9 pages).
\end{thebibliography}

\title{Adaptive Multiscale Finite Element Methods Based on Reduced Order Fine Scale Approximation}
\tocauthor{Mats G. Larson} \author{} \institute{}
\maketitle
\begin{center}
{\large Mats G. Larson}\\
Mathematics, Umea University, Sweden\\
{\tt mats.larson@math.umu.se}
\\ \vspace{4mm}
{\large Tor Troeng}\\
Mathematics, Umea University, Sweden\\
{\tt tor.troeng@math.umu.se}
\end{center}

\section*{Abstract}
We present a multiscale finite element method based on the variational multiscale method together with a component mode synthesis representation for the fine scale part of the solution. More precisely, the fine scale is expanded in terms of numerically computed eigenmodes associated with the coarse scale elements and edges. The eigenmodes captures fine scale features of the solution and the accuracy in the fine scale representation is determined by the number of modes included in the expansion. We present an a posteriori error estimate in the energy norm for the error in the multiscale method, compared to the full Galerkin solution on the fine scale, which measures the effect of reduction of the fine scale representation. Based on the error estimates we develop adaptive algorithms for automatic tuning of the local number of modes in the fine scale representation. Finally, we present numerical examples confirming the theoretical results.

\bibliographystyle{plain}
\begin{thebibliography}{10}
\bibitem{Adaptive Component Mode Synthesis in Linear Elasticity}
{\sc H. Jakobsson and F. Bengzon and and M. G. Larson}. {Adaptive Component Mode Synthesis in Linear Elasticity}. Int. J. Numer. Meth. Engng. 200.41-44 (2011) 829-844.

\bibitem{Adaptive Variational Multiscale Methods based on A Posteriori Error Estimation Energy Norm Estimates for Elliptic Problems}
{\sc M. G. Larson and A. Malqvist}. {Adaptive Variational Multiscale Methods based on A Posteriori Error Estimation: Energy Norm Estimates for Elliptic Problems}. Comput. Meth. Appl. M. 196.21 (2007) 2313-2324.

\bibitem{A Reduced Order Multiscale Finite Element Method based on Component Mode Synthesis}
{\sc T. Troeng and H. Jakobsson and M. G. Larson}. {A Reduced Order Multiscale Finite Element Method based on Component Mode Synthesis}. Technical report (2013).
\end{thebibliography}

\title{DGTD  Method  on  Non-conforming  Structured-unstructured  Meshes  for Nanophotonics}
\tocauthor{Clément DurochatStéphane Lanteri} \author{} \institute{}
\maketitle
\begin{center}
{\large Clément Durochat}\\
Inria Sophia Antipolis-Méditerranée\\
{\tt Clement.Durochat@inria.fr}
\\ \vspace{4mm}
{\large \underline{Stéphane Lanteri}}\\
Inria Sophia Antipolis-Méditerranée\\
{\tt Stephane.Lanteri@inria.fr}
\\ \vspace{4mm}
{\large Raphaël Léger}\\
Inria Sophia Antipolis-Méditerranée\\
{\tt Raphael.Leger@inria.fr}
\\ \vspace{4mm}
{\large Claire Scheid}\\
University of Nice-Sophia Antipolis and Inria Sophia Antipolis-Méditerranée\\
{\tt Claire.Scheid@unice.fr}
\\ \vspace{4mm}
{\large Jonathan Viquerat}\\
Inria Sophia Antipolis-Méditerranée\\
{\tt Jonathan.Viquerat@inria.fr}
\end{center}

\section*{Abstract}
Nanophotonics is concerned with the behavior of light at the nanometer
scale.   It is  considered as  a branch  of optical  engineering which
deals  with optics,  or the  interaction  of light  with particles  or
substances,  at deeply  subwavelength length  scales.  Because  of its
numerous scientific and  technological applications (e.g.  in relation
to    telecommunication,   energy    production    and   biomedicine),
nanophotonics  represents  an active  field  of research  increasingly
relying  on  numerical  modeling  beside  experimental  studies.   The
numerical  study of electromagnetic  wave propagation  in nanophotonic
devices  requires  among  others  the integration  of  an  appropriate
dispersion model,  such as the  Drude or the Drude-Lorentz  models, in
numerical methodologies.  Such a  dispersion model allows to establish
a dependency  between the  electrical permittivity and  the electrical
conductivity of the material and the angular frequency of the incident
electromagnetic wave.  In practice one  has to deal with the numerical
treatment  of the  system  of time-domain  Maxwell  equations for  the
electromagnetic  field,  coupled to  a  system  of auxiliary  ordinary
differential  equations  modeling  the  electrical polarization  or  a
polarization current, depending on the considered dispersion mdoel.

A lot of methods have been developed for the numerical solution of the
time-domain Maxwell  equations.  Finite difference  time-domain (FDTD)
methods based  on Yee's  scheme (a time  explicit method defined  on a
staggered mesh)  are still prominent  because of their  simplicity and
their  non-dissipative  nature   (they  hold  an  energy  conservation
property which is an  important ingredient in the numerical simulation
of  unsteady  wave  propagation  problems).   When  it  comes  to  the
simulation  of realistic  nanophotonic applications,  the  FDTD method
raises several important limitations, essentially  due to the use of a
(structured) cartesian  discretization grid.   In the last  ten years,
the  Discontinuous-Galerkin  Time-Domain  method  (DGTD)  has  met  an
increased  interest in  the purpose  of simulating  complex industrial
problems.  Indeed,  these methods somehow  can be seen as  a crossover
between FETD methods (their precision depends of the order of a chosen
local  polynomial basis upon  which the  solution is  represented) and
FVTD  methods  (the  neighboring  cells  are  connected  by  numerical
fluxes).   Thus, DGTD  methods offer  a wide  range of  flexibility in
terms  of geometry  (as  the use  of  unstructured and  non-conforming
meshes  is naturally  permitted)  as well  as  local order  refinement
strategies, which are of useful practical interest.

In  the  present  work,  we  are  concerned with  the  study  and  the
development  of a  non-conforming  multi-element DGTD  method for  the
solution  of  the  3D  time-domain  Maxwell equations  coupled  to  an
appropriate  dispersion model  for metals  at frequencies  relevant to
nanophotonic  applications.   By  doing  so,  we  are  aiming  at  the
possibility  of   discretizing  the  close   neighborhood  of  complex
geometries using  an unstructured tetrahedral mesh  while treating the
rest  of  the domain  using  a uniform  cartesian  grid,  in order  to
decrease  the   needs  in  computational  resources   for  the  target
applications.

\bibliographystyle{plain}
\begin{thebibliography}{10}
\bibitem{Numerical solution of initial boundary value problems involving Maxwell's equations in isotropic media}
{\sc K.S. Yee}. {Numerical solution of initial boundary value problems involving Maxwell's equations in isotropic media}. IEEE Trans. Antennas and Propagat. 14(3) (1966) 302-307.

\bibitem{Convergence and stability of a discontinuous Galerkin time-domain method for the heterogeneous Maxwell equations on unstructured meshes}
{\sc L. Fezoui and S. Lanteri and S. Lohrengel and S. Piperno}. {Convergence and stability of a discontinuous Galerkin time-domain method for the heterogeneous Maxwell equations on unstructured meshes}. ESAIM: Math. Model. and Numer. Anal. 39(6) (2006) 1149-1176.
\end{thebibliography}

\title{Recend Advances in Mathematical Modeling and Analysis of Wave Propagation in Metamaterials}
\tocauthor{Jichun Li} \author{} \institute{}
\maketitle
\begin{center}
{\large Jichun Li}\\
University of Nevada Las Vegas\\
{\tt jichun@unlv.nevada.edu}
\end{center}

\section*{Abstract}
Since 2000, the study of metamaterials have
become a very hot research topic across many
disciplines. In this talk, I'll first present an overview of
the short history of metamaterials, followed by some
popular governing equations for metamaterial
simulations. Then I'll present some time-domain
finite element schemes we developed recently for
solving metamaterial Maxwell's equations. Finally,
interesting numerical results will be presented to
demonstrate the re-focusing property and invisibility
cloak produced by metamaterials.

\bibliographystyle{plain}
\begin{thebibliography}{10}
\bibitem{Time-Domain Finite Element Methods for Maxwell's Equations in Metamaterials}
{\sc J. Li and Y. Huang}. {Time-Domain Finite Element Methods for Maxwell's Equations in Metamaterials}.  Springer Series in Computational Mathematics, vol.43, Springer, Dec. 2012..

\bibitem{Developing a timedomain finiteelement method for modeling of electromagnetic cylindrical cloaks}
{\sc J. Li and Y. Huang and W. Yang and  }. {Developing a timedomain finiteelement method for modeling of electromagnetic cylindrical cloaks,}. Journal of Computational Physics 231 (2012) 2880-2891..
\end{thebibliography}

\title{Multi-scale Simulations for Dusty Gas Flows}
\tocauthor{Shengtai Li} \author{} \institute{}
\maketitle
\begin{center}
{\large Shengtai Li}\\
Los Alamos National Laboratory\\
{\tt sli@lanl.gov}
\end{center}

\section*{Abstract}
We solve a dusty gas flow problem that consists of compressible Euler equations for the gas coupled to a similar system of equations for the dust. These set of equations are coupled via drag terms, which requires much shorter timescale than the dynamical timescale of the gas flow. We propose a method to bridge these two timescales without resort to small time steps and speed up the simulation by at least two-order of magnitude. 

The dust can also diffuse in the gaseous flow due to the turbulence. We propose a new set of consistent equations for the dust to remove the instability caused by the diffusive transport of the dust. Numerical results demonstrate the effectiveness of our numerical method for a dusty accretion disk problem. 

\bibliographystyle{plain}
\begin{thebibliography}{10}
\bibitem{Dust Distrubtion in a Protoplanetary Disk}
{\sc S. Li and H. Li}. {Dust Distrubtion in a Protoplanetary Disk}. to be submitted.
\end{thebibliography}

\title{On the Stability and Accuracy of a Compact Scheme for 3D Acoustic Wave Equation}
\tocauthor{Wenyuan Liao} \author{} \institute{}
\maketitle
\begin{center}
{\large Wenyuan Liao}\\
University of Calgary\\
{\tt wliao@ucalgary.ca}
\end{center}

\section*{Abstract}
In this work we propose an efficient high-order compact finite difference scheme for solving the three-dimensional acoustic wave equation.
Combined with the alternating direction implicit(ADI) technique and Pade approximation, the standard second-order finite difference scheme has been improved to fourth-order and solved as a sequence of one-dimensional problems with high computational efficiency. Stability analysis shows that the new scheme is conditionally stable  and superior to some existing methods in terms of the Courant-Friedrichs-Lewy (CFL) condition. Three numerical examples are solved to demonstrate the accuracy and efficiency of the new method.

\bibliographystyle{plain}
\begin{thebibliography}{10}
\bibitem{An efficient fourth-order low dispersive finite difference scheme for  2-D acoustic wave equation}
{\sc S. Das and W. Liao and A. Gupta}. {An efficient fourth-order low dispersive finite difference scheme for  2-D acoustic wave equation}. Submitted to Journal of Computational and Applied Mathematics.
\end{thebibliography}

\title{A Model-based Approach to Computing and Quantifying Uncertainty in the Energy Spectrum of Fusion Neutrons}
\tocauthor{Aaron Luttman} \author{} \institute{}
\maketitle
\begin{center}
{\large \underline{Aaron Luttman}}\\
National Security Technologies, LLC\\
{\tt luttmaab@nv.doe.gov}
\\ \vspace{4mm}
{\large Micha\l\, Odyniec}\\
National Security Technologies, LLC\\
{\tt odyniem@nv.doe.gov}
\end{center}

\section*{Abstract}
The energy spectrum of neutrons created in a controlled fusion reaction can be computed from detectors placed at a variety of distances from the fusion source via an inverse Radon transform. The signals seen at the detectors are noisy, not always perfectly consistent with each other, and represent only a small portion of the Radon domain, making the inverse problem of computing the contaminated, sparse inverse Radon transform quite difficult. We propose a functional model for the energy spectrum and derive the corresponding form of the signals seen by the detectors in the Radon domain. Computationally, we will present two methods for determining optimal model parameters and quantifying their uncertainty. The first is a stochastic least squares approach based on sampling the parameter space and computing a posterior distribution on the optimized parameters, and the second is a Markov Chain Monte Carlo with intrinsic uncertainty estimates. We will present results on synthetic data contaminated with both Gaussian and Poisson noises and on real neutron signals from a dense plasma focus controlled fusion reactor.

 This work was done by National Security Technologies, LLC, under Contract No. DE-AC52-06NA25946 with the U.S. Department of Energy. DOE/NV/25946--1665

\bibliographystyle{plain}
\begin{thebibliography}{10}
\bibitem{Fusion Neutron Energies and Spectra}
{\sc H. Brysk}. {Fusion Neutron Energies and Spectra}. Plasma Physics 15 (1973) 611-617.

\bibitem{Neutron Energy Distribution Function Reconstructed from Time of Flight Signals in Deuterium Gas-Puff z-Pinch}
{\sc D. Klir et al}. {Neutron Energy Distribution Function Reconstructed from Time of Flight Signals in Deuterium Gas-Puff z-Pinch}. IEEE Trans. Plasma Sci. 37 (2009) 425-432.

\bibitem{Evaluation of Reconstruction Methods for Time-Resolved Spectroscopy of Short-pulsed Neutron Sources}
{\sc I. Tiseanu and T. Craciunescu}. {Evaluation of Reconstruction Methods for Time-Resolved Spectroscopy of Short-pulsed Neutron Sources}. Nuclear Sci. Eng. 122 (1996) 384-394.
\end{thebibliography}

\title{Space-time Adaptive FEM Simulation of Corona Discharge Problems}
\tocauthor{Franti\v{s}ek Mach} \author{} \institute{}
\maketitle
\begin{center}
{\large Franti\v{s}ek Mach}\\
University of West Bohemia\\
{\tt fmach@kte.zcu.cz}
\\ \vspace{4mm}
{\large Pavel Karban}\\
University of West Bohemia\\
{\tt karban@kte.zcu.cz}
\\ \vspace{4mm}
{\large Pavel K\r{u}s}\\
University of West Bohemia\\
{\tt pkus@kte.zcu.cz}
\\ \vspace{4mm}
{\large Ivo Dole\v{z}el}\\
University of West Bohemia\\
{\tt idolezel@kte.zcu.cz}
\end{center}

\section*{Abstract}
The time-domain problem of direct-current corona discharge in air is analyzed. Its mathematical model is described by a system of second-order partial differential equations that are solved numerically by a higher-order finite element method. With respect to the input data and expected results (places with a high concentration of electron or positive ions), the solution will be realized using fully automatic space and time adaptivities whose algorithms are implemented into our own codes Agros2D and Hermes. The methodology is illustrated by a typical example from the domain of high-voltage practice (investigation of electric processes in a high-voltage switching device), and its results are discussed.

\bibliographystyle{plain}
\begin{thebibliography}{10}
\bibitem{Charging of polymeric surfaces by positive impulse corona}
{\sc S. Kumara and Y. V. Serdyuk and S. M. Gubanski}. {Charging of polymeric surfaces by positive impulse corona}. IEEE Transactions on Dielectrics and Electrical Insulation, Volume 16, ISSN 10709878, 2009.

\bibitem{Numerical simulation and experimental validation of a direct current air corona discharge under atmospheric pressure}
{\sc F. Wu and R. Liao and X. Liu and F. Yang and L. Yang and Z. Zhou and Y. Luo}. {Numerical simulation and experimental validation of a direct current air corona discharge under atmospheric pressure}. Chinese Physics B, Volume 21, ISSN 16741056, 2012.

\bibitem{Numerical solution of coupled problems using code Agros2D}
{\sc P. Karban and F. Mach and P. K\r{u}s and D. P\'{a}nek and I Dole\v{z}el}. {Numerical solution of coupled problems using code Agros2D}. Computing, Springer Vienna, ISSN 0010-485X, 2013.

\bibitem{Hermes - Higher-Order Modular Finite Element System (User's Guide)}
{\sc P. Solin et al.}. {Hermes - Higher-Order Modular Finite Element System (User's Guide)}. http://hpfem.org/.
\end{thebibliography}

\title{Estimating Dynamic Surface Position From Sparse Interferometric Measurements}
\tocauthor{Eric Machorro} \author{} \institute{}
\maketitle
\begin{center}
{\large Eric Machorro}\\
National Security Technologies, LLC\\
{\tt machorea@nv.doe.gov}
\end{center}

\section*{Abstract}
{\bf{DOE/NV/25946--1664.}}  Photonic Doppler Velocimetry (PDV) is a form of laser interferometry whereby laser light is shone on a surface, and surface velocity is computed based on the frequency shift in the reflected light. The quantity measured is a voltage on an oscilloscope, and there are many different ways to compute from voltage the velocity or position of the point where the laser hit the surface. Precise estimation of the position of the entire surface would require densely sampled points between which to interpolate, but this is not practical since each measurement point corresponds to a distinct laser. In this work we have developed a method using Local Polynomial Approximation (LPA) that allows us to estimate the position of the entire surface from a small number of points and thus a small number of laser and corresponding PDV probes. The strength of this particular technique is that there is a natural statistical hypothesis test associated with the LPA that assigns error bars to the surface positions, directly quantifying the associated uncertainty. We will present the developed algorithms as well as computational results with data from a laboratory experiment designed to benchmark the computed surface positions against positions measured by a high-speed camera.{\it This work was done by National Security Technologies, LLC, under Contract No. DE-AC52-06NA25946 with the U.S. Department of Energy.}

\bibliographystyle{plain}
\begin{thebibliography}{10}
\bibitem{Accuracy and precision in photonic Doppler velocimetry}
{\sc D. H. Dolan}. {Accuracy and precision in photonic Doppler velocimetry,}. Rev. Sci. Instru- ments, 81 (2010), no. 5, 053905 - 053905-7..

\bibitem{Photonic doppler velocimetry in shock physics experiments}
{\sc P. Mercier and J. Benier and A. Azzolina and J.M. Lagrange and D. Partouche and undefined undefined}. {Photonic doppler velocimetry in shock physics experiments}. J. Phys. IV France, 134 (2006), 805-812..

\bibitem{Push-pull analysis of photonic Doppler velocimetry measurements}
{\sc D. H. Dolan and S. C. Jones}. {Push-pull analysis of photonic Doppler velocimetry measurements,}. Rev. Sci. Instruments, 78 (2007), no. 7, 076102 - 076102-3..
\end{thebibliography}

\title{A Potential Development of Halitosis (Bad Breath) Gas Sensor Using an Open Path Fibre Technique}
\tocauthor{Hadi Manap} \author{} \institute{}
\maketitle
\begin{center}
{\large Hadi Manap}\\
University of Malaysia Pahang (UMP)\\
{\tt hadi@ump.edu.my}
\end{center}

\section*{Abstract}
Halitosis or bad breath is normally measured to diagnose dental hygiene in clinical dentistry. The main chemical constituents of oral odorous chemicals are volatile organic compounds such as Methyl Mercaptan. Human beings are sensitive to halitosis in others but unable to assess the halitosis in their own breath. There are many optical breath sensors have been investigated and developed but they are for different kind of breath analysis usage. Morisawa and Muto [1] has developed simple breathing condition sensor to measure humidity in breathing gases. Lewicki et al [2] has developed breath sensor to detect ammonia due to the presence of bacteria in the stomach. Breath sensors for halitosis are also reported and developed previously but they are using different technology such as MEMs and CMOS sensor which have their own drawbacks as discussed in previous paper [3]. Therefore a development of a new breath sensor using an optical fibre based sensor is necessary as an alternative to the current sensors.

\bibliographystyle{plain}
\begin{thebibliography}{10}
\bibitem{A novel breathing condition sensor using plastic optical fiber}
{\sc S. Morisawa and M. Muto }. {A novel breathing condition sensor using plastic optical fiber}. Proc. IEEE Sensors, vol. 3, p.1277 , 2004.

\bibitem{Real Time Ammonia Detection in Exhaled Human Breath with a Quantum Cascade Laser Based Sensor}
{\sc R. Lewicki and A. A. Kosterev and Y. A.}. {Real Time Ammonia Detection in Exhaled Human Breath with a Quantum Cascade Laser Based Sensor}. Optical Society of America, paper CMS6, 2009.

\bibitem{Integrated MEMS in Conventional CMOS}
{\sc Fedder  and Gary K}. {Integrated MEMS in Conventional CMOS}. Robotics Institute. Paper 309, 1997.
\end{thebibliography}

\title{A Galerkin Finite Element Domain Decomposition Technique and Its Application in Conservation Problems}
\tocauthor{Bradley McCaskill} \author{} \institute{}
\maketitle
\begin{center}
{\large \underline{Bradley McCaskill}}\\
University of Wyoming\\
{\tt bmccaski@uwyo.edu}
\\ \vspace{4mm}
{\large Lawrence Bush}\\
University of Wyoming\\
{\tt lbush4@uwyo.edu}
\\ \vspace{4mm}
{\large Victor Ginting}\\
University of Wyoming\\
{\tt vginting@uwyo.edu}
\end{center}

\section*{Abstract}
In this presentation, we develop a nonoverlapping iterative domain decomposition technique for solving elliptic boundary value problems that employs Galerkin Finite Element Method. A checkerboard procedure is used to solve the global problem where every iteration only involves exchanging information at the interfaces of the subdomains in the form of Robin boundary conditions. In the interest of making the whole calculation more efficient, we propose to represent the approximate solution as a set of multiscale basis functions that are associated with each subdomain. Moreover, it is well known that Galerkin Finite Element solutions do not possess the local conservation property. Based on the approximate solution obtained from this method, we devise a simple postprocessing technique to find conservative flux at the finest scale on the global domain. We then use this flux to solve a transport equation involving a first order hyperbolic problem. A set of numerical examples are presented to illustrate the performance of the method.

\bibliographystyle{plain}
\begin{thebibliography}{10}
\bibitem{On the Application of the Contionus Galerkin Finite Element Method for Conservation Problems}
{\sc L. Bush and V. Ginting}. {On the Application of the Contionus Galerkin Finite Element Method for Conservation Problems}. submitted".

\bibitem{A Parallel Iterative Procedure Applicable to the Approximate Solution of Second Order Partial Differential Equations by Mixed Finite Element Methods}
{\sc J. Douglas Jr. and P. Paes-Leme and J. Roberts and J. Wang}. {A Parallel Iterative Procedure Applicable to the Approximate Solution of Second Order Partial Differential Equations by Mixed Finite Element Methods}. Numer. Math., 65(1):95-108, 1993.
\end{thebibliography}

\title{Riemann Solutions for Spacetime Discontinuous Galerkin Methods}
\tocauthor{Scott Miller} \author{} \institute{}
\maketitle
\begin{center}
{\large \underline{Scott Miller}}\\
Applied Research Lab, Penn State University\\
{\tt scott.miller@psu.edu}
\\ \vspace{4mm}
{\large Reza Abedi}\\
The University of Tennessee, Space Institute\\
{\tt rabedi@utsi.edu}
\end{center}

\section*{Abstract}
Spacetime discontinuous Galerkin finite element methods [1--3] rely on `target fluxes' on element boundaries that are computed via local one-dimensional Riemann solutions in the direction normal to element face.  In this work,
we demonstrate a generalized solution procedure for linearized hyperbolic systems based on diagonalisation of the governing system of partial differential equations.  We show that source terms do not influence the Riemann solution in the spacetime setting.  We provide details for implementation of coordinate transformations and Riemann solutions.  Exact Riemann solutions for some linear systems of equations are provided as examples.

\bibliographystyle{plain}
\begin{thebibliography}{10}
\bibitem{An h-adaptive spacetime-discontinuous Galerkin method for linearized elastodynamics}
{\sc R. Abedi and R.B. Haber and S. Thite and J. Erickson}. {An h-adaptive spacetime-discontinuous Galerkin method for linearized elastodynamics}. Revue Européenne des Eléments Finis, 2005.

\bibitem{A spacetime discontinuous Galerkin method for hyperbolic heat conduction}
{\sc S.T. Miller and R.B. Haber}. {A spacetime discontinuous Galerkin method for hyperbolic heat conduction}. CMAME 198:2 (2008) 194--209.

\bibitem{An adaptive spacetime discontinuous Galerkin method for cohesive models of elastodynamic fracture}
{\sc R. Abedi and M.A. Hawker and R.B. Haber and K. Matou{\v{s}}}. {An adaptive spacetime discontinuous Galerkin method for cohesive models of elastodynamic fracture}. IJNME 81:10 (2010) 1207--1241.
\end{thebibliography}

\title{High-order DG Methods: Scalable Algorithms and Performance for Electromagnetics Applications}
\tocauthor{Misun Min} \author{} \institute{}
\maketitle
\begin{center}
{\large Misun Min}\\
Argonne National Laboratory\\
{\tt mmin@mcs.anl.gov}
\end{center}

\section*{Abstract}
I will present high-order spectral element discontinuous Galerkin methods and algorithm development for solving electromagnetics problems arising in nanoscience applications and particle accelerator modeling. 

Discussion will include performance analysis for large scale simulations on the leadership-class computing systems, including recent development on hybrid MPI/threads approach. 



\bibliographystyle{plain}
\begin{thebibliography}{10}
\bibitem{An efficient high-order time integration method for spectral-element discontinuous Galerkin simulations in electromagnetics}
{\sc M. Min and P. Fischer}. {An efficient high-order time integration method for spectral-element discontinuous Galerkin simulations in electromagnetics}. J. Sci. Comp., under revision, 2013.
\end{thebibliography}

\title{Performance of $Hp$-Adaptive Strategies for Elliptic Partial Differential Equations}
\tocauthor{William Mitchell} \author{} \institute{}
\maketitle
\begin{center}
{\large William Mitchell}\\
National Institute of Standards and Technology\\
{\tt william.mitchell@nist.gov}
\end{center}

\section*{Abstract}
In the $hp$-adaptive version of the finite element method for solving partial differential equations, the grid is adaptively refined in both $h$, the size of the elements, and $p$, the degree of the piecewise polynomial approximation over the element.  The selection of which elements to refine is determined by a local $a~posteriori$ error indicator, and is well established.  But the determination of whether the element should be refined by $h$ or $p$ or some combination is still open.  Several strategies have been proposed for making this determination [1].  These strategies have been implemented in the adaptive finite element code PHAML.  To determine the effectiveness of the strategies, a numerical experiment was performed using a collection of several 2D elliptic test problems [2] which exhibit various behaviors suitable for adaptive refinement, such as large gradients and singularities.  In this talk we present a summary of the results of this experiment.  The full results are available in [3].

\bibliographystyle{plain}
\begin{thebibliography}{10}
\bibitem{A Survey of hp-Adaptive Strategies for Elliptic Partial Differential Equations}
{\sc W.F. Mitchell and M.A. McClain}. {A Survey of $hp$-Adaptive Strategies for Elliptic Partial Differential Equations}. in Recent Advances in Computational and Applied Mathematics (T. E. Simos, ed.), Springer, 2011, pp. 227-258..

\bibitem{A Collection of 2D Elliptic Problems for Testing Adaptive Algorithms}
{\sc W.F. Mitchell}. {A Collection of 2D Elliptic Problems for Testing Adaptive Algorithms}. NISTIR 7668, 2010.

\bibitem{A Comparison of hp-Adaptive Strategies for Elliptic Partial Differential Equations (long version)}
{\sc W.F. Mitchell and M.A. McClain}. {A Comparison of $hp$-Adaptive Strategies for Elliptic Partial Differential Equations (long version)}. NISTIR 7824, 2011.
\end{thebibliography}

\title{Designing Efficiency Low Power LED Driver Using LTspice Software}
\tocauthor{Muhammad Ikram Mohd Rashid} \author{} \institute{}
\maketitle
\begin{center}
{\large Muhammad Ikram Mohd Rashid}\\
University of Malaysia Pahang\\
{\tt mikramump@gmail.com}
\end{center}

\section*{Abstract}
Many of today’s portable electronics require backlight LED-driver solutions with the following features: direct control of current, high efficiency, PWM dimming, overvoltage protection, load disconnect, small size, and ease of use. This article discusses each of these features and how they are achieved, and concludes with a typical circuit that implements each of these features. In this project cover designing circuit using LTspice and hardware implementation.

\bibliographystyle{plain}
\begin{thebibliography}{10}
\bibitem{Introduction to Power Electronics}
{\sc D.W. Hart}. {Introduction to Power Electronics}. Prentice Hall International 2008.
\end{thebibliography}

\title{Genetic Algorithm for Multi-objective Flowshop Scheduling Problem}
\tocauthor{Noraini Mohd Razali} \author{} \institute{}
\maketitle
\begin{center}
{\large Noraini Mohd Razali}\\
Universiti Malaysia Pahang\\
{\tt norainimbr@ump.edu.my}
\end{center}

\section*{Abstract}
This paper presents a genetic algorithm for finding a set of nondominated solutions of a multi-objective flowshop scheduling problem with the objective of minimizing makespan and total tardiness. The proposed algorithm is a combination of local search and improved genetic operators. A local search procedure is applied to the new solution by genetic operation in order to quickly improve the quality of the population. Genetic operators with different combination of crossover and mutation procedure improve the search ability and preserve the diversity. The nondominated solutions obtained by the proposed algorithm are compared with the existing algorithms on several problem sets with different scales. The computational experiments demonstrated that the presented genetic algorithm outperforms the existing algorithms in terms of solution quality, as well as solution diversity.

\bibliographystyle{plain}
\begin{thebibliography}{10}
\bibitem{Algorithms for flowshop scheduling problems}
{\sc T. Murata and H. Ishibuchi and H. Tanaka}. {Algorithms for flowshop scheduling problems}. Comp Ind Eng. 30 (1996) 1061-1701.

\bibitem{A new heuristic method for the flowshop sequencing problem}
{\sc M. Widmer and A. Hertz}. {A new heuristic method for the flowshop sequencing problem}. Eur J Oper Res. 41 (1989) 186-193.

\bibitem{Theory and methodology - an application of genetic algorithms for flowshop problems}
{\sc C. L. Chen and V. S. Vempati and N. Aljaber}. {Theory and methodology - an application of genetic algorithms for flowshop problems}. Eur J Oper Res. 80 (1995) 389-396.
\end{thebibliography}

\title{Equal-Order Approximation of Coupled Stokes-Darcy Problems}
\tocauthor{Kamel Nafa} \author{} \institute{}
\maketitle
\begin{center}
{\large Kamel Nafa}\\
Sultan Qaboos University\\
{\tt nkamel@squ.edu.om}
\end{center}

\section*{Abstract}
The basic Galerkin finite element method for solving steady state incompressible fluid fow problems may suffer because the velocity and pressure approximations do not satisfy the discrete Babuska-Brezzi condition. As a remedy to the instability of the Galerkin finite element formulation, symmetric stabilization techniques such as the continuous interior penalty, the subgrid and local projection methods were proposed and analysed in Burman and Hansbo(2006), Badia and  Codina(2009), Becker and Braack(2001), and Nafa and Wathen(2009). In this work we consider a coupled Stokes-Darcy problem where in one part of the domain the fluid motion is described by Stokes equations and for the other part the fluid is in a porous medium and described by Darcy law and the conservation of mass. Velocity and pressure on these two parts are mutually coupled by interface conditions derived by Saffman(1971). Such systems can be discretized by heterogeneous finite elements in the two parts, such as Taylor-Hood or MINI elements for the Stokes part, and mixed elements of Raviart-Thomas or Brezzi-Douglas-Marini elements type for the Darcy region. Such approach is analysed by Layton et al.(2003). In more recent work, unified approaches become more popular. The mixed finite element method of Karper et al.(2009) and the discontinuous Galerkin method introduced by Girault and Riviere(2009) are examples of this type of approach. Here, we discretize by standard equal-order finite elements and use local projection stabilization technique to stabilize the scheme. However, to obtain
optimal error estimates we have to choose two different stabilization parameters for the two different regions.

\bibliographystyle{plain}
\begin{thebibliography}{10}
\bibitem{Local projection stabilized Galerkin approximations for the generalized Stokes problem}
{\sc K. Nafa and A. J. Wathen}. {Local projection stabilized Galerkin approximations for the generalized Stokes problem}. Comput. Methods. Appl. Mech. Engrg. 198, Issues 5-8 (2009) 877-883.

\bibitem{Coupling fluid flow with porous media flow}
{\sc W J. Layton and F. Schieweck and I. Yotov}. {Coupling fluid flow with porous media flow}. SIAM Journal on Numerical Analysis 2003; 40(6):2195-2218.

\bibitem{Unified finite element discretizations of coupled Darcy-Stokes flow  }
{\sc T. Karper and K A. Mardal and R. Winther}. {Unified finite element discretizations of coupled Darcy-Stokes flow  }. Numerical Methods for Partial Differential Equations 2009; 25(2):311-326.

\bibitem{DG approximation of coupled Navier-Stokes and Darcy equations by Beaver-Joseph-Saffman interface conditions}
{\sc V. Girault and B. Riviere}. {DG approximation of coupled Navier-Stokes and Darcy equations by Beaver-Joseph-Saffman interface conditions}. SIAM J. Numer. Anal., 47 (2009), no. 3, pp. 2052-2089.
\end{thebibliography}

\title{Advances of Leray Regularization for Fluid Flow Problems}
\tocauthor{Abigail BowersMonika Neda} \author{} \institute{}
\maketitle
\begin{center}
{\large Abigail Bowers}\\
Clemson University, USA\\
{\tt abowers@clemson.edu}
\\ \vspace{4mm}
{\large Tae-Yeon  Kim}\\
McGill University, Canada\\
{\tt tykimsay@gmail.com}
\\ \vspace{4mm}
{\large \underline{Monika Neda}}\\
University of Nevada Las Vegas\\
{\tt monika.neda@unlv.edu}
\\ \vspace{4mm}
{\large Leo Rebholz}\\
Clemson University, USA\\
{\tt rebholz@clemson.edu}
\\ \vspace{4mm}
{\large Eliot  Fried}\\
McGill University, Canada\\
{\tt eliot.fried@mcgill.ca}
\end{center}

\section*{Abstract}
Accurate direct numerical simulation (DNS) of the Navier–Stokes equations (NSE) at high Reynolds numbers is well known to be a formidable computational challenge. To overcome this, improved modeling techniques have been proposed. We study a fluid flow regularization based on the Leray-alpha model that uses deconvolution in the nonlinear term and dissipation scale modeling in the viscous term. In particular, we establish that this new regularization model has an energy cascade with an enhanced energy dissipation. We also propose and analyze an efficient finite element algorithm method for the proposed model. In addition to establishing stability of the method, we demonstrate the convergence theory too. A numerical experiment for the two-dimensional flow around an obstacle is also discussed. Results show that enhancing the Leray-alpha model with deconvolution and dissipation scale modeling can significantly increase accuracy.

\bibliographystyle{plain}
\begin{thebibliography}{10}
\bibitem{The Leray-alpha-beta-deconvolution model energy analysis and numerical algorithms}
{\sc A. Bowers and T.-Y. Kim and M. Neda and L.G. Rebholz and E. Fried}. {The Leray-alpha-beta-deconvolution model: energy analysis and numerical algorithms}. Applied Mathematical Modelling, 37(2013), pp. 1225-1241.

\bibitem{A continuum mechanical framework for turbulence giving a generalized NavierStokes-alpha equation with complete boundary conditions}
{\sc E. Fried and M.E. Gurtin}. {A continuum mechanical framework for turbulence giving a generalized Navier–Stokes-alpha equation with complete boundary conditions}. Theor. Comput. Fluid Dynam., 22 (2008), pp. 433–470.
\end{thebibliography}

\title{On Implementing  the IEEE Interval Standard P1788}
\tocauthor{Marco Nehmeier} \author{} \institute{}
\maketitle
\begin{center}
{\large \underline{Marco Nehmeier}}\\
University of W\"urzburg\\
{\tt nehmeier@informatik.uni-wuerzburg.de}
\\ \vspace{4mm}
{\large J\"urgen Wolff von Gudenberg}\\
University of W\"urzburg\\
{\tt wolff@informatik.uni-wuerzburg.de}
\\ \vspace{4mm}
{\large John Pryce}\\
Cardiff University\\
{\tt prycejd1@cardiff.ac.uk}
\end{center}

\section*{Abstract}
Computers should be enabled 
to check their results for correctness. 
This is mandatory for applications which require a certain kind of accuracy of 
the computations like chemical engineering 
and control theory  as well as computer graphics and computer-aided 
design. 
 
Interval arithmetic is a tool that can help in this situation. Instead of 
calculating with floating point numbers, i.e. approximations of real numbers, 
interval arithmetic computes rigorous bounds for ``real'' real numbers. 

Note, however, that this guarantee is generally not obtained by only changing 
the data type from floating point to interval because dependencies between 
intervals lead to overestimation. Hence, new algorithms have to be defined and 
the underlying arithmetic has to compute rigorous bounds. 
This has been done in the last 50 years and we 
now have on the algorithmic side  
the interval Newton method, 
branch and bound algorithms for global optimization, 
self validating algorithms for linear and nonlinear equations, 
rigorous ODE solvers, 
and many more. 
 
To enlarge the acceptance of interval arithmetic the IEEE interval 
standard working group P1788 has been founded in 2008. 
The presumable 
standard  defines intervals as connected, closed, not necessarily 
 bounded subsets of the reals. 
 The basic arithmetic operations are defined as powerset 
 operations. 
 The interval operations compute the interval hull of these sets. 
Since a computer representation of an interval uses floating point numbers for 
the bounds, directed rounding toward $-\infty$ or $+\infty$ is necessary to 
compute a true enclosure. 

In this talk we will present the key concepts of the tentative standard as 
well as a faithful C++ implementation  which can serve as a reference model for 
other implementations. 

\bibliographystyle{plain}
\begin{thebibliography}{10}
\bibitem{IEEE Interval Standard Working Group - P1788}
{\sc IEEE P1788}. {IEEE Interval Standard Working Group - P1788}. http://grouper.ieee.org/groups/1788/.

\bibitem{IEEE Interval Standard Working Group - P1788 Current Status}
{\sc W. Edmonson and G. Melquiond}. {IEEE Interval Standard Working Group - P1788: Current Status}. In Proceedings of the 2009 19th IEEE Symposium on Computer Arithmetic, ARITH 09, pages 231-234, Washington, DC, USA, 2009. IEEE Computer Society .

\bibitem{Interval Computations Introduction Uses and Resources }
{\sc R. B. Kearfott}. {Interval Computations: Introduction, Uses, and Resources }. Euromath Bulletin, 2:95-112, 1996 .

\bibitem{FILIB++ a fast interval library supporting containment computations}
{\sc M. Lerch and G. Tischler and J. W. von Gudenberg and W. Hofschuster and W. Kr\"amer }. {FILIB++, a fast interval library supporting containment computations}. ACM Trans. Math. Softw., 32(2):299-324, 2006.
\end{thebibliography}

\title{Adaptive Time-Integration for Discontinuous Galerkin Time-Domain Simulations of Maxwell's Equations}
\tocauthor{Jens Niegemann} \author{} \institute{}
\maketitle
\begin{center}
{\large \underline{Jens Niegemann}}\\
Lab for Electromagnetic Fields and Microwave Electronics (IFH), ETH Zurich, Switzerland\\
{\tt jens.niegemann@ifh.ee.ethz.ch}
\\ \vspace{4mm}
{\large Ueli Koch}\\
Lab for Electromagnetic Fields and Microwave Electronics (IFH), ETH Zurich, Switzerland\\
{\tt kochue@student.ethz.ch}
\end{center}

\section*{Abstract}
The discontinuous Galerkin (DG) approach has gained considerable attention as an efficient and accurate method for solving Maxwell's equations in time-domain. Its ability to allow explicit time integration while offering a higher-order spatial discretization on unstructured meshes makes it a very attractive method for complex electromagnetic systems [1]. In order to match the accurate spatial discretization one typically also requires an efficient higher-order time integration method. In practice, explicit low-storage Runge-Kutta (LSRK) schemes were shown to offer an excellent compromise of accuracy, performance and memory consumption.

Here, we will present several new low-storage Runge-Kutta methods which significantly improve both the efficiency and the accuracy of DG time-domain simulations of Maxwell's equations. First, we present novel LSRK schemes in the 2N formulation with up to 22 stages. Besides optimized schemes of 4th order [2], we also discuss new methods of 5th order. In addition, we also demonstrate embedded Runge-Kutta pairs in the low-storage 3S* formulation [3] with optimized stability contours for both the main and the embedded schemes. Using the embedded scheme for error estimation then allows us to automatically adapt the timestep in large scale DG calculations.

\bibliographystyle{plain}
\begin{thebibliography}{10}
\bibitem{Discontinuous Galerkin methods in nanophotonics}
{\sc  K. Busch and M. K\"onig and J. Niegemann}. {Discontinuous Galerkin methods in nanophotonics}. Laser \& Photon. Rev. 5 (2011) 773-809.

\bibitem{Efficient low-storage Runge-Kutta schemes with optimized stability regions}
{\sc J. Niegemann and R. Diehl and K. Busch}. {Efficient low-storage Runge-Kutta schemes with optimized stability regions}. J. Comput. Phys. 231 (2012) 364-372 .

\bibitem{Runge-Kutta methods with minimum storage implementations}
{\sc D. Ketcheson}. {Runge-Kutta methods with minimum storage implementations}. J. Comput. Phys. 229 (2010) 1763-1773.
\end{thebibliography}

\title{Numerical Analysis of the Flowfield Around a Vertical Axis Wind Turbine}
\tocauthor{Michele NiniValentina Motta} \author{} \institute{}
\maketitle
\begin{center}
{\large Michele Nini}\\
Politecnico di Milano\\
{\tt michele.nini@mail.polimi.it}
\\ \vspace{4mm}
{\large \underline{Valentina Motta}}\\
Politecnico di Milano\\
{\tt motta@aero.polimi.it}
\\ \vspace{4mm}
{\large Alberto Guardone}\\
Politecnico di Milano\\
{\tt guardone@aero.polimi.it}
\end{center}

\section*{Abstract}
Due to  the tridimensional nature of the flow, the possible presence of dynamic stall of the blades and the tip vortices, the aerodynamic field around a Vertical Axis Wind Turbine (VAWT) is very complicated. The interaction of these phenomena has not been extensively studied yet and it is still far from being fully understood.

The purpose of this work is a CFD analysis of the aerodynamic field of a three straight blades VAWT using the numerical solver ROSITA (ROtorcraft Software ITAly). ROSITA is a finite-volume code, which can solve RANS equations in overset systems of moving multi-block grids, coupled with the Chimera technique.
In order to reduce the computational effort, the flow symmetry with respect to the horizontal mid-span plane is exploited.

It is worthy to notice that the present is one of the first 3D CFD studies for a complete VAWT, therefore it should be considered as a preliminary study and some aspects have to be investigated more accurately.
Nevertheless, numerical results, compared to the experimental data, confirm some general aspects of the experimental measurements, such the asymmetry in the normal plane. Moreover, interaction from the blades and their tips are captured, as well as the separation regions where blade dynamic stall occurs.


\bibliographystyle{plain}
\begin{thebibliography}{10}
\bibitem{RANS computations of rotor/fuselage unsteady interaction- al aerodynamics}
{\sc M. Biava}. {RANS computations of rotor/fuselage unsteady interaction- al aerodynamics}. Doctor of Philosophy Thesis, Politecnico di Milano, Milano, Italia, 2007.

\bibitem{Wind Turbine Design - With Emphasis on Darrieus Concept}
{\sc I. Paraschivoiu }. {Wind Turbine Design - With Emphasis on Darrieus Concept}. Presses International Polytecnique (2009, 1st edittion 2002) .

\bibitem{Aerodynamic Measurements on a Vertical Axis Wind Turbine in a Large Scale Wind Tunnel}
{\sc L. Battisti and L. Zanne and S. Dell'Anna and V. Dossena and G. Persico and B. Paradiso}. {Aerodynamic Measurements on a Vertical Axis Wind Turbine in a Large Scale Wind Tunnel}. Journal of Energy Resources Technology.
\end{thebibliography}

\title{On Integration of Synthesized Microstructural Enrichment Functions in Partition of Unity and Trefftz Method}
\tocauthor{Jan  Nov\'{a}k} \author{} \institute{}
\maketitle
\begin{center}
{\large Jan  Nov\'{a}k}\\
Brno University of Technology, Faculty of Civil Engineering, Institute of Structural Mechanics, Veve\v{r}\'{i} 331/95, Brno, Czech Republic\\CTU in Prague, Faculty of Civil Engineering, Department of Mechnics, Th\'akurova 7, Prague, Czech Republic \\
{\tt novakj@cml.fsv.vut.cz}
\end{center}

\section*{Abstract}
The main objective of this contribution is drawn towards the integration of complex microstructural functions, with physical meaning of e.g. thermomechanical fields, that may be used in Partition of Unity and hybrid finite element environments as the enrichment functions. In modelling of heterogeneous materials, both finite element formulations exchange approximation properties of standard element spaces by involving a class of functions mirroring the subscale phenomena, thereby alleviating the need for detailed meshing. The intrinsic feature of such a process is, however, a translation of the mesh-related difficulties to a problem of efficient calculation and integration of complex, often singular and discontinuous, functions over the element domains~\cite{Novak:CMAME:2012}. Recently, a technique of synthesis of microstructural enrichment functions has been reported~\cite{Novak:PRE:2012,Novak:MSMSE:2012}. It uses small sets of characteristic square samples, called Wang tiles~\cite{Wang:BSTJ:1961}, to synthesize micromechanical fields in large macro-scopic domains rather then their direct evaluation. The method is efficient as for the accuracy and computational overhead, however, the integration of resulting fields remains an open issue. The goal of this contribution is thus to investigate several methods for their integration with respect to a distribution of Wang tiles underneath the macroscopic mesh, a grid of points the micro fields are defined, and the dimension and shape of the elements. In particular, direct numerical integration based on interpolating quadratures, Monte Carlo integration along with the transformation of domain integrals to their boundary analogs will be explored. In addition, a potential of fast summation techniques will be also discussed.

\bibliographystyle{plain}
\begin{thebibliography}{10}
\bibitem{A micromechanics-enhanced finite element formulation for modelling heterogeneous materials}
{\sc J. Nov\'{a}k and \L. Kaczmarczyk and P. Grassl and J. Zeman and C. J. Pearce}. {A micromechanics-enhanced finite element formulation for modelling heterogeneous materials}. Computer Methods in Applied Mechanics and Engineering, (2012) 201-204:53-64.

\bibitem{Compressing random microstructures via stochastic Wang tilings}
{\sc J. Nov\'{a}k and A. Ku\v{c}erov\'{a} and J. Zeman}. {Compressing random microstructures via stochastic Wang tilings}. Physical Review E, (2012 86:040104.

\bibitem{Microstructural enrichment functions based on stochastic Wang tilings}
{\sc J. Nov\'{a}k and A. Ku\v{c}erov\'{a} and J. Zeman}. {Microstructural enrichment functions based on stochastic Wang tilings}. Available at: http://arxiv.org/abs/1110.4183.

\bibitem{Proving theorems by pattern recognition-II}
{\sc H. Wang}. {Proving theorems by pattern recognition-II}. Bell Systems Technical Journal, (1961) 40(2):1-41.
\end{thebibliography}

\title{Dispersive and Dissipative Errors in the DPG Method With Scaled Norms for Helmholtz Equation}
\tocauthor{Jay GopalakrishnanNicole Olivares} \author{} \institute{}
\maketitle
\begin{center}
{\large Jay Gopalakrishnan}\\
Department of Mathematics and Statistics, Portland State University\\
{\tt gjay@pdx.edu}
\\ \vspace{4mm}
{\large Ignacio Muga}\\
Instituto de Matemáticas, Pontificia Universidad Católica de Valparaíso\\
{\tt ignacio.muga@ucv.cl}
\\ \vspace{4mm}
{\large \underline{Nicole Olivares}}\\
Department of Mathematics and Statistics, Portland State University\\
{\tt nmo@pdx.edu}
\end{center}

\section*{Abstract}
I will present a study of dispersive and dissipative errors in the lowest order Discontinuous Petrov-Galerkin (DPG) [3, 4] solution to the Helmholtz equation, using a modified graph norm for the test space. The modification scales one of the terms of the graph norm by an arbitrary positive parameter. Through mathematical analysis we show that as the parameter approaches zero, the error in an ideal DPG method must improve. However a typical practical implementable DPG method differs from the ideal DPG method. Whether the actually observed numerical error of the practical DPG method improves is investigated through a dispersion analysis [1, 5]. Since the DPG method has multiple interacting stencils, to find the discrete wavespeed, we must solve a nonlinear system for every given wavespeed. The dispersion analysis indicates that the performance does improve in certain cases. It also shows that the discrete wavenumbers of the method are complex, which explains the numerically observed artificial dissipation in the computed wave approximations. The performance of the DPG method is compared with a standard least-squares method [2] and finite element method.

\bibliographystyle{plain}
\begin{thebibliography}{10}
\bibitem{Discrete Dispersion Relation for hp-Version Finite Element Approximation at High Wave Number}
{\sc M. Ainsworth}. {Discrete Dispersion Relation for {$hp$}-Version Finite Element Approximation at High Wave Number}. SIAM J. Numer. Anal. 42-2 (2004) 553-575.

\bibitem{First-Order System Least Squares for Second-Order Partial Differential Equations Part I}
{\sc Z. Cai and R. Lazarov and T. A. Manteuffel and S. F. McCormick}. {First-Order System Least Squares for Second-Order Partial Differential Equations: Part I}. SIAM J. Numer. Anal. 31-6 (1994) 1785-1799.

\bibitem{A Class of Discontinuous Petrov-Galerkin Methods. Part II Optimal Test Functions}
{\sc L. Demkowicz and J. Gopalakrishnan}. {A Class of Discontinuous Petrov-Galerkin Methods. Part II: Optimal Test Functions}. Numer. Methods Partial Differential Eq. 27 (2011) 70-105.

\bibitem{Wavenumber Explicit Analysis of a DPG Method for the Multidimensional Helmholtz Equation}
{\sc L. Demkowicz and J. Gopalakrishnan and I. Muga and J. Zitelli}. {Wavenumber Explicit Analysis of a DPG Method for the Multidimensional Helmholtz Equation}. Comput. Methods Appl. Mech. Engrg. 213-216 (2012) 126–138.

\bibitem{Dispersion and Pollution of the FEM Solution for the Helmholtz Equation in One Two and Three Dimensions}
{\sc A. Deraemaeker and I. Babu\v{s}ka and P. Bouillard}. {Dispersion and Pollution of the FEM Solution for the Helmholtz Equation in One, Two and Three Dimensions}. Int. J. Numer. Meth. Engng. 46 (1999) 471-499.
\end{thebibliography}

\title{A Posteriori Estimation of Hierarchical Type for the  Schrodinger Operator With Inverse Square Potential on Graded Meshes}
\tocauthor{Jeffrey Ovall} \author{} \institute{}
\maketitle
\begin{center}
{\large Jeffrey Ovall}\\
University of Kentucky\\
{\tt jovall@ms.uky.edu}
\\ \vspace{4mm}
{\large Hengguang Li}\\
Wayne State University\\
{\tt hli@math.wayne.edu}
\end{center}

\section*{Abstract}

We develop an a posteriori error estimate for mixed boundary value problems of the form $(-\Delta + V )u = f$ , where the potential $V$ may possess inverse-square singularities at finitely many points in the domain. Such potentials arise in quantum mechanics, and require
a more specialized analysis in weighted Sobolev spaces---not the standard $H^1$-space for diffusion problems---for well-posedness and regularity results, as well as for the 
development of effective numerical algorithms.  The solutions of such problems can have
$r^{\alpha}$-type singularities for any $\alpha>0$ at the points where $V$ is unbounded,
in addition to the usual boundary singularities at re-entrant corners in the domain and
point where the type of boundary condition changes.  Therefore, some type of adaptive approximation is needed.  

In two-dimensions, the types of singularities which can occur are sufficiently well-understood so that a systematic grading strategy may be chosen \textit{a priori} which guarantees optimal order convergence.  Even in this case, however, a cheaply-computable error estimate is desirable, if for no other reason than to establish a practical stopping criterion.  We present a hierarchical-type \textit{a posteriori} error estimate, which we prove can be efficiently computed and is equivalent to the actual error in the energy norm on a family of geometrically graded meshes appropriate for singular solutions of such problems.  Experimentally, we see that the estimate is (nearly) asymptotically exact in both the energy norm and the $H^1$-seminorm, but it is not (yet) proven that this behavior should be expected in general.  As a matter of interest, we provide direct numerical comparisons between the convergence behavior of the geometrically graded mesh approach upon which our theory is based and adaptive refinement which is driven our \textit{a posteriori} estimate.  Effectivity comparisons between the two approaches are also given.




\bibliographystyle{plain}
\begin{thebibliography}{10}
\bibitem{A posteriori estimation of hierarchical type for the  Schr"odinger operator with inverse square potential on graded meshes}
{\sc H. Li and J.S. Ovall}. {A posteriori estimation of hierarchical type for the  Schr\"odinger operator with inverse square potential on graded meshes}. in review.

\bibitem{Inverse-square potential and the quantum vortex}
{\sc H. Wu and D. Sprung}. {Inverse-square potential and the quantum vortex}. Physical Review A,  49:4305–4311, 1994.

\bibitem{Analysis of a modified Schr"odinger operator in 2D regularity index  and FEM}
{\sc H. Li and V. Nistor}. {Analysis of a modified Schr\"odinger operator in 2D: regularity, index,  and FEM}. J. Comput. Appl. Math., 224(1):320–338, 2009.
\end{thebibliography}

\title{Algebraic Extraction of Topology and Geometry From 2D/3D Images}
\tocauthor{Vittorio CecchettoAlberto Paoluzzi} \author{} \institute{}
\maketitle
\begin{center}
{\large Vittorio Cecchetto}\\
Roma Tre University\\
{\tt v.cecchetto@gmail.com}
\\ \vspace{4mm}
{\large Antonio DiCarlo}\\
Roma Tre University\\
{\tt adicarlo@mac.com}
\\ \vspace{4mm}
{\large Francesco Furiani}\\
Roma Tre University\\
{\tt fra.furiani@stud.uniroma3.it}
\\ \vspace{4mm}
{\large Enrico Marino}\\
Roma Tre University\\
{\tt marino@dia.uniroma3.it}
\\ \vspace{4mm}
{\large \underline{Alberto Paoluzzi}}\\
Roma Tre University\\
{\tt apaoluzzi@me.com}
\\ \vspace{4mm}
{\large Federico Spini}\\
Roma Tre University\\
{\tt spini@dia.uniroma3.it}
\end{center}

\section*{Abstract}
New problems in science and technology, requiring modeling/simulation of big geometric models, demand most methods underlying geometric and physical modeling to be rethought from scratch. In the past years, the prevailing trend was towards ad hoc solutions to different classes of modeling problems and data structures supporting them. Contrariwise, we contend that|on account of the opportunities offered and the constraints imposed by novel web-based platforms|one should now head in the opposite direction. Emerging technologies push us into researching simple, general-purpose and dimension-independent geometric data structures and computational methods. Motivated by this belief, we engaged in developing an innovative computer representation of topological and geometric data, in the framework of the IEEE-SA P3333.2 WG ``Three-Dimensional Model Creation Using Unprocessed 3D Medical Data''. Our LAR (Linear Algebraic Representation) scheme is a simple algebraic rep for cell complexes that uses a CSR (Compressed Sparse Row) form for coding the characteristic matrices of linear spaces of (co)chains. LAR enjoys a neat mathematical format, being based on chains (the domains of discrete integration) and cochains (the discrete prototype of differential forms). The input of a LAR representation is a cell decomposition of the space, where each cell is described by the (unordered) set of indices of its vertices. In this paper, we show how to extract lower-dimensional skeletons of a cell complex, and hence to compute boundary and coboundary operators on the (co)chain spaces supported by the complex, using simple and fast algebraic methods (mostly multiplication of sparse binary matrices). This technique may be directly applied to block-decomposed 2D/3D images. The resulting complex provides a complete topological and geometrical representation of the discrete domain sampled by the image. From a technological viewpoint, LAR may attain good parallel performance via OpenCL and WebCL|Khronos's APIs (industry standard) for parallel heterogeneous computing|, since it does not require traversing or searching linked data structures. We are confident that it may help dealing with state-of-the-art biomedical applications which require high performances with big geometric data.

\bibliographystyle{plain}
\begin{thebibliography}{10}
\bibitem{Linear algebraic representation of big geometric data}
{\sc A. Paoluzzi and A. DiCarlo and and V. Shapiro.}. {Linear algebraic representation of big geometric data}. Submitted paper (http://paoluzzi.dia.uniroma3.it/web/pao/doc/paoAdcVs-2013.pdf).
\end{thebibliography}

\title{Bayesian Estimation of Turbulence Model Parameters Using High-Order Sensitivity Analysis}
\tocauthor{Dimitrios Papadimitriou} \author{} \institute{}
\maketitle
\begin{center}
{\large Dimitrios Papadimitriou}\\
Department of Mechanical Engineering, University of Thessaly, Volos, Greece\\
{\tt dpapadim@uth.gr}
\\ \vspace{4mm}
{\large Costas Papadimitriou}\\
Department of Mechanical Engineering, University of Thessaly, Volos, Greece\\
{\tt costasp@uth.gr}
\end{center}

\section*{Abstract}
A Bayesian inference framework is used for estimating the uncertainties in turbulence model parameters employed in computational fluid dynamics simulations. The uncertainties in the model parameters are quantified by the posterior distribution of the model parameters obtained using the Bayesian formulation. Based on Laplace asymptotic methods [1], the posterior distribution is approximated by a Gaussian distribution centered at the most probable value of the model parameters obtained by minimizing an objective function defined as minus the logarithm of the posterior distribution. The covariance matrix of the Gaussian posterior distribution is the inverse of the Hessian of the objective function evaluated at the most probable value. The adjoint approach [2] is extended to compute the first-order sensitivities of the objective function with respect to the turbulence model parameters. Based on these sensitivities, a descent approach is used to find the most probable solution. The required Hessian matrix formulated by the first and second-order sensitivities of the objective function with respect to the turbulence model parameters is computed using the most efficient combination of the direct differentiation of the flow equations and the adjoint equations. The first and second-order sensitivities are also used to evaluate by asymptotic techniques the multi-dimensional probability integrals that arise in the propagation of uncertainties in various output quantities of interest. The proposed approach is demonstrated for the case of the flow through a backward facing step using the Spalart-Allmaras turbulence model. The experimental measurements that are used for quantifying the uncertainties in the model parameters consist of the normal to the wall velocity profiles at five longitudinal positions, the pressure coefficients at the bottom and top walls after the step and the friction coefficient distribution at the bottom wall after the step, where the flow is separated. Results demonstrate that two among the eight parameters of the model are identified with sufficient accuracy, while the other six parameters involve significant uncertainty that also affects uncertainties in the predictions of various output flow quantities of interest.

\bibliographystyle{plain}
\begin{thebibliography}{10}
\bibitem{Updating robust reliability using structural test data}
{\sc C. Papadimitriou and J.L. Beck and L.S. Katafygiotis}. {Updating robust reliability using structural test data}. Probabilistic Engineering Mechanics, 16(2):103-113, 2001.

\bibitem{Aerodynamic shape optimization using first and second order adjoint and direct approaches}
{\sc D.I. Papadimitriou and K.C. Giannakoglou}. {Aerodynamic shape optimization using first and second order adjoint and direct approaches}. Archives of Computational Methods in Engineering, 15(4):447-488, 2008.
\end{thebibliography}

\title{Numerical Solution of Compressible and Incompressible Unsteady Flows in Channel}
\tocauthor{Petra Po\v{r}\'izkov\'a} \author{} \institute{}
\maketitle
\begin{center}
{\large Petra Po\v{r}\'izkov\'a}\\
Czech Technical University in Prague, Karlovo n\'am\v{e}st\'i 13, 121 35, Prague 2, Czech Republic\\
{\tt puncocha@marian.fsik.cvut.cz}
\\ \vspace{4mm}
{\large Karel Kozel}\\
Institute of Thermomechanics Academy of Sciences, Dolej\v{s}kova 5, Prague 8, Czech Republic\\
{\tt kozelk@fsik.cvut.cz}
\\ \vspace{4mm}
{\large Jarom\'ir Hor\'a\v{c}ek}\\
Institute of Thermomechanics Academy of Sciences, Dolej\v{s}kova 5, Prague 8, Czech Republic\\
{\tt jaromirh@it.cas.cz}
\end{center}

\section*{Abstract}
This study deals with the numerical solution of a 2D unsteady flow of a viscous fluid in a channel for low inlet airflow velocity. The unsteadiness of the flow is caused by a prescribed periodic motion of a part of the channel wall with large amplitudes, nearly closing the channel during oscillations. The channel is a simplified model of the glottal space in the human vocal tract \cite{Principles of Voice Production}. Goal is numerical simulation of flow in the channel which involves attributes of real flow causing acoustic perturbations. Particular attention is paid to the acoustic analysis of pressure signal from the channel. Four governing systems are considered to describe the unsteady laminar flow of a viscous fluid in the channel: 1. {\it Full system} - 2D system of Navier-Stokes (NS) equations closed with static pressure expression for ideal gas $p=f(\rho, u, v, e)$ describes flow of compressible viscous fluid, 5 equations. 2. {\it Iso-energetic system} - 2D NS equations closed with pressure expression which is independent on total energy variable $e$ describes flow of compressible viscous fluid, 4 eqs. 3. {\it Adiabatic system} - 2D NS equations closed with pressure expression which is independent on variables $e,u,v$ describes flow of compressible viscous fluid, 4 eqs. 4. {\it Incompressible system} - 2D NS equations where density $\rho=const$ describes steady state flow of incompressible viscous fluid, 3 eqs. Solution is computed using Artificial Compressibility Method \cite{A numerical method for solving incompressible viscous flow problems}. The numerical solution is implemented using the finite volume method (FVM) and the predictor-corrector MacCormack scheme with artificial viscosity using a grid of quadrilateral cells. The unsteady grid of quadrilateral cells is considered in the form of conservation laws using Arbitrary Lagrangian-Eulerian method. The numerical simulations of flow fields in the channel, acquired from a developed program, are presented for inlet velocity $\hat u_{\infty}=4.12 {\rm ms^{-1}}$ and Reynolds number Re$_{\infty} = 4481$ and the wall motion frequency 100 Hz.

\bibliographystyle{plain}
\begin{thebibliography}{10}
\bibitem{Principles of Voice Production}
{\sc I.R. Titze}. {Principles of Voice Production}. National Centre for Voice and Speech, Iowa City, 2000. ISBN 0-87414-122-2.

\bibitem{A numerical method for solving incompressible viscous flow problems}
{\sc A.J. Chorin}. {A numerical method for solving incompressible viscous flow problems}. J. Comput. Phys., 2, 12–26, (1967).
\end{thebibliography}

\title{Visual Cryptography Based on Chaotic Oscillations}
\tocauthor{Minvydas Ragulskis} \author{} \institute{}
\maketitle
\begin{center}
{\large Minvydas Ragulskis}\\
Kaunas University of Technology\\
{\tt minvydas.ragulskis@ktu.lt}
\end{center}

\section*{Abstract}
Dynamic visual cryptography scheme based on chaotic oscillations is proposed in this paper. Visual cryptography is a cryptographic technique which allows visual information to be encrypted in such a way that decryption is performed by the superposition of shares and does not require a computer [1]. Dynamic visual cryptography [2, 3] exploits a single share, but the decryption is based on periodic oscillations of the cover image instead. These schemes require special computational algorithms for hiding the secret image in the cover moiré grating, but the decryption of the secret is completely visual; the secret image is leaked in the form of time-averaged geometric moiré fringes. The proposed scheme is based on the chaotic visual decryption when the cover image is oscillated by a chaotic law. The relationship among the standard deviation of the stochastic time variable, the pitch of the moiré grating and the pixel size ensuring visual decryption of the secret is derived. The parameters of these chaotic oscillations must be carefully preselected before the secret image is leaked from the cover image. Several computational experiments are used to illustrate the functionality and the applicability of the proposed image hiding
technique.

\bibliographystyle{plain}
\begin{thebibliography}{10}
\bibitem{Visual cryptography}
{\sc M. Naor and A. Shamir.}. {Visual cryptography}. Lecture Notes in Computer Science 950 (1995) 1..

\bibitem{Image hiding based on time-averaged fringes produced by non-harmonic oscillations }
{\sc M. Ragulskis and A. Aleksa and Z. Navickas. }. {Image hiding based on time-averaged fringes produced by non-harmonic oscillations }. Journal of Optics A: Pure and Applied Optics. 2009, vol. 11, no. 12, Art.No. 125411. .

\bibitem{Image hiding based on time-averaging moire }
{\sc M. Ragulskis and A. Aleksa. }. {Image hiding based on time-averaging moire }. Optics Communications. 2009, vol. 282, p.2752-2759..
\end{thebibliography}

\title{Regularity Criterion for 3D MHD Equations Passing Through the Porous Medium in Terms of Gradient Pressure}
\tocauthor{Saeed  Rahman} \author{} \institute{}
\maketitle
\begin{center}
{\large Saeed  Rahman}\\
Northwestern Polytecnical University\\
{\tt saeed@ciit.net.pk}
\end{center}

\section*{Abstract}
This research contribution has been drafted to focus on regularity criteria for the weak solution of fluid passing through the porous media in $R^{3}$.  Results have proved that if $ \nabla P\in L^{\alpha,\gamma}$ with $\frac {2}{\alpha}+\frac {3}{\gamma}\leq3,$ $1\leq\gamma\leq\infty$, then the weak solution is regular and unique.

Let us consider the 3D flow of an incompressible fluid passing through the porous medium. Let $u_{1},$ $u_{2}$ and $u_{3}$ are the components of velocity
$u.$ It is well know that when the fluid passes through the porous medium, there exists a pressure drop. This pressure drop is given by using the Darcy's law
\begin{equation*}
R=-\frac {\phi} {K} u,
\end{equation*}
where $\phi$ is the porosity of the medium and $K$ is the permeability of the medium. The fundamental equations which governs the 3D equations under the assumption of an incompressible, unsteady MHD fluid passing through the porous medium are
\begin{equation}
\frac{\partial u}{\partial t}+u\cdot\nabla u
=\nu_{1}\triangle u-\triangledown P+
b\cdot\nabla b-mu,
\end{equation}
\begin{equation}
\frac{\partial b}{\partial t}+u\cdot\nabla b
=\nu_{2}\triangle b + b\cdot\nabla u,
\end{equation}
\begin{equation}
\nabla\cdot u= \nabla\cdot b=0,
\end{equation}
\begin{equation}\label {6}
u(x,0)=u_{0}(x),
\quad b(x,0)=b_{0}(x),
\end{equation}
where $u$ is the velocity, $b$ is the magnetic field, $\nu_{1}$ is the kinematic viscosity, $\nu_{2}$ is the magnetic diffusivity, and $P$ is the pressure of the medium. For simplicity we let $m=\frac {\phi}{K}$, also we take $\nu_{1}=\nu_{2}=1.$


Our main results are

\textbf{Theorem 1.} Suppose  that $u_{0},b_{0}\in H^{s}(R^{3}),$ $s\geq 3$ and $\nabla \cdot u_{0}=0=\nabla \cdot b_{0}$, in the sense of distribution. Assume that a pair of $(u,b)$ is the Leray-Hopf weak solution satisfying $(1.1)-(1.4)$ on $[0,T].$ If $ \nabla P\in L^{\alpha, \gamma}$ and $b\in L^{3\alpha, 3\gamma}$ with $\frac {2} {\alpha}+\frac {3}{\gamma}\leq 3,$
$1\leq \gamma\leq\infty,$ or $\left\|\nabla P\right\|_{L^ {1}},$ $\left\|\nabla P\right\|_{L^ {\frac{2}{3},\infty}},$ $\left\|b\right\|_{L^ {3}},$ and $\left\|b\right\|_{L^ {2,\infty}}$ are sufficiently small, then solution remains smooth on $[0,T].$


\bibliographystyle{plain}
\begin{thebibliography}{10}
\bibitem{ Regularity criteria for the 3D MHD equations in terms of pressure}
{\sc "Y. Zhou"}. { Regularity criteria for the 3D MHD equations in terms of pressure}. Non-Linear Mech. 41(10) (2006), 1174-1180.

\bibitem{Global On the Regularity of Weak Solution}
{\sc C. He and Z. Xin}. {Global On the Regularity of Weak Solution}. J. Diff. Eqs. 2 (2005), 225-254..

\bibitem{On regularity Criteria in terms of pressure for the 3D Viscous MHD equations}
{\sc H. Duan}. {On regularity Criteria in terms of pressure for the 3D Viscous MHD equations}. Appl. Anal. 91 (5) (2012), 947-952.
\end{thebibliography}

\title{Evaluation of the Vapor-liquid Equilibrium of Multi-parameter Thermodynamics Models Using Differential Algebra}
\tocauthor{Barbara Re} \author{} \institute{}
\maketitle
\begin{center}
{\large \underline{Barbara Re}}\\
Politecnico di Milano\\
{\tt barbara.re@mail.polimi.it}
\\ \vspace{4mm}
{\large Roberto Armellin}\\
Politecnico di Milano\\
{\tt armellin@aero.polimi.it}
\\ \vspace{4mm}
{\large Alberto Guardone}\\
Politecnico di Milano\\
{\tt alberto.guardone@polimi.it}
\\ \vspace{4mm}
{\large Nawin Ryan Nannan}\\
Anton de Kom Universiteit van Suriname\\
{\tt ryan.nannan@yahoo.com}
\end{center}

\section*{Abstract}
The computation of the liquid-vapor saturation or coexistence curve using differential algebra techniques is carried out in the present work.
The arbitrary order Taylor expansion of the algebraic system of equations for the specific volume at the liquid and vapor boundaries with respect to the initial condition is used to implement an accurate and computationally efficient algorithm to compute the location of the saturation curve separating the liquid/two-phase/vapor regions.
The considered thermodynamic models are the multi-parameter Span-Wagner equation of state in Helmholtz form and the reference equation of state for carbon dioxide.
The use of differential algebraic techniques is proposed to overcome the two main drawbacks of existing algorithms, namely, the high computational effort required for each solution of the saturation problem and the steep increase of the number of function evaluations close to the liquid-vapor critical point.
Differential algebra allowed us to substitute pointwise solution of the coexistence problem with its Taylor expansion over temperature/pressure intervals.


\bibliographystyle{plain}
\begin{thebibliography}{10}
\bibitem{Modern map methods in particle beam physics}
{\sc M. Berz}. {Modern map methods in particle beam physics}. Academic Press, San Diego, 1999.

\bibitem{Equations of state for technical applications. I. Simultaneously optimized functional form for non-polar and polar fluids}
{\sc R. Span and W. Wagner}. {Equations of state for technical applications. I. Simultaneously optimized functional form for non-polar and polar fluids}. International Journal of Thermophysics, 24(1) pp1-39, 2003.
\end{thebibliography}

\title{Adaptivity in a Discontinuous Petrov-Galerkin Methodology Using Camellia}
\tocauthor{Nathan V. Roberts} \author{} \institute{}
\maketitle
\begin{center}
{\large \underline{Nathan V. Roberts}}\\
University of Texas at Austin\\
{\tt nroberts@ices.utexas.edu}
\\ \vspace{4mm}
{\large Leszek Demkowicz}\\
University of Texas at Austin\\
{\tt leszek@ices.utexas.edu}
\\ \vspace{4mm}
{\large Robert Moser}\\
University of Texas at Austin\\
{\tt rmoser@ices.utexas.edu}
\\ \vspace{4mm}
{\large Jesse Chan}\\
University of Texas at Austin\\
{\tt jchan@ices.utexas.edu}
\end{center}

\section*{Abstract}
The discontinuous Petrov-Galerkin methodology with optimal test functions (DPG) proposed by L. Demkowicz and J. Gopalakrishnan guarantees the optimality of the solution in an energy norm, and provides several features facilitating adaptive schemes.  Whereas Bubnov-Galerkin methods use identical trial and test spaces, Petrov-Galerkin methods allow these function spaces to differ. In DPG, test functions are computed on the fly and are chosen to minimize the residual. For well-posed problems with sufficiently regular solutions, DPG can be shown to converge at optimal rates---the inf-sup constants governing the convergence are mesh-independent, and of the same order as those governing the continuous problem. DPG also provides an accurate mechanism for \emph{measuring} (not merely estimating) the error, and this can be used to drive adaptive mesh refinements.\\
\\
In this work, we use Camellia---a software toolbox for rapid development of DPG solvers, built atop Sandia's Trilinos library of packages---to investigate adaptivity, particularly in the context of incompressible flow problems.  Camellia supports 2D meshes of triangles and quads of variable polynomial order, provides mechanisms for easy specification of DPG variational forms, and supports $h$- and $p$- refinements and distributed computation of the stiffness matrix, among other features.

\bibliographystyle{plain}
\begin{thebibliography}{10}
\bibitem{A Class of Discontinuous Petrov-Galerkin Methods. Part II Optimal Test Functions}
{\sc L. Demkowicz and J. Gopalakrishnan}. {A Class of Discontinuous Petrov-Galerkin Methods. Part II: Optimal Test Functions}. Numerical Methods for Partial Differential Equations, 27(1):70-105, 2011.

\bibitem{The DPG Method for the Stokes Problem}
{\sc N.V. Roberts and T. Bui-Thanh and L. Demkowicz}. {The DPG Method for the Stokes Problem}. Technical Report 12-22, ICES, 2012.

\bibitem{A Toolbox for a Class of Discontinuous Petrov-Galerkin Methods Using Trilinos}
{\sc N.V. Roberts and D. Ridzal and P. B. Bochev and L. Demkowicz}. {A Toolbox for a Class of Discontinuous Petrov-Galerkin Methods Using Trilinos}. Technical Report SAND2011-6678, Sandia National Laboratories, 2011.
\end{thebibliography}

\title{Energy-Conserving Semi-Lagrangian Discontinuous Galerkin Schemes for the Vlasov-Poisson System}
\tocauthor{James Rossmanith} \author{} \institute{}
\maketitle
\begin{center}
{\large \underline{James Rossmanith}}\\
Iowa State University\\
{\tt rossmani@iastate.edu}
\\ \vspace{4mm}
{\large David Seal}\\
Michigan State University\\
{\tt seal@math.msu.edu}
\\ \vspace{4mm}
{\large Andrew Christlieb}\\
Michigan State University\\
{\tt andrewch@math.msu.edu}
\end{center}

\section*{Abstract}
The Vlasov-Poisson system describes the evolution of a collisionless plasma, represented through a probability density function (PDF) that self-interacts via an electrostatic force. One of the main difficulties in numerically solving this system is the severe time-step restriction that arises from parts of the PDF associated with moderate-to-large velocities. The dominant approach in the plasma physics community for removing these time-step restrictions is the so-called particle-in-cell (PIC) method, which discretizes the distribution function into a set of macro-particles, while the electric field is represented on a mesh. The focus of the current work is on so-called semi-Lagrangian methods, which begin with a grid-based Eulerian representation of both the PDF and the electric field, followed by a Lagrangian evolution of the PDF, and ending with a projection that returns the PDF onto the original Eulerian mesh. In particular, we develop in this work a method that discretizes the Vlasov-Poisson system via a high-order discontinuous Galerkin (DG) method in phase space, and an operator split, semi-Lagrangian method in time. The focus of the current work is to resolve all of the Lagrangian dynamics in such a way that (1) high-order accuracy is achieved, (1) mass is exactly conserved, (3) positivity of the PDF is maintained, and (4) total energy is conserved. We apply the resulting scheme to various test cases.

\bibliographystyle{plain}
\begin{thebibliography}{10}
\bibitem{ A positivity-preserving high-order semi-Lagrangian discontinuous Galerkin scheme for the Vlasov-Poisson equations}
{\sc J.A. Rossmanith and D.C. Seal}. { A positivity-preserving high-order semi-Lagrangian discontinuous Galerkin scheme for the Vlasov-Poisson equations}. J. Comput. Phys. 230 (2011) 6203-6232.
\end{thebibliography}

\title{ViennaMesh - a Highly Flexible Meshing Framework}
\tocauthor{Florian Rudolf} \author{} \institute{}
\maketitle
\begin{center}
{\large \underline{Florian Rudolf}}\\
Institute for Microelectronics, Technische Universität Wien\\
{\tt rudolf@iue.tuwien.ac.at}
\\ \vspace{4mm}
{\large Karl Rupp}\\
MCS Division, Argonne National Laboratory\\
{\tt rupp@mcs.anl.gov}
\\ \vspace{4mm}
{\large Siegfried Selberherr}\\
Institute for Microelectronics, Technische Universität Wien\\
{\tt selberherr@iue.tuwien.ac.at}
\end{center}

\section*{Abstract}
While the mesh size 'h' is the main parameter for error estimations in finite element methods, a common assumption is that all cells of a mesh are of 'good' shape. What can be considered a 'good' shape is, however, problem-specific [1]. Even though numerous existing meshing algorithms are able to produce meshes with certain properties, such as the Delaunay property with guaranteed quality and isotropic meshing elements, e.g. [2], the simultaneous use of different meshing algorithms is challenging because of non-unified interfaces. We resolve this deficiency by presenting results and experiences gathered during our work on ViennaMesh [3], a meshing framework offering a common programming interface for a plethora of internally and externally provided meshing algorithms. We discuss the set of abstract concepts defining the generic requirements on functionalities and data structures, which allow for wrapping external data structures and software libraries such as Tetgen, Netgen, and VERDICT. These abstract concepts guarantee exchangeability of algorithms and functionalities in a convenient way. The internal data structure of ViennaMesh was developed with a high level of flexibility regarding the topological structure of mesh elements in mind. The topological connectivity of mesh elements is inspired by [4], and was extended to allow for a more general setting. To avoid performance issues of traditional object-oriented approaches, the required data structures can be configured at compile time to match the particular needs using C++ meta programming techniques. This enables to support the full family of simplices and hypercubes of arbitrary order. Finally, we present applications of ViennaMesh to finite element problems in microelectronics for the solutions of the drift-diffusion system. The required meshes tend to have complicated geometries and very thin layers at material interfaces as well as strong variations of material properties.

\bibliographystyle{plain}
\begin{thebibliography}{10}
\bibitem{What Is a Good Linear Finite Element Interpolation Conditioning Anisotropy and Quality Measures}
{\sc J. R. Shewchuk}. {What Is a Good Linear Finite Element? Interpolation, Conditioning, Anisotropy, and Quality Measures}. CS report, UC Berkeley, 2002.

\bibitem{Guaranteed Quality Tetrahedral Delaunay Meshing for Medical Images}
{\sc P. A. Foteinos et al}. {Guaranteed Quality Tetrahedral Delaunay Meshing for Medical Images}. In Proc. of ISVD, 2010 .

\bibitem{ViennaMesh}
{\sc J. Weinbub et al}. {ViennaMesh}. http://viennamesh.sourceforge.net/.

\bibitem{Efficient Representation of Computational Meshes}
{\sc A. Logg}. {Efficient Representation of Computational Meshes}. Intl. J. Comput. Sci. Eng., 4(4), 2009.
\end{thebibliography}

\title{Simulation of Fluid Flow Through a System of Chambers for Active Vibration Control of Rotating Structures.}
\tocauthor{Rafał Rumin} \author{} \institute{}
\maketitle
\begin{center}
{\large \underline{Rafał Rumin}}\\
AGH University of Science and Technology\\
{\tt rumin@agh.edu.pl}
\\ \vspace{4mm}
{\large Jacek Cieślik}\\
AGH University of Science and Technology\\
{\tt cieslik@agh.edu.pl}
\\ \vspace{4mm}
{\large Marek Bergander}\\
AGH University of Science and Technology\\
{\tt bergander@hartford.edu}
\end{center}

\section*{Abstract}
In this work, we present a simulation of fluid flow through a designed device of the active balancing of rotors with correction mass (liquid). Proposed finite element calculations is based on model of ring with internal chambers, filled with fluid in controlled sequence We describe the mathematical model of the equivalent correction mass allocation.
The numerical model has been tested against an analytical solution for a simplified problem. Numerical modelling allows to investigate physical phenomena and to predict the pressure inside chambers and before selected valves. This paper presents the application of  supply and control of the fluid flow into the selected chamber. 

Keywords: Finite element method/analysis;   water flow inside a rotating machine, balancing rotor, unbalance, rotor dynamics, balance disc

\bibliographystyle{plain}
\begin{thebibliography}{10}
\bibitem{System for Automatic Rotor Balancing Using a Continuous Change of the Correction Mass Distribution}
{\sc R. Rumin and J. Cieślik}. {System for Automatic Rotor Balancing Using a Continuous Change of the Correction Mass Distribution}. Vibrations in Physical Systems, vol. 24 (2010) 337–342.

\bibitem{A Brief History of Early Rotor Dynamics}
{\sc F. C. Nelson}. {A Brief History of Early Rotor Dynamics}. Sound Vib. 37 (2003).

\bibitem{Active Balancing and Vibration Control of Rotating Machinery A Survey}
{\sc S. Zhou and J. Shi }. {Active Balancing and Vibration Control of Rotating Machinery: A Survey}. The Shock and Vibration Digest (2001).

\bibitem{Balancing of rotating systems during operation}
{\sc  J. Van De Vegte and R. T. Lake }. {Balancing of rotating systems during operation}. Journal of Sound and Vibration, 2, 57 (1978).

\bibitem{A calculation processure for heat mass and momentum transfer in the three-dimensional parabolic flows}
{\sc S. V. Pantankar and D. B. Spalding}. {A calculation processure for heat, mass and momentum transfer in the three-dimensional parabolic flows}. International Journal of Heat Mass Transfer, 15, 1787-1806 (1972).

\bibitem{A study on active balancing for rotating machinery using influence coefficient method}
{\sc L. Soo-Hun and K. Bong-Suk and M. Jong-Duck and K. Do-Hyung}. {A study on active balancing for rotating machinery using influence coefficient method}. Computational Intelligence in Robotics and Automation, 659–664, (2005).
\end{thebibliography}

\title{On Level Scheduling for Incomplete LU Factorization Preconditioners on Accelerators}
\tocauthor{Karl Rupp} \author{} \institute{}
\maketitle
\begin{center}
{\large \underline{Karl Rupp}}\\
Argonne National Laboratory\\
{\tt rupp@mcs.anl.gov}
\\ \vspace{4mm}
{\large Barry Smith}\\
Argonne National Laboratory\\
{\tt bsmith@mcs.anl.gov}
\end{center}

\section*{Abstract}
The application of the finite element method for the numerical solution of partial differential equations naturally leads to
large systems of linear equations represented by a sparse system matrix $A$ and right hand side $b$. These systems are commonly solved using iterative solvers, particularly Krylov subspace methods, which are typically accelerated using preconditioners to obtain good convergence rates [1]. One of the most popular families of preconditioners are incomplete LU factorization (ILU) preconditioners, where the system matrix $A$ is factored approximately into a sparse lower-triangular matrix $L$ and a sparse upper-triangular matrix $U$. Then, each application of the preconditioner to a residual vector $z$ involves one forward-substitution $Ly = z$ and one backward substitution $Ux = y$. \\

A drawback of ILU preconditioners is the limited amount of parallelism both in the factorization and in the triangular substitutions. This complicates the efficient implementation on parallel computing architectures such as graphics processing units (GPUs) considerably. Our work is based on previous work by Li and Saad using level scheduling for ILU preconditioners on CUDA-enabled GPUs [2]. We refine their approach by considering modifications of reordering algorithms such as Cuthill-McKee or Gibbs-Poole-Stockmeyer for a higher degree of parallelism and thus higher computational efficiency of ILU preconditioners. Furthermore we apply these techniques to block-ILU preconditioners and compare with the Power(q)-method by Heuveline, Lukarski, and Weiss [3]. Results obtained on GPUs from AMD and NVIDIA as well as on INTEL's many-integrated-core (MIC) architecture will be presented. Our implementations are freely available in the open-source library ViennaCL [4], which is currently integrated into the distributed solver package PETSc [5].

\bibliographystyle{plain}
\begin{thebibliography}{10}
\bibitem{Iterative Methods for Sparse Linear Systems}
{\sc Y. Saad}. {Iterative Methods for Sparse Linear Systems}. 2nd Edition, SIAM (2003).

\bibitem{GPU-Accelerated Preconditioned Iterative Linear Solvers}
{\sc R. Li and Y. Saad}. {GPU-Accelerated Preconditioned Iterative Linear Solvers}. Technical report, University of Minnesota (2010).

\bibitem{Enhanced Parallel ILU(p)-based Preconditioners for Multi-core CPUs and GPUs - The Power(q)-pattern Method}
{\sc V. Heuveline and D. Lukarski and J.-P. Weiss}. {Enhanced Parallel ILU(p)-based Preconditioners for Multi-core CPUs and GPUs - The Power(q)-pattern Method}. EMCL Preprint 2011-08 (2011).

\bibitem{ViennaCL Web Page}
{\sc K. Rupp and F. Rudolf and J. Weinbub}. {ViennaCL Web Page}. http://viennacl.sourceforge.net/.

\bibitem{PETSc Web page}
{\sc S. Balay and J. Brown and K. Buschelman and W. D. Gropp and D. Kaushik and M. G. Knepley and L. C. McInnes and B. F. Smith and H. Zhang}. {PETSc Web page}. http://www.mcs.anl.gov/petsc/.
\end{thebibliography}

\title{An Unfitted Finite Element Method for the Approximation of Void Electromigration}
\tocauthor{Andrea Sacconi} \author{} \institute{}
\maketitle
\begin{center}
{\large Andrea Sacconi}\\
Department of Mathematics - Imperial College London, UK\\
{\tt a.sacconi11@imperial.ac.uk}
\end{center}

\section*{Abstract}
Microelectronic circuits usually contain small voids or cracks, and if those defects are large enough to sever the line, they cause an open circuit. We present a numerical method for investigating the migration of voids in the presence of both surface diffusion and electric loading. Our mathematical model involves a bulk-interface coupled system, with a moving interface governed by a fourth-order geometric evolution equation and a bulk where the electric potential is computed. Thanks to a proper approximation of the interface, equidistribution of its vertices is guaranteed, therefore no re-meshing is necessary for the interface.
Numerical challenges include the coupling of the two sets of equations and the proper definition of the bulk mesh at each time step: the algorithm used to identify cut, inside and outside elements, as well as local adaptivity are explained in detail. Various examples are performed with a C++-based code to demonstrate the accuracy of the method.

\bibliographystyle{plain}
\begin{thebibliography}{10}
\bibitem{On the Parametric Finite Element Approximation of Evolving Hypersurfaces in R3}
{\sc J. W. Barrett and H. Garcke and R. N\"{u}rnberg}. {On the Parametric Finite Element Approximation of Evolving Hypersurfaces in ${R}^3$}. J. Comput. Phys. 227 (2008) 4281-4307.

\bibitem{Finite Element Approximation of a Phase Field Model for a Void Electromigration}
{\sc J. W. Barrett and R. N\"{u}rnberg and V. Styles}. {Finite Element Approximation of a Phase Field Model for a Void Electromigration}. SIAM J. Numer. Anal.  42 (2) (2004) 738-772.

\bibitem{On Stable Parametric Finite Element Methods for the Stefan Problem and the Mullins-Sekerka Problem with Applications to Dendritic Growth}
{\sc J. W. Barrett and H. Garcke and R. N\"{u}rnberg}. {On Stable Parametric Finite Element Methods for the Stefan Problem and the Mullins-Sekerka Problem with Applications to Dendritic Growth}. J. Comput. Phys. 229 (2010) 6270-6299.
\end{thebibliography}

\title{High-Performance Computing in Simulating Carbon Dioxide Geologic Sequestration}
\tocauthor{Eduardo Sanchez} \author{} \institute{}
\maketitle
\begin{center}
{\large \underline{Eduardo Sanchez}}\\
San Diego State University\\
{\tt esanchez@sciences.sdsu.edu}
\\ \vspace{4mm}
{\large Christopher Paolini}\\
San Diego State University\\
{\tt paolini@engineering.sdsu.edu}
\\ \vspace{4mm}
{\large Jose Castillo}\\
San Diego State University\\
{\tt jcastillo@mail.sdsu.edu}
\end{center}

\section*{Abstract}
Carbon Dioxide (CO$_2$) Capture, Utilization and Geologic Sequestration
is a collection of technologies that seek to minimize the environmental
impacts of greenhouse gases. Specifically, to study and simulate the long-term
effects of geologic CO$_2$ sequestration on reservoir pressure and induced
seismicity, numerical water-rock interaction and reactive transport models are
employed. Traditionally, numerical codes that simulate water-rock interaction
and reactive transport sequentially solve an elemental mass-balance equation
for a given lithology containing some fraction
of brine, and a number of charged aqueous solute species. Pressure,
temperature, and solute concentrations are then sequentially solved in separate
modules and coupled through an iterative process, until a convergence criteria
is satisfied. Coupling is achieved by iterating between the discretization of
conservation of mass and energy equations, together with equations modeling kinetic and
equilibrium reactions. Mass-transfer coefficient matrices constructed from
formation and injectant water velocities and solute concentrations, derived
from a previous iteration, are structured and then solved using direct
methods for systems of linear equations. However,
this formulation is not well suited for execution on many-core distributed
clusters. In this work, we present a numerical scheme
whereby all solute concentrations in all control volumes are solved
simultaneously by constructing a large block-banded sparse matrix of rank
$N_a\times N_x$, where $N_a$ is the number of solutes species, and $N_x$
is the number of control volumes. The generated matrices are then factored using
tools provided by different APIs implementing direct methods for the solution
of systems of linear equations. Performance metrics are considered to compare
our large block-banded matrix scheme against a sequential implementation on the
$\sim$10K core \textit{XSEDE} cluster \texttt{trestles.sdsc.edu}. Simulations based
on the Frio Formation pilot test case are studied with respect to achieved speedup,
efficiency, scalability and grid refinement.

\bibliographystyle{plain}
\begin{thebibliography}{10}
\bibitem{A Distributed Mimetic Approach to Simulating Water-Rock Interaction following CO2 Injection in Sedimentary Basins}
{\sc C. Paolini and E. Sanchez and A. Park and J. Castillo}. {A Distributed Mimetic Approach to Simulating Water-Rock Interaction following CO2 Injection in Sedimentary Basins}. 2011 SIAM Conference on Analysis of Partial Differential Equations, San Diego, California, November 14-17, 2011.
\end{thebibliography}

\title{Advances on the Development of the Mimetic Methods Toolkit (MTK): An Object-oriented API Implementing Mimetic Discretization Methods for the Numerical Solution of Partial Differential Equations}
\tocauthor{Eduardo Sanchez} \author{} \institute{}
\maketitle
\begin{center}
{\large \underline{Eduardo Sanchez}}\\
San Diego State University\\
{\tt ejspeiro@gmail.com}
\\ \vspace{4mm}
{\large Christopher Paolini}\\
San Diego State University\\
{\tt paolini@engineering.sdsu.edu}
\\ \vspace{4mm}
{\large Jose Castillo}\\
San Diego State University\\
{\tt jcastillo@mail.sdsu.edu}
\end{center}

\section*{Abstract}
The  Mimetic Methods Toolkit (MTK) is an object-oriented Application Programming Interface, which novelty lies on the implementation of Mimetic Discretization Methods in developing scientific computer applications, where the numerical solution of Partial Differential Equations may be required. The MDMs yield numerical solutions that guarantee a uniform order of accuracy, while ensuring the satisfaction of conservative laws. The MTK was designed based on the Castillo–Grone Method for the construction of discrete differential operators that mimic important properties of their continuous counterparts. The MTK is built as a collection of abstract and concrete classes, thus allowing for an extensible framework, which fosters code reutilization, while intuitively educating the user about the important theoretical aspects of the Mimetic Discretization Methods. We present an introduction to Mimetic Discretization Methods, and we discuss the computational modeling of the related concepts; in this way, we explain how the MTK implements these methods. We present the collection of resources that are available online, as for example, its related documentation, available tutorials; as well as demos from problems depicting the advances on the development of the MTK.

\bibliographystyle{plain}
\begin{thebibliography}{10}
\bibitem{Mimetic Discretization Methods}
{\sc J. Castillo and G. Miranda}. {Mimetic Discretization Methods}. CRC Press. (2013).

\bibitem{A  matrix analysis approach to higher-order approximations for divergence and gradients satisfying a global conservation law}
{\sc J. Castillo and R. Grone}. {A  matrix analysis approach to higher-order approximations for divergence and gradients satisfying a global conservation law}. Siam J. Matrix Anal. Appl. 25 (2003) 182-142.

\bibitem{Linear systems arising from second-order mimetic divergence and gradient discretizations}
{\sc J. Castillo and M. Yasuda}. {Linear systems arising from second-order mimetic divergence and gradient discretizations}. Journal of Mathematical Modeling and Algorithms. 4 (2005) 128-142.
\end{thebibliography}

\title{Higher-order Solver for Nonlinear Bioheat Equation Modeling Magnetic Fluid Hyperthermia}
\tocauthor{Bartosz Sawicki} \author{} \institute{}
\maketitle
\begin{center}
{\large \underline{Bartosz Sawicki}}\\
Warsaw University of Technology\\
{\tt bartosz.sawicki@ee.pw.edu.pl}
\\ \vspace{4mm}
{\large Arkadiusz Miaskowski}\\
University of Life Sciences in Lublin\\
{\tt arek.miaskowski@up.lublin.pl}
\end{center}

\section*{Abstract}
Magnetic Fluid Hyperthermia (MFH) is a promising therapeutic tool for cancer treatment. The roots of this method are based on the observation that human tissues are sensitive to the temperature changes and heating them above 42 degrees Celsius activates self-destruction, which is called apoptosis in biology.  
The main concern of hyperthermia is to precisely control the amount and location of heat provided to the cancer. Overheating the surrounding tissues should be minimized since it could cause side-effects. Among different technologies used for the increase of  the body temperature MFH is special, because it is based on magnetic nanoparticles, which are injected directly into the place of interest. External, low frequency magnetic field is passing through the body without interferences, only the area with nanoparticles is excited.

The main challenges of numerical modeling of MFH are related to the complicated nature of living tissues (complex shapes, nonlinearity),  multi physics (electic, magnetic and thermal) and time domain simulations. 
In this paper we focus of the simulation of bioheat transfer process. Pennes differential equation is a classical solution for such a problem. It is usually solved by simple backward Euler scheme with fixed time step, which is known to be stable. 

We are taking advantage of more advanced numerical integration algorithms. The stability, efficiency and correctness of Euler methods will be compared with a trapezoidal rule and higher-order algorithms from Runge-Kutta family. Finally, we will present singly diagonally implicit Runge-Kutta method with an explicit first stage (ESDIRK),which outperforms others. Adaptive time step size approach allows to reduce computational costs. 
Numerical experiments will we be conducted on the basis of three dimensional models of the breast with tumorous tissue, with different levels of mesh complexity. 

\bibliographystyle{plain}
\begin{thebibliography}{10}
\bibitem{Impact of nonlinear heat transfer on temperature control in regional hyperthermia.}
{\sc Lang  and Jens  and Bodo Erdmann and and Martin Seebass}. {Impact of nonlinear heat transfer on temperature control in regional hyperthermia.}. Biomedical Engineering, IEEE Transactions on 46.9 (1999): 1129-1138..

\bibitem{Singly diagonally implicit RungeKutta methods with an explicit first stage}
{\sc Kværnø  and Anne }. {Singly diagonally implicit Runge–Kutta methods with an explicit first stage}. BIT Numerical Mathematics 44.3 (2004): 489-502..

\bibitem{Synthesizing Distributions of Magnetic Nanoparticles for Clinical Hyperthermia}
{\sc Paolo Di Barba and Fabrizio Dughiero and and Elisabetta Sieni}. {Synthesizing Distributions of Magnetic Nanoparticles for Clinical Hyperthermia}. Magnetics, IEEE Transactions on 48.2 (2012): 263-266..

\bibitem{Magnetic fluid hyperthermia modeling based on phantom measurements and realistic breast model}
{\sc Arkadiusz Miaskowski and Bartosz Sawicki}. {Magnetic fluid hyperthermia modeling based on phantom measurements and realistic breast model}. IEEE Transactions on Biomedical Engineering, (accepted for publication), 2013.
\end{thebibliography}

\title{Dynamical $Hp$-Meshes With Specifiable Error Tolerances for Discontinuous Galerkin Time-Domain Computations}
\tocauthor{Sascha M Schnepp} \author{} \institute{}
\maketitle
\begin{center}
{\large Sascha M Schnepp}\\
Laboratory for Electromagnetic Fields and Microwave Electronics (IFH), ETH Zurich, Switzerland\\
{\tt schnepps@ethz.ch}
\end{center}

\section*{Abstract}
The computation of numerical solutions using adaptive methods has a number 
of advantages, the most obvious one being reduced demands on memory 
consumption and shorter computing times. However, in the context of Finite 
Element Methods adaptivity usually is applied for problems in statics or in the frequency 
domain, if at all. The problem is solved on a given mesh, elementwise errors are estimated and 
those elements with the largest errors are refined. This cycle is repeated until a 
given error tolerance is met. However, if translated into the time-domain, this 
scheme causes excessive computational costs, thus, missing its purpose and losing 
practicality. In this talk, an adaptive scheme for time-domain problems based on the 
Discontinuous Galerkin Method is presented. The extra computational costs of the 
scheme are comparatively small, and it allows for specifying an error tolerance, 
which is respected throughout the simulation. Its benefit will be demonstrated with a number of examples.


\bibliographystyle{plain}
\begin{thebibliography}{10}
\bibitem{Adaptive hp-FEM with dynamical meshes for transient heat and moisture transfer problems}
{\sc P. Solin and L. Dubcova and J. Kruis}. {Adaptive hp-FEM with dynamical meshes for transient heat and moisture transfer problems}. J. Comput. Appl. Math. 233(12) 3103-3112.

\bibitem{Computing with hp-adaptive finite elements}
{\sc L. Demkowicz}. {Computing with hp-adaptive finite elements}. Chapman \& Hall.
\end{thebibliography}

\title{Mathematical Modeling of Ground Water Conditions Using Electrical Resistivity Method in Kaushambi Region, India}
\tocauthor{Pitam Singh} \author{} \institute{}
\maketitle
\begin{center}
{\large \underline{Pitam Singh}}\\
Department of Mathematics, Motilal nehru National Institute of Technology, Allahabad, India\\
{\tt pitamsn@gmail.com}
\\ \vspace{4mm}
{\large Priyamvada Singh}\\
Department of  Earth and Planetary Sciences, Universty of Allahabad,  India\\
{\tt priyam028@yahoo.com}
\\ \vspace{4mm}
{\large Shusil Maurya}\\
Department of Earth and Planetray Sciences, University of Allahabad, India\\
{\tt shusileps31yahoo.com}
\end{center}

\section*{Abstract}
The vertical electrical sounding is a suitable geophysical method for underground water studies. Geophysical exploration using the $50$ vertical electrical sounding was carried out to determine the ground water conditions during the year 2004-05.The present work described the results of vertical electrical sounding survey carried out using Shlumberger configuration in parts of Naveda blocks Kaushambi district( Latitude, $25.2776^{0}$N to $25.4489^0$ and Longitude $81.5034^0$ to $81.8118^0$), Uttar Pradesh, India. The interpretation was done with the help of three and two layer master curves and auxiliary point charts. The study showing that the average depth of aquifer is $60$ meters and thickness is 45 meter. The section which is parallel to the Yamuna river is indicating a lower depth of aquifer then the others. The first layer of surface is soil and then medium grain send and then clay layer, this is the average average formation in this area.

\bibliographystyle{plain}
\begin{thebibliography}{10}
\bibitem{A Simple Computer Iteration Technique for the interpretation of vertical electical sounding}
{\sc M.B. Asokhia and S.O. Azi and Ujuanbi undefined}. {A Simple Computer Iteration Technique for the interpretation of vertical electical sounding}. J. Nig. Assoc.Math. Phys.(2000) 269.

\bibitem{Investigations of Surface Ground Water resources within the Eastern Niger-Delta}
{\sc J. O. Etu-Efeotoer and A. Michalski and E.J.I. Alabo}. {Investigations of Surface Ground Water resources within the Eastern Niger-Delta}. J. Mining Geol.25(1971)51-54.

\bibitem{The use of vertical electrical sounding resistivity method for the location of low salinity groundwater for irrigation in Chaj and Rachna Doabs}
{\sc  Pervaiz Sikandar and Allah Bakhsh and Muhammad Arshad and Tariq Rana}. {The use of vertical electrical sounding resistivity method for the location of low salinity groundwater for irrigation in Chaj and Rachna Doabs}.  Environmental Earth Sciences (2010) 60(5) 1113-1129.
\end{thebibliography}

\title{Seismic Wave Attenuation Characteristics  From Coda Analysis in Garhwal Himalayas, India}
\tocauthor{Priyamvada Singh} \author{} \institute{}
\maketitle
\begin{center}
{\large \underline{Priyamvada Singh}}\\
Uinversity of Allahabad\\
{\tt priyam028@yahoo.com}
\\ \vspace{4mm}
{\large J. N. Tripathi}\\
Uinversity of Allahabad\\
{\tt jntripa@gmail.com}
\end{center}

\section*{Abstract}
Digital seismogram data of $75$ earthquakes that occurred in Gharhwal  Himalya region during 2004 to 2006 and recorded at different  stations been analyzed to study the seismic coda wave attenuation characteristic . Seismic coda wave attenuation,$Q_{c},$ characteristic in this region is studied using  128 seismic observations from local earthquake events with hypocentral distance less than 250 Km and magnitude range between 1.0 to 4.0. Coda wave attenuation $Q_{c}$ is estimated using the single isotropic scattering model. The coda wave is consider to be composed of single back scattered waves from randomly distributed inhomogeneities.

$Q_{c}$ values are estimated at $10$ central frequencies $1.5, 3, 5, 7, 9, 12, 16, 20, 24$ and $28$ Hz using several starting lapse-times and coda window-lengths. Lapse time is considered from the origin of an earthquake. The $Q_{c}$  values are frequency dependent. In the considered frequency range they fit the frequency dependent power-law $Q_{c} = Q_{0}f^{n}$. The frequency dependent power law for $10$ sec lapse-time $10$ sec coda window-length is $Q_{c} = 47.42f^{1.012}$, which correlates well with the values obtained in other seismically and tectonically active and heterogeneous regions of the world.

The variation of coda attenuation has been estimated for different lapse-time and window-length combinations to observe the affect with depth. Results indicate that in the study region the heterogeneity decreases with increasing depth. Attenuation parameter $Q_{c}$  is an important factor for understanding the physical mechanism of seismic wave attenuation in relation to the composition and physical condition of the Earth's interior (Sato, 1992) and it is also an indispensable parameter  for the quantitative prediction of strong ground motion for the viewpoint of engineering seismology. Hence numerous studies of $Q_{c}$  have been carried out worldwide by using different methods and concentrate on seismically active zones and densely populated area.

\bibliographystyle{plain}
\begin{thebibliography}{10}
\bibitem{Analysis of the seismic coda of local earthquakes as scattered waves}
{\sc K. Aki}. {Analysis of the seismic coda of local earthquakes as scattered waves}. J. Geophys. Res.74(1969) 615-631.

\bibitem{Scattering and attenuation of shear waves in the lithosphere}
{\sc K. Aki}. {Scattering and attenuation of shear waves in the lithosphere}. J. Geophys. Res.85(1980)6496–6504.

\bibitem{Origin of the coda waves source attenuation and scattering effects}
{\sc K . Aki and B. Chouet}. {Origin of the coda waves: source, attenuation and scattering effects}. J. Geophys. Res.80(1975) 3322–3342.

\bibitem{ Attenuation of coda waves in the Garhwal Himalayas India}
{\sc  S.C. Gupta and V.N. Singh and A. Kumar}. { Attenuation of coda waves in the Garhwal Himalayas, India}. Phys. Earth Planet Inter.87(1995)247–253.

\bibitem{Variations of seismic coda wave attenaution in the Garhwal region Northwestern Himalaya }
{\sc J.N undefined and Tripathi undefined and P. Singh and M.L. Sharma}. {Variations of seismic coda wave attenaution in the Garhwal region, Northwestern Himalaya }. Pure and Applied Geophy.169(2012)71-88.
\end{thebibliography}

\title{Combined Adaptive Hp-FEM/hp-DG With Dynamical Meshes for Time-Dependent Multiphysics Coupled Problems}
\tocauthor{Pavel Solin} \author{} \institute{}
\maketitle
\begin{center}
{\large Pavel Solin}\\
University of Nevada, Reno\\
{\tt solin.pavel@gmail.com}
\end{center}

\section*{Abstract}
When using advanced computational methods such as hp-FEM and hp-DG for multiphysics coupled problems, the design of mathematical methods and the design of the software platform for their implementation are tightly connected both ways. In particular, good software design makes it possible to develop better mathematical methods. We will present several novel techniques for higher-order adaptive FEM and DG methods and show how they can be combined elegantly in an object-oriented software design. This software architecture was created in the framework of NEUP project sponsored by the U.S. Department of Energy. Using this approach, we will show how to combine various physical fields and equations in multiphysics PDE systems, use different meshes to approximate specific behavior of each field, split the spatial and temporal discretization to eliminate interference between spatial discretizations and time integration methods, and to combine hp-FEM and hp-DG in a monolithic way for problems where only some of the quantities need to be discretized via DG. We will also show the importance of controlled-accuracy time integration based on adaptive implicit higher order time-integration methods.

\bibliographystyle{plain}
\begin{thebibliography}{10}
\bibitem{Adaptive hp-FEM with Dynamical Meshes for Problems with Traveling Sharp Fronts}
{\sc P. Solin and L. Korous}. {Adaptive hp-FEM with Dynamical Meshes for Problems with Traveling Sharp Fronts}. Computing, DOI 10.1007/s00607-012-0243-7, 2012..

\bibitem{Adaptive Higher-Order Finite Element Methods for Transient PDE Problems Based on Embedded Higher-Order Implicit Runge-Kutta Methods}
{\sc P. Solin and L. Korous}. {Adaptive Higher-Order Finite Element Methods for Transient PDE Problems Based on Embedded Higher-Order Implicit Runge-Kutta Methods}. Journal of Computational Physics, Volume 231, Issue 4, pp. 1635-1649.

\bibitem{Solving the Nonstationary Richards Equation with Adaptive hp-FEM}
{\sc P. Solin and M. Kuraz}. {Solving the Nonstationary Richards Equation with Adaptive hp-FEM}. Advanced Water Resources, Vol. 34, Issue 9, 2011, pp. 1062-1081.
\end{thebibliography}

\title{Trigonometric Approximation of Periodic Functions Belonging to $Lip(\xi(t), P)$- Class}
\tocauthor{Shailesh Kumar Srivastava} \author{} \institute{}
\maketitle
\begin{center}
{\large \underline{Shailesh Kumar Srivastava}}\\
Indian Institute of Technology Roorkee, Roorkee, Uttarakhand, 247667\\
{\tt shaileshiitr2010@gmail.com}
\\ \vspace{4mm}
{\large Uaday Singh}\\
Indian Institute of Technology Roorkee, Roorkee, Uttarakhand, 247667\\
{\tt usingh2280@yahoo.co.in}
\end{center}

\section*{Abstract}
Various investigators have defined function class  $Lip(\xi(t), p),$ $p\geq1$ in terms of integral modulus of continuity as follows

\begin{equation}
Lip(\xi(t), p)=\left\{f\in L_{p}[0,2\pi]: \left(\int_{0}^{2\pi}\mid f(x+t)-f(x)\mid^{p}dx \right)^{1/p}= O(\xi(t))\right\},
 \end{equation}
 where $\xi(t)$ is a positive increasing function. Khan and Ram [On the degree of approximation, Ser. Math. Inform. 18(2003), 47-57] defined $Lip(\xi(t), p)$ for $p> 1$ in a different manner as follows
 \begin{equation}
Lip(\xi(t), p)=\left\{f\in L_{p}[0,2\pi]: \mid f(x+t)-f(x)\mid^{p}= O(t^{-1/p}\xi(t))\right\}, 0 < t<\pi ,
 \end{equation}
and discussed the degree of approximation of $f \in Lip(\xi(t), p),$ $p >1$ using the Euler`s means of Fourier series. In this paper, we discuss a relationship between two definitions and obtain degree of approximation of $f \in Lip(\xi(t), p),$ $p\geq 1$ using matrix means of Fourier series of $f$ generated by $T\equiv (a_{n,k}).$

\bibliographystyle{plain}
\begin{thebibliography}{10}
\bibitem{On The Degree Of Approximation}
{\sc H. H. Khan and G. Ram}. {On The Degree Of Approximation}. Ser. Math. Inform. 18 (2003), 47-57.
\end{thebibliography}

\title{Design and Development of Mobile Access for DSpace Open Source Institutional Repository on Hybrid Cloud Computing}
\tocauthor{Adisak Sukul} \author{} \institute{}
\maketitle
\begin{center}
{\large Adisak Sukul}\\
Department of Computer Science, Faculty of Science, King Mongkut's Institute of Technology Ladkrabang, Thailand.\\
{\tt adisak.sukul@gmail.com}
\end{center}

\section*{Abstract}
Since March 2000, Massachusetts Institute of Technology (MIT) Libraries and Hewlett-Packard Labs have been collaborating on the development of an open source system called DSpace™ that functions as a repository for the digital research and educational material produced by members of a research university or organization.

Running such an institutionally-based, multidisciplinary repository is increasingly seen as a natural role for the libraries and archives of research and teaching organizations. 

As their constituents produce increasing amounts of original material in digital formats, the repository play an important role of new media in education system.

This paper describes the design and development of mobile application for accessing the Dspace Repository.
The case study demonstrates the design and development of large digital library for educational collection in Thailand. It is provide access to millions of digital items, including legislative documents and related e-books, journals, newspapers, videos for the country wide access. As the number and size of documents grow dramatically every year, we need a scalable yet reliable system to handle this massive load. 

The working group chose to substantially extend the DSpace, an open source repository system, to serve as a cost effective core repository system. Along with 10+ private cloud instance for server farm infrastructure, Google Search Appliance for the search engine front-end, LDAP server is implemented as main authentication service, a Digital Rights Management system to authorize user access, web and mobile client to allow patrons to download and access digital contents off-line via Windows, iPad and Android devices.

Since the repository comprise of both open-access and licensed documents, the DRM-Digital Rights Management system have been developed to manage the access policies, encode the documents and authorize user. There are also strict requirement from the documents owner that their documents have to store only on in-house datacenter, the system have to be hybrid cloud to meet requirements and maximize reliabilities. In this project we have also developed dedicated mobile appellations for Windows, iOS and Android devices. 

The heart of DSpace is a free and open source (FOSS) storage and retrieval system, which allows the repository to be scalable and customizable. Potentially, DSpace will lead to the creation of a virtual library that meshes the collections of various research libraries. 

DSpace repository has OAI-PMH built-in which means that it is ready to communicate with other repositories, thereby making future cooperation of different legislative departments with other libraries very easy.


\bibliographystyle{plain}
\begin{thebibliography}{10}
\bibitem{DSpace An Open Source Dynamic Digital Repository}
{\sc Smith undefined and MacKenzie; Barton and Mary; Bass and Mick; Branschofsky and Margret; McClellan and Greg; Stuve and Dave; Tansley and Robert; Walker and Julie Harford}. {DSpace: An Open Source Dynamic Digital Repository}. D-Lib Magazine, Volume 9 Number 1, January 2003..
\end{thebibliography}

\title{A Dirichlet/Robin Iteration-by-subdomain Method for an Anisotropic, Nonisothermal Two-phase Transport Model of PEM Fuel Cell With Micro-porous Layer}
\tocauthor{Pengtao Sun} \author{} \institute{}
\maketitle
\begin{center}
{\large Pengtao Sun}\\
Department of Mathematical Sciences, University of Nevada Las Vegas\\
{\tt pengtao.sun@unlv.edu}
\end{center}

\section*{Abstract}
In this paper, we present a nonoverlapping domain decomposition method, Dirichlet/Robin type alternately iteration-by-subdomain
method, for a 3D anisotropic, nonisothermal, multiphysics, two-phase transport model of proton exchange membrane fuel cell (PEMFC), particularly bearing micro-porous layer (MPL) and wet gas channel (GC). In this method, the necessary transmission conditions are introduced by nonlinear Dirichlet/Robin interfacial boundary conditions at the interfaces of some specific subdomains, i.e., of gas diffusion layer (GDL) and MPL due to the discontinuous liquid saturation [1] therein, and of GDL and GC due to the discontinuous Kirchhoff transformation arising from wet gas channel [2]. We then use these transmission conditions to design a type of nonoverlapping domain decomposition method with the combined finite element-upwind finite volume discretizations to deal with water transports amongst such multi-layer diffusion media. Kirchhoff transformation and its inverse techniques are employed to overcome the discontinuous and degenerate water diffusivity in the coexisting single- and two-phase regions. Numerical simulations demonstrate that the presented techniques are effective to obtain a fast and convergent nonlinear iteration for such a 3D multiphysics PEMFC model in contrast with the oscillatory and nonconvergent iteration conducted by standard FEM/FVM. A series of numerical convergence tests are also carried out to verify the efficiency and accuracy of the presented numerical techniques.

\bibliographystyle{plain}
\begin{thebibliography}{10}
\bibitem{Two-phase transport and the role of micro-porous layer in polymer electrolyte fuel cells}
{\sc U. Pasaogullari and C. Wang}. {Two-phase transport and the role of micro-porous layer in polymer electrolyte fuel cells}. Electrochim. Acta, 49 (2004), 4359–4369..

\bibitem{A domain decomposition method for two-phase transport model in the cathode of a polymer electrolyte fuel cell}
{\sc P. Sun and G. Xue and C.Wang undefined and J. Xu}. {A domain decomposition method for two-phase transport model in the cathode of a polymer electrolyte fuel cell}. J. Comput. Phys., 228 (2009), 6016–6036..
\end{thebibliography}

\title{Error Estimates of Finite Element MethodWith Kirchhoff Transformation for a Two-phase Transport Model of Proton Exchange Membrane Fuel Cell}
\tocauthor{Yuzhou Sun} \author{} \institute{}
\maketitle
\begin{center}
{\large \underline{Yuzhou Sun}}\\
University of Nevada, Las Vegas\\
{\tt SUNY5@UNLV.NEVADA.EDU}
\\ \vspace{4mm}
{\large Mingyan He}\\
Tongji University\\
{\tt hemingyan1985@yahoo.com.cn}
\\ \vspace{4mm}
{\large Pengtao Sun}\\
University of Nevada, Las Vegas\\
{\tt pengtao.sun@unlv.edu}
\end{center}

\section*{Abstract}
In this paper we analyze the error estimates of finite element approximation with Kirchhoff transformation
for a 2D two-phase transport model of water species in the cathode gas diffusion layer
of proton exchange membrane fuel cell (PEMFC). We introduce an equivalent water concentration
equation by using Kirchhoff transformation [1,2] and address how efficiently it deals with the discontinuous
and degenerate diffusivity. The semi-discrete finite element scheme and fully discrete
Crank-Nicolson-finite element scheme are designed for the present model, and their optimal error
estimates in H1 norm are derived. Numerical experiments show that the obtained numerical solutions
coincide with the theoretical results.

\bibliographystyle{plain}
\begin{thebibliography}{10}
\bibitem{Fast numerical simulation of two-phase transport model in the cathode of a polymer electrolyte fuel cell}
{\sc P. SUN and G. XUE and C. WANG and J. XU}. {Fast numerical simulation of two-phase transport model in the cathode of a polymer electrolyte fuel cell}. Communications in Computational Physics, 6 (2009), 49–71.

\bibitem{A domain decomposition method for two-phase transport model in the cathode of a polymer electrolyte fuel cell}
{\sc P. SUN and G. XUE and C. WANG and J. XU}. {A domain decomposition method for two-phase transport model in the cathode of a polymer electrolyte fuel cell}. Journal of Computational Physics, 228 (2009), 6016–6036.
\end{thebibliography}

\title{On Numerical Simulation of Three-Dimensional Flow Problems by Finite Element and Finite Volume Techniques}
\tocauthor{Petr Sváček} \author{} \institute{}
\maketitle
\begin{center}
{\large Petr Sváček}\\
Czech Technical University in Prague, Faculty of Mechanical Engineering, Department of Technical Mathematics\\
{\tt Petr.Svacek@fs.cvut.cz}
\\ \vspace{4mm}
{\large Petr Louda}\\
Czech Technical University in Prague, Faculty of Mechanical Engineering, Department of Technical Mathematics\\
{\tt louda@marian.fsik.cvut.cz}
\end{center}

\section*{Abstract}
In this paper the numerical approximation of three-dimensional incompressible turbulent flow 
in three dimensional computational domains is considered.
The mathematical model is based on Reynolds averaged Navier-Stokes (RANS) equations enclosed by a turbulence model.
The discretization of the problem is done by the finite volume and finite element numerical techniques.


First, the finite element method is applied for discretization of the problem.
The applications of general grid is allowed. This allows the use of hexahedral grids 
which are popular due to better approximation properties  within the same number of vertices.
On the other hand for complicated domains it can be difficult to create such a mesh for some domains arising in technical applications.
The presented finite element method allows the flexible use of different types of elements,
i.e. grid with combination of hexahedrons, tetrahedrons, pyramids and prisms.
The piecewise trilinear/linear finite elements are used for both velocity and pressure components approximations.
The streamline-upwind/Petrov-Galerkin together with pressure stabilizing/Petrov-Galerkin techniques is used.



Second, the finite volume technique is applied. In this case the turbulent flow is  mathematically modeled by RANS equations
coupled with the explicit algebraic Reynolds stress model. The numerical solution is carried out by the implicit finite volume method,
where the incompressibility constraint is solved with the aid of artificial compressibility method.



The numerical results of both methods are compared. The comparison with experimental data is also done.


\bibliographystyle{plain}
\begin{thebibliography}{10}
\bibitem{Turbulence Modeling for CFD}
{\sc D. C. Wilcox.}. {Turbulence Modeling for CFD}. DCW Industries, La Ca\~{n}ada, CA, 1993..

\bibitem{Numerical Simulation of Separated Flows in Channels}
{\sc P. Louda and K. Kozel and P. Sváček and J. Příhoda}. {Numerical Simulation of Separated Flows in Channels}. Journal of Thermal Science. 2012, vol. 21, no. 2, p. 145-153.

\bibitem{ Flow Over Backward Facing Step with Inclined Wall Solved by Finite Volume and Finite Element Method}
{\sc P. Louda and P. Sváček and K. Kozel and J. Příhoda}. { Flow Over Backward Facing Step with Inclined Wall Solved by Finite Volume and Finite Element Method}. In: Numerical Analysis and Applied Mathematics, Vols I - III. New York: American Institute of Physics, 2010, p. 99-102.

\bibitem{On spurious oscillations at layers diminishing (SOLD) methods for convectiondiffusion equations Part I  A review.}
{\sc V. John and P. Knobloch}. {On spurious oscillations at layers diminishing (SOLD) methods for convection–diffusion equations: Part I – A review.}. Computer Methods in Applied Mechanics and Engineering, 196(17–20):2197 - 2215, 2007..
\end{thebibliography}

\title{On Mathematical Modeling of Fluid - Structure Interactions With Nonlinear Effects: Application of Finite Element Method.}
\tocauthor{Petr  Sváček} \author{} \institute{}
\maketitle
\begin{center}
{\large Petr  Sváček}\\
Czech Technical University in Prague, Faculty of Mechanical Engineering, Department of Technical Mathematics\\
{\tt Petr.Svacek@fs.cvut.cz}
\end{center}

\section*{Abstract}
The coupled problems describing the interactions of fluid flow with elastic structure are of great importance in many engineering applications. 
In the technical practice typically the determination of the stability of the system is of interest only, and thus the 
problem is modeled only in linear regime. Recently, the research focuses also on 
numerical modeling of nonlinear coupled problems. The behaviour in post-critical states
cannot be captured within the linear analysis. 
The nonlinear approach allows the determination of the character of the flutter boundary, which can be either acceptable(moderate sustained vibration amplitudes) or catastrophic.

Here, the  finite element method is  employed for the numerical solution of the mutual interaction of incompressible viscous flow and elastic structure.
The mathematical model consists of fluid flow description, structure motion
equations and the interface conditions.
The structure motion described by the equations of motions (system of ordinary differential equations). The flow is governed by Navier-Stokes  equations.
In order to capture the motion of the computational domains the Arbitrary Lagrangian-Eulerian(ALE) method is employed.
For structure mathematical models with nonlinearities are used. 

The numerical solution of several benchmark problems is performed and the numerical technique is applied for simulation 
of post-critical behaviour of the airfoil.


\bibliographystyle{plain}
\begin{thebibliography}{10}
\bibitem{Numerical Simulation of Glottal Flow in Interaction with Self Oscillating Vocal Folds Comparison of Finite Element Approximation with a Simplified Model}
{\sc P. Sváček and J. Horáček}. {Numerical Simulation of Glottal Flow in Interaction with Self Oscillating Vocal Folds: Comparison of Finite Element Approximation with a Simplified Model}. Communications in Computational Physics. 2012, vol. 12, no. 3, p. 789-806. .

\bibitem{Numerical Analysis of Flow - Induced Nonlinear Vibrations of an Airfoil with Three Degrees of Freedom}
{\sc M. Feistauer and J. Horáček and M. Růžička and P. Sváček}. {Numerical Analysis of Flow - Induced Nonlinear Vibrations of an Airfoil with Three Degrees of Freedom}. Computers \& Fluids. 2011, vol. 49, no. 1, p. 110-127..
\end{thebibliography}

\title{Fuzzy Soft Traffic Accident Alert Model}
\tocauthor{GHANSHYAM THAKUR} \author{} \institute{}
\maketitle
\begin{center}
{\large GHANSHYAM THAKUR}\\
MANIT,Bhopal\\
{\tt ghanshyamthakur@gmail.com}
\end{center}

\section*{Abstract}
This paper presents Fuzzy Soft set based model to predict the traffic accidents. In this work I illustrated by a numerical example, showing the spots of accidental Zones. An application of Fuzzy Soft sets approach is used in the traffic accidents. In this research work a Fuzzy Soft Traffic Accident Alert Model is developed to predict the highly accidental spots in, Bhopal city the capital of Madhya Pradesh state, India

\bibliographystyle{plain}
\begin{thebibliography}{10}
\bibitem{Soft set theoryfirst results}
{\sc D. Molodtsov}. {Soft set theory—first results}. Comput. Math. Appl. 37 (1999) 19–31..

\bibitem{ Soft set theory}
{\sc P.K. Maji and R. Biswas and A.R. Roy}. { Soft set theory}. Comput. Math. Appl. 45 (2003) 555–562..

\bibitem{Fuzzy Sets}
{\sc L.A. Zadeh and  }. {Fuzzy Sets}. Inform. Control 8 (1965) 338–353..

\bibitem{A fuzzy soft set theoretic approach to decision making problems}
{\sc A.R. Roy and P.K. Maji}. {A fuzzy soft set theoretic approach to decision making problems}. J. Comput. Appl. Math. 203 (2007) 412–418..

\bibitem{Rough sets}
{\sc Z. Pawlak}. {Rough sets}. Int. J. Inform. Comput. Sci. 11 (1982) 341–356..

\bibitem{ Fuzzy soft sets}
{\sc P.K. Maji and R. Biswas and A.R. Roy}. { Fuzzy soft sets}.  J. Fuzzy Math. 9 (3) (2001) 589–602.

\bibitem{Soft sets and soft groups}
{\sc H. Aktas and N. Cagman}. {Soft sets and soft groups}. Inform. Sci. 177 (2007) 2726–2735.
\end{thebibliography}

\title{A Cyberinfrastructure-Based Computational Environment for Unified Curvilinear Ocean Atmospheric Model  (UCOAM)}
\tocauthor{Mary  Thomas} \author{} \institute{}
\maketitle
\begin{center}
{\large Mary  Thomas}\\
San Diego State University\\
{\tt mthomas@mail.sdsu.edu}
\\ \vspace{4mm}
{\large Jose Castillo}\\
San Diego State University\\
{\tt jcastillo@mail.sdsu.edu}
\end{center}

\section*{Abstract}
The UCOAM model, developed by Abouali and Castillo, is a high-resolution (sub-km) Large Eddie Simulation (LES) CFD model capable of running  ocean and atmospheric simulations [1]. It is the only environmental model in existence today that uses a full, 3D curvilinear coordinate system, which results in increased accuracy, resolution, and reduced times to solution. UCOAM is a petascale model: it requires significant memory ($10^{2}$ arrays with $10^{10}$ elements) ; communication  along all 3 dimensions; and simulations generate TBytes of data. To facilitate simulations, we have developed a computational environment (CE) that includes a parallel, MPI framework for the model, and cyberinfrastructure-based services [2].

For the parallel model, we have designed a modular, parallel framework (PFW), written in F95, that supports staggered grid, CFD applications. The framework includes modules that allow each processing element (PE) to track the execution environment including: communicator groups; local and global scope data decomposition; ghost and halo communication cells; the location and distribution of the staggered grid variables; and utility tools (file I/O, timing, plotting, analysis). 

The CE uses the Cyberinfrastructure Web Application Framework (CyberWeb) to access high-end computational resources [3]. CyberWeb has three core components: Pylons, a Web 2.0  application framework based on emerging Web technologies; (2) Dynamic database, with admin Web pages; (3) the job distribution Web service framework (Jodis). System capabilities include: user accounts; authentication; task execution; data management; visualization; use of existing software and services on resources; dynamic discovery and use of configured services and resources. CyberWeb has been used to develop applications (ocean modeling, thermochemistry), and has been used to access large-scale (e.g. XSEDE) and local compute and archival systems. 

In this paper we discuss our experiences in developing and testing the parallel framework, and we discuss the challenges associated with developing the computational environment to run these simulations.

\bibliographystyle{plain}
\begin{thebibliography}{10}
\bibitem{Unified Curvilinear Ocean Atmosphere Model A vertical velocity case study}
{\sc M. Abouali and J. E. Castillo}. {Unified Curvilinear Ocean Atmosphere Model: A vertical velocity case study}. Math. and Comp. Model., pp. 1–11, Mar. 2011.

\bibitem{Parallelization of the 3D Unified Curvilinear Coastal Ocean Model  Initial Results}
{\sc M. P. Thomas and J. E. Castillo}. {Parallelization of the 3D Unified Curvilinear Coastal Ocean Model : Initial Results}. 12th Intl. Conf. on Comp. Sci. and Its Apps, ICCSA'11 (Salvador de Bahia, Brazil), 2012.

\bibitem{Integrating HPC Resources  Services  and Cyberinfrastructure to Develop Science Applications Using Web Application Frameworks}
{\sc  M. P. Thomas and C. Cheng and S. More and and H. Shah}. {Integrating HPC Resources , Services , and Cyberinfrastructure to Develop Science Applications Using Web Application Frameworks}. Intl. Conf. on Par. and Dist. Proc. Tech. and Apps., PDPTA'12 (Las Vegas, NV), 2012.
\end{thebibliography}

\title{Multiscale Considerations in DNS of Multiphase Flows}
\tocauthor{Gretar Tryggvason} \author{} \institute{}
\maketitle
\begin{center}
{\large Gretar Tryggvason}\\
University of Notre Dame\\
{\tt gtryggva@nd.edu}
\end{center}

\section*{Abstract}
Direct numerical simulations (DNS) of multiphase flows, where every continuum length and time scale is fully resolved for a system large enough to include non-trivial scale interactions, have now advanced to the point where it is possible to study in considerable detail fairly complex problems, such as the flow of hundreds of bubbles, drops, and solid particles. After a brief review of numerical strategies used for DNS of multiphase flows [1] and the unique challenges that a moving phase boundary poses, we discuss such simulations from a multiscale perspective.

We focus on two multiscale aspects: First of all, DNS results can help with the development of closure relations of unresolved processes in simulations of large-scale ``industrial'' systems. As an example we discuss recent results for deformable bubbles in weakly turbulent channel flows. The lift induced lateral migration of the bubbles controls the flow, but the lift is very different for nearly spherical and more deformable bubbles, resulting in different flow features and flow rates. Nevertheless, the results show that the collective motion of many bubbles leads to relatively simple flow structure. Nearly spherical bubbles move to the walls for upflow, resulting in an increase in the mixture density in the middle of the channel, and away from walls in downflow, reducing the average density. For sufficiently high void fraction the process continues until the weight of the mixture in the center is balanced by the imposed pressure gradient. The bubbles pushed to the wall for upflow form a high void fraction wall-layer whose void fraction can be found from the average void fraction in the channel and in the center. Similarly, for downflow the bubbles that move away from the wall leave a bubble free zone whose width can be computed. In both cases the flow rate is determined by what happens in the wall layer, since the velocity in the middle of the channel is constant. For deformable bubbles the lift is generally nearly zero or sometimes in the opposite direction, causing major changes in the flow as the size of the bubbles increases. The results demonstrate the importance of examining large systems with a large range of scale to capture structures that emerge due the collective motion of many bubbles.

The other multi-scale aspect results from the fact that multiphase flows often produce ``features'' such as thin films, filaments, and bubbles or drops that are much smaller than the “dominant” flow scale, such as the average size of the bubbles. Such small-scale features also often arise if we add new physical processes that evolve on different spatial and temporal scales than the fluid flow. Resolving these features often requires excessive grid resolution and can be very expensive. The geometry of these features is, however, often simple, since surface tension effects are strong and inertia effects are relatively small and in isolation these features are therefore sometimes well described by analytical or semi-analytical models. We first discuss briefly the use of a thin film model to capture the flow in a thin film between a drop and a sloping wall [2] and then develop a boundary layer approach to examine mass transfer in liquids [3], where the diffusion of mass in much slower than the diffusion of momentum. The challenges in transferring information between the computational grid used for the flow and the mass transfer away from the bubble boundary and the model are addressed and the validation of the results, by comparisons with highly resolved full simulations and experimental results discussed.

We conclude by discussing some of the current challenges in DNS of multiphase systems and near-term directions [4].

\bibliographystyle{plain}
\begin{thebibliography}{10}
\bibitem{Direct Numerical Simulations of Gas-Liquid Multiphase Flows..}
{\sc G. Tryggvason and R. Scardovelli and S. Zaleski. }. {Direct Numerical Simulations of Gas-Liquid Multiphase Flows..}. Cambridge University Press. 2011.

\bibitem{Multiscale computations of thin films in multiphase flows.}
{\sc S. Thomas and A. Esmaeeli and G. Tryggvason. }. {Multiscale computations of thin films in multiphase flows.}. Int'l J. Multiphase Flow 36 (2010), 71-77..

\bibitem{Multiscale computations of mass transfer from buoyant bubbles.}
{\sc B. Aboulhasanzadeh and S. Thomas and M. Taeibi-Rahni and and G. Tryggvason.}. {Multiscale computations of mass transfer from buoyant bubbles.}. Chemical Engineering Science 75 (2012) 456–467..

\bibitem{Multiscale Considerations in DNS of Multiphase Flows.}
{\sc G. Tryggvason and S. Dabiri and B. Aboulhasanzadeh and J. Lu. }. {Multiscale Considerations in DNS of Multiphase Flows.}. Physics of Fluids. Available Online. .
\end{thebibliography}

\title{Numerical Investigation on the Reshaping of Cylindrical Converging Shocks in Real Gas by Means of Aerodynamic Obstacles}
\tocauthor{Federica Vignati} \author{} \institute{}
\maketitle
\begin{center}
{\large \underline{Federica Vignati}}\\
Politecnico di Milano, Dipartimento di Scienze e Tecnologie Aerospaziali\\
{\tt federica.vignati@mail.polimi.it}
\\ \vspace{4mm}
{\large Alberto Guardone}\\
Politecnico di Milano, Dipartimento di Scienze e Tecnologie Aerospaziali\\
{\tt alberto.guardone@polimi.it}
\end{center}

\section*{Abstract}
This work illustrates the effects of the interaction of aerodynamic obstacles on cylindrical shock waves, converging towards a focus point in a real gas.

Previous studies confirm the possibility of reshaping these shock in case of ideal gas such as, for instance, dilute fluid. Actually, the most interesting case is related to real gases, due to the possibility of observing nonclassical phenomena, such as front instabilities, in correspondence of the critical point, due to the non-linearity of the equations of state.

The explored set of shocks consists of cylindrical implosions caused by an axysymmetrical pressure step imposed on still gas. The generated shock is forced to interact with some obstacles which, if properly located and shaped, turn its shape from cylindrical into prismatic. Such a shaped shock front is more stable, and allows to attain higher values of pressure and temperature in the focus point of the implosion.

The main focus concerns the results of numerical simulations performed using a cartesian Eulerian solver: the physical phenomenon of interest is the  reshaping process due to the impingement of the shock front on solid and fixed obstacles, in case of MDM siloxane, modeled as a Van der Waals gas.

\bibliographystyle{plain}
\begin{thebibliography}{10}
\bibitem{Thermal radiation from a converging shock implosion}
{\sc M. Kjellander and N. Tillmark and N. Apazidis}. {Thermal radiation from a converging shock implosion}. Physics of Fluids, Vol. 22, 2010, pp.1-12.

\bibitem{On focusing of shock waves}
{\sc V. Eliasson}. {On focusing of shock waves}. Ph.D. Dissertation, Department of Mechanics, KTH, Sweden, 2007.

\bibitem{Controlling the form of strong converging shocks by means of disturbances}
{\sc V. Eliasson and N. Apazidis and N. Tillmark}. {Controlling the form of strong converging shocks by means of disturbances}. Shock Waves, Vol 17, 2007, pp. 29-42.

\bibitem{The interaction between shock waves and solid spheres arrays in a shock tube}
{\sc H. Shi and K. Yamamura}. {The interaction between shock waves and solid spheres arrays in a shock tube}. Acta Mechanica Sinica, Vol. 20, No. 3, 2004, pp. 219-227.

\bibitem{Equivalence condition for finite volume/element discretizations in cylindrical coordinates}
{\sc D. De Santis and G. Geraci and A. Guardone}. {Equivalence condition for finite volume/element discretizations in cylindrical coordinates}. V European Conference on Computational Fluid Dynamics ECCOMAS CFD, Lisbon, Portugal, 2010, pp. 1-17.

\bibitem{Regular versus Mach reflection for converging polygonal shocks}
{\sc V. Eliasson and M. Kjellander and N. Apazidis}. {Regular versus Mach reflection for converging polygonal shocks}. Shock Waves, Vol. 17, 2007, pp. 43-50.
\end{thebibliography}

\title{Numerical Simulation of Ductile Fiber-reinforced Cement-based Composite}
\tocauthor{Jan Vorel} \author{} \institute{}
\maketitle
\begin{center}
{\large \underline{Jan Vorel}}\\
Czech Technical University in Prague\\
{\tt jan.vorel@fsv.cvut.cz}
\\ \vspace{4mm}
{\large William Peter Boshoff}\\
Stellenbosch University\\
{\tt bboshoff@sun.ac.za}
\end{center}

\section*{Abstract}
Strain Hardening Cement-based Composite (SHCC) is a type of High Performance Concrete that was developed to overcome the brittleness of conventional concrete. Even though there is no significant compressive strength increase compared to conventional concrete, it exhibits superior behavior in tension. It has been shown to reach a tensile strain capacity of more than 4% during a pseudo strain hardening phase (Li and Wang, 2001; Boshoff and van Zijl, 2007). This pseudo strain hardening is achieved by the formation of fine, closely spaced multiple cracks with crack widths not exceeding 100\,$\mu m$ (Li and Wang, 2001). Due to the tight crack width compared to an ordinary concrete, the SHCC significantly resists the migration of aggressive substances into the bulk material, thereby exhibiting significant durability enhancement. Based on the reviewed properties it is argued that SHCC materials can be used in selected locations of structural members to improve their overall durability performance (Ahmed and Mihashi, 2007). The primary objective of this contribution is to develop a constitutive model or numerical approach that can be utilized to simulate structural components with SHCC under different types of loading conditions.

\bibliographystyle{plain}
\begin{thebibliography}{10}
\bibitem{Tensile strain-hardening behavior of PVA-ECC}
{\sc V. Li and S. Wang}. {Tensile strain-hardening behavior of PVA-ECC}. ACI Materials Journal, 98(6):483-492, 2001.

\bibitem{Time-dependant response of ECC Characterisation of creep and rate dependence}
{\sc W.P. Boshoff and G. van Zijl}. {Time-dependant response of ECC: Characterisation of creep and rate dependence}. Cement and Concrete Research, 37:725-734, 2007.

\bibitem{A review on durability properties of strain hardening fibre reinforced cementitious composites (SHFRCC)}
{\sc S. Ahmed and H. Mihashi}. {A review on durability properties of strain hardening fibre reinforced cementitious composites (SHFRCC)}. Cement \& Concrete Composites, 29(5):365-376, 2007.
\end{thebibliography}

\title{A Fast and Interactive Heat Conduction Simulator on GPUs}
\tocauthor{Zhangping Wei} \author{} \institute{}
\maketitle
\begin{center}
{\large \underline{Zhangping Wei}}\\
The University of Mississippi\\
{\tt zpwei@ncche.olemiss.edu}
\\ \vspace{4mm}
{\large Byunghyun Jang}\\
The University of Mississippi\\
{\tt bjang@cs.olemiss.edu}
\\ \vspace{4mm}
{\large Yafei Jia}\\
The University of Mississippi\\
{\tt jia@ncche.olemiss.edu}
\end{center}

\section*{Abstract}
GPU offers a number of unique benefits to scientific simulation and visualization. Its superior computing capability and interoperability with graphics library are two of those that make GPU the platform of choice. In this paper, we present a fast and interactive heat conduction simulator on GPUs using CUDA and OpenGL. Numerical solution of two-dimensional heat conduction equation is decomposed into two directions to solve tri-diagonal linear systems. A widely used implicit solver, alternating direction implicit (ADI) is accelerated on GPUs using GPU-based parallel tri-diagonal solver, Parallel Cyclic Reduction (PCR). Our implementation improves existing simulations in three aspects. First, we have developed a new thread mapping technique that eliminates the limitations of previous PCR solvers for scalable simulation. Second, our parallel ADI solver outperforms existing solvers through higher underlying hardware utilizations. Third, our design takes advantage of efficient CUDA-OpenGL interoperability to make the simulation interactive in real-time. The proposed interactive visualization simulator can be served as a building block for numerous advanced emergency management systems in engineering practices.

\bibliographystyle{plain}
\begin{thebibliography}{10}
\bibitem{The numerical solution of parabolic and elliptic differential equations}
{\sc D. Peaceman and H. Rachford Jr}. {The numerical solution of parabolic and elliptic differential equations}. Journal of the Society for Industrial \& Applied Mathematics 3 (1) (1955) 28–41.

\bibitem{Fast tridiagonal solvers on the GPU}
{\sc Y. Zhang and J. Cohen and J. Owens}. {Fast tridiagonal solvers on the GPU}. ACM Sigplan Notices 45 (5) (2010) 127–136.

\bibitem{A hybrid method for solving tridiagonal systems on the GPU}
{\sc Y. Zhang and J. Cohen and A. Davidson and J. Owens}. {A hybrid method for solving tridiagonal systems on the GPU}. GPU Computing Gems Jade Edition (2011) 117.

\bibitem{OpenGL programming guide the official guide to learning OpenGL versions 3.0 and 3.1}
{\sc D. Shreiner}. {OpenGL programming guide: the official guide to learning OpenGL, versions 3.0 and 3.1}. Addison-Wesley Professional, 2010, VOL. 1.
\end{thebibliography}

\title{Increasing Flexibility and Reusability of Finite Element Simulations With ViennaX}
\tocauthor{Josef Weinbub} \author{} \institute{}
\maketitle
\begin{center}
{\large \underline{Josef Weinbub}}\\
Institute for Microelectronics, Technische Universität Wien\\
{\tt weinbub@iue.tuwien.ac.at}
\\ \vspace{4mm}
{\large Karl Rupp}\\
MCS Division, Argonne National Laboratory\\
{\tt rupp@mcs.anl.gov}
\\ \vspace{4mm}
{\large Siegfried Selberherr}\\
Institute for Microelectronics, Technische Universität Wien\\
{\tt selberherr@iue.tuwien.ac.at}
\end{center}

\section*{Abstract}
The ever-increasing availability of finite element (FE)-related software tools puts pressure on the software development of simulation applications. Albeit the fact that feature-rich packages such as the deal.II library are available, solely utilizing a static set of tools for an application limits the capabilities of the implementation with respect to flexibility and reusability [1]. For instance, exchanging the linear solver typically requires actual coding and thus implies in-depth knowledge of the code base. A usual way to tackle these challenges are component approaches, such as the Cactus framework [2]. However, the available frameworks either offer an almost prohibitively high entry-level for users or focus on domain-specific applications, like computational fluid dynamics. We introduce our approach, being the plugin execution framework ViennaX [3], which aims for general applicability in scientific computing. Albeit the framework's genericity, in this work we focus on FE-based applications. Engineering simulations such as stress simulations are decoupled into reusable ViennaX components, for instance, assembly and linear solver components. These components are connected via the data communication layer to setup a modular simulation. Due to the ViennaX's run-time system, components can be exchanged without programming efforts, thus increasing flexibility significantly. For instance, different error estimators or mesh adaptation components can be utilized. Additional aspects essential for FE applications like adaptive FE approaches are discussed as well. We depict the utilization of GPUs via the ViennaCL [4] library as well as clusters to underline the applicability for high-performance computing. It is shown that utilizing ViennaX not only results in high flexibility and reusability but also in high performance simulations. 

\bibliographystyle{plain}
\begin{thebibliography}{10}
\bibitem{A Component Architecture for High-Performance Scientific Computing}
{\sc D.E. Bernholdt et al.}. {A Component Architecture for High-Performance Scientific Computing}. Intl J High Perform C 20(2) (2006).

\bibitem{The Cactus Framework and Toolkit}
{\sc T. Goodale et al.}. {The Cactus Framework and Toolkit}. In Proc. of VECPAR (2002).

\bibitem{ViennaX A Parallel Plugin Execution Framework for Scientific Computing}
{\sc J. Weinbub et al.}. {ViennaX: A Parallel Plugin Execution Framework for Scientific Computing}. Eng. Comput. (2013), in press. Web page: http://viennax.sourceforge.net/.

\bibitem{ViennaCL}
{\sc K. Rupp et al.}. {ViennaCL}. Web Page: http://viennacl.sourceforge.net/.
\end{thebibliography}

\title{Error Analysis of Surrogate Models for Improved Uncertainty Quantification}
\tocauthor{Tim Wildey} \author{} \institute{}
\maketitle
\begin{center}
{\large \underline{Tim Wildey}}\\
Sandia National Labs\\
{\tt tmwilde@sandia.gov}
\\ \vspace{4mm}
{\large Troy Butler}\\
Colorado State University\\
{\tt butler.troy.d@gmail.com}
\end{center}

\section*{Abstract}
There is considerable interest in developing efficient and accurate methods to quantify the uncertainty in computational models.  Often a two stage approach to solve this problem is formulated. First, a large number of samples of model parameters or input data are determined in terms of realizations of a stochastic process. Second, the probability distribution is approximately propagated through the computational model to the output or observable data. 

Complicating the task of quantifying the uncertainty reliably is the fact that each output sample is polluted by discretization error in the computational model.  A secondary source of error in using a standard Monte Carlo method is the statistical error resulting from the use of finite sample sizes. Unfortunately, this statistical error can be quite large if only a limited number of samples can be computed.

An alternative approach for propagation of distributions is to construct a surrogate response surface and propagate samples using this surrogate rather than the full computational model. This approach essentially eliminates the statistical error component from the computed distribution since each sample of the surrogate model output has a very low computational cost implying a large number of samples may be taken.  However, each of these samples may be contaminated by deterministic error from the numerical construction of the surrogate approximation. Thus, the trade-off is that statistical error may be neglected at the cost of possibly large deterministic sources of error.  Fortunately, recent work has shown that these deterministic source of error can be efficiently estimated using a surrogate to the adjoint model.

We demonstrate these error estimates for linear functionals of a solution to a parameterized linear system of equations.  We also use this error estimate to define an improved linear functional and we show that this improved functional converges at a much faster rate than the original linear functional.  Then, we describe the extension of these error estimates and the improved linear functional to numerical approximations of stochastic differential equations.  Finally, we show how these error estimates can be used to locally adapt the stochastic approximation to efficiently approximate probabilistic quantities.

\bibliographystyle{plain}
\begin{thebibliography}{10}
\bibitem{A posteriori error analysis of parameterized linear systems using spectral methods}
{\sc T. Butler and P. Constantine and and T. Wildey}. {A posteriori error analysis of parameterized linear systems using spectral methods}.  SIAM. J. Matrix Anal. Appl., 33 (2012), pp. 195–209.

\bibitem{A posteriori error analysis of stochastic spectral methods}
{\sc T. Butler and C. Dawson and and T. Wildey}. {A posteriori error analysis of stochastic spectral methods}.  SIAM J. Sci. Comput., 33 (2011), pp. 1267–1291.
\end{thebibliography}

\title{PyFR: An Open Source Python Framework for High-Order CFD on Many-Core Platforms}
\tocauthor{Freddie Witherden} \author{} \institute{}
\maketitle
\begin{center}
{\large Freddie Witherden}\\
Imperial College London\\
{\tt freddie.witherden08@imperial.ac.uk}
\\ \vspace{4mm}
{\large Antony Farrington}\\
Imperial College London\\
{\tt antony.farrington07@imperial.ac.uk}
\\ \vspace{4mm}
{\large Peter Vincent}\\
Imperial College London\\
{\tt p.vincent@imperial.ac.uk}
\end{center}

\section*{Abstract}
Theoretical studies and numerical experiments suggest that high-order methods for unstructured grids can solve hitherto intractable fluid flow problems in the vicinity of complex geometrical configurations. The \emph{flux reconstruction} (FR) approach, developed by Huynh, is simple yet efficient and particularly amenable to the requirements of modern hardware architectures---including GPUs. Moreover, using the FR approach it is possible to unify various popular high-order methods such as nodal discontinuous Galerkin (DG) and spectral difference schemes.

The presented framework, PyFR, solves the compressible Euler and Navier stokes equations in 3D using the FR approach. It combines symbolic manipulation with an innovative backend architecture based around \emph{run-time code generation} in order to target a variety of many-core computing platforms. The specific combination of a high-level scripting language with computer algebra facilitates an almost direct translation from the mathematical formalism to an efficient implementation. Finally, deferred code generation makes it possible to apply a broad spectrum of problem- and domain-specific optimizations that are simply not possible in statically compiled codes.

We will present results demonstrating performance and scalability on both CPU and GPU clusters with in excess of 100 compute nodes.


\bibliographystyle{plain}
\begin{thebibliography}{10}
\bibitem{A Flux Reconstruction Approach to High-Order Schemes Including Discontinuous Galerkin Methods}
{\sc H. T. Huynh}. {A Flux Reconstruction Approach to High-Order Schemes Including Discontinuous Galerkin Methods}. AIAA Conference Paper, 2007-4079, 2007.

\bibitem{Facilitating the Adoption of Unstructured High-Order Methods Amongst a Wider Community of Fluid Dynamicists}
{\sc P. E. Vincent and A. Jameson}. {Facilitating the Adoption of Unstructured High-Order Methods Amongst a Wider Community of Fluid Dynamicists}. Math. Mod. Nat. Phenom.  6(3) 2011, 97-140.
\end{thebibliography}

\title{Interoperable Executive Library for the Simulation of Biomedical Processes}
\tocauthor{Kwai Wong, Andrew Kail, Xiaopeng  Zhao} \author{} \institute{}
\maketitle
\begin{center}
{\large Kwai Wong, Andrew Kail, Xiaopeng  Zhao}\\
University of Tennessee\\
{\tt kwong@utk.edu}
\end{center}

\section*{Abstract}
The complexity of simulating any system-wise process in the human body represents a daunting challenge to unravel. Often in many types of these biomedical processes the interplay of various biological, chemical, and physical models go beyond the reach of a single set of computer code. In this paper we present an Interoperable Executive Library (IEL) that has been designed to run, in parallel, a collection of multi-component simulations focusing in simulating a complex biomedical process. 
 
The IEL[1] is a light-weight integrator comprised of three major components.  It is responsible for managing the distribution of data and memory, coordinates communication among parallel processes, and directs execution of a set of loosely coupled numerical and physics tasks. The three components of the IEL are the configuration file, the executive and the communicator library (COMMLIB). The configuration file specifies the shared boundary conditions for the interacting computing modules. The Executive reads in the configuration file and sets up the computing modules and initializes the communications using COMMLIB. COMMLIB is a series of functions, built on top of the MPI protocols, implemented in the IEL to transfer shared boundary data. The IEL also incorporates a number of sparse and dense parallel solvers from the Trilinos, Magma, ScaLAPACK, and PETSc libraries. In addition, IEL has also built in a set of preprocessing and postprocessing units based on the I/O format of HDF5 in conjunction with CUBIT and Paraview. 
 
In this paper we present the simulation of the heart utilizing the IEL. The heart simulation uses computing modules of electrical signal propagation, structural deformation, and fluid-structure interaction [2]. For the electrical module the Beeler-Reuter and Fox-McHarg-Gilmour model coupled with the reaction-diffusion equations are used. The fluid module utilizes INS formulation while the structural model is formulated using Kirkchhoff stress equilibrium. CUBIT is used to generate the mesh and a preprocessing unit is used to partition the input data for parallel processing. The computing modules have been formulated using the Finite Element Method. The results of the heart simulation obtained from running on Kraken (CRAY XT5 at NICS) will be analyzed and shown.


\bibliographystyle{plain}
\begin{thebibliography}{10}
\bibitem{An Interoperable Executive Library for Multi-physics Simulation }
{\sc R. Cortese and S. Mandry and K. Wong and K. Seymour}. {An Interoperable Executive Library for Multi-physics Simulation, }. poster presentation, XSEDE12, 2012.

\bibitem{A Fully Coupled Model for Electromechanics of the Heart}
{\sc H. Xia and K. Wong and X. Zhao}. {A Fully Coupled Model for Electromechanics of the Heart}. Computational and Mathematical Methods in Medicines, vol. 2012, Article ID 927279.
\end{thebibliography}

\title{Solving a Large Scale Thermal Radiosity Problem on GPU-Based Parallel Computer}
\tocauthor{Kwai Wong, Shiquan  Su} \author{} \institute{}
\maketitle
\begin{center}
{\large Kwai Wong, Shiquan  Su}\\
kwong@utk.edu\\
{\tt University of Tennessee, Knoxville}
\\ \vspace{4mm}
{\large Edurado  D'Azevedo}\\
Oak Ridge National Laboratory\\
{\tt e6d@ornl.gov}
\\ \vspace{4mm}
{\large Zhiang Hu}\\
Chinese University of Hong Kong\\
{\tt hza8816415@gmail.com}
\end{center}

\section*{Abstract}
The amount of average radiating energy on a set of Lambertian surfaces can be obtained by solving a linear system of the radiosity equations. The equation is simple to formulate but is challenging to solve when the number of Lambertian surfaces associated with an application becomes large. For an unsteady thermal problem in which the mesh configuration remains unchanged, the LU or LLT factor of the radiosity matrix can be computed once and be reused subsequently in every time step. Although direct solver uses a lot more memory, they are more efficient and the basic kernels are readily available on parallel supercomputers.
The radiosity matrix requires the computation of the view factors. Based on Watson’s serial view factor algorithms, we have extended his View3d [1] code to compute the view factors on a parallel computer equipped with GPU accelerators. A parallel Cholesky decomposition solver, based on a hybrid CPU/GPU ScaLAPACK library [2], is built to factor and solve the matrix. Coupling with the direct radiosity solver, the energy equation is formulated in finite element method and solved using sparse iterative solvers from Trilinos. In this paper, we will present the methodologies solving the thermal radiosity problem on a large scale hybrid parallel computer with GPUs. Results of the parallel procedure obtained on keeneland at NICS are shown and examined.


\bibliographystyle{plain}
\begin{thebibliography}{10}
\bibitem{View3d calculation}
{\sc J. Watson}. {View3d calculation}. 2001.

\bibitem{LU Decomposituon on GPU}
{\sc E. D'Azevedo and J. Hill}. {LU Decomposituon on GPU}. 2010.
\end{thebibliography}

\title{Finite Element Analysis and Simulation of Mass Concrete Construction}
\tocauthor{Juncai Xu} \author{} \institute{}
\maketitle
\begin{center}
{\large \underline{Juncai Xu}}\\
hohai University\\
{\tt juncaixu@yahoo.com.cn}
\\ \vspace{4mm}
{\large Zhenzhong Shen}\\
hohai University\\
{\tt zhzhshen@hhu.edu.cn}
\\ \vspace{4mm}
{\large Guenter Hofstetter}\\
innsbruck university\\
{\tt guenter.hofstetter@uibk.ac.at}
\end{center}

\section*{Abstract}
Mass concrete structures are important in structural civil engineering. One of the biggest engineering problems is the cracking of mass concrete. Therefore, control analysis for the cracking of mass concrete is necessary. The concrete heat release model, the mechanical model of the concrete, the process of temperature control for the pipe model, and other such models should be considered in mass concrete crack control analysis. A certain quantity of heat is produced after concrete is poured and hydration reaction occurs, Feedback for the analysis of the coefficient of adiabatic temperature as well as other coefficients should also be considered to obtain an accurate reading of the adiabatic heating model parameters. The amount of liberated heat by the concrete can be obtained based on the heat release model because of the inversion of the parameters. A differential evolution algorithm is adopted in the inverse analysis to obtain better results.  The temperature range varies between the inner and outer concrete because of the heat release in the concrete and the changes in the external environment. The difference in temperature facilitates the formation of temperature stress. A certain number of water pipes are usually installed to prevent the temperature difference and to eliminate the temperature effects, both of which reduce temperature stress. Water pipes have a cooling effect on the interior layout of concrete. In this paper, the water pipe equivalent algorithm is used to achieve better cooling performance for mass concrete.  This paper adopts an elastic creep model because the construction of concrete also tends to creep with the load duration. To reduce the workloads for exploitation further, this model is based on Abaqus platform for secondary development. The Python language is used to control the proceeding analysis. Fortran subroutines are applied to the aforementioned model. Finally, a second development of the code is used to find a reasonable solution to the problem with one actual aqueduct in hydraulic engineering based on the actual computation results.

\bibliographystyle{plain}
\begin{thebibliography}{10}
\bibitem{A numerical model for predicting the thermomechanical conditions during hydration of early age concrete}
{\sc J. Hattel}. {A numerical model for predicting the thermomechanical conditions during hydration of early age concrete}. A.Mathe.model.27(2003) 1-26.

\bibitem{Thermal stress and temperature control of mass concrete}
{\sc Z. Bofang}. {Thermal stress and temperature control of mass concrete}. China electric power press.(1999) 379-409.

\bibitem{Numerical analysis of the thermal active restrained shrinkage ring test to study the early age behavior of massive concrete structures}
{\sc M. Briffaut}. {Numerical analysis of the thermal active restrained shrinkage ring test to study the early age behavior of massive concrete structures}. Engineering Structures.33(2011)1390–1401.
\end{thebibliography}

\title{Creeping Motion of an Assemblage of Porous Cylindrical Shells}
\tocauthor{Pramod Kumar Yadav} \author{} \institute{}
\maketitle
\begin{center}
{\large \underline{Pramod Kumar Yadav}}\\
Motilal Nehru National Institute of Technology Allahabad, India\\
{\tt pramod547@gmail.com}
\\ \vspace{4mm}
{\large Manoj Kumar Yadav}\\
NIT, Patna, India\\
{\tt ayushmanoj2007@gmail.com}
\end{center}

\section*{Abstract}
The solution of the problem of symmetrical creeping flow of an assemblage of porous cylindrical shells in a cylindrical cavity with Happel boundary  condition is investigated. The effect of the hydrodynamic interaction among the porous shell particles is taken into account by employing a cell-model representation. In the limit of small Reynolds number, the Stokes and Brinkman equations are solved for the flow field around a single particle in a unit cell. The hydrodynamic drag force acting on each porous cylindrical particle in a cell and permeability of membrane built up by cylindrical particles with a porous shell are evaluated. Some previous results for hydrodynamic drag force and dimensionless hydrodynamic permeability have been verified. Variation of the drag coefficient and dimensionless hydrodynamic permeability with permeability parameter , particle volume fraction   has been studied and some new results are reported. The flow patterns through the regions have been analyzed by stream lines. Effect of particle volume fraction  and permeability parameter on flow pattern is also discussed.

\bibliographystyle{plain}
\begin{thebibliography}{10}
\bibitem{Viscous flow in multiparticle systems slow viscous flow through a mass of particles}
{\sc S. Uchida}. {Viscous flow in multiparticle systems: slow viscous flow through a mass of particles}. Ind. Engng. Chem. 46 (1994)1194-1195.

\bibitem{Drag of porous cylinders in a viscous fluid at low Reynolds Numbers}
{\sc I. B. Stechkina}. {Drag of porous cylinders in a viscous fluid at low Reynolds Numbers}. Fluid Dynamics 14 (6) (1979)  912-915.

\bibitem{Flow past a circular cylinder embedded in a porous medium based on the Brinkman model}
{\sc I. Pop and P. Cheng}. {Flow past a circular cylinder embedded in a porous medium based on the Brinkman model}. Int. J. Engng. Sci. 30(2) (1992)  257-262.

\bibitem{Creeping Flow Past a Porous Spherical Shell}
{\sc Y. Qin and P. N. Kaloni}. {Creeping Flow Past a Porous Spherical Shell}. ZAMM 77(2) (1993) 77- 84.

\bibitem{Mathematical modeling of the hydrodynamic permeability of a membrane built up from porous particles with a permeable shell}
{\sc A. N. Filippov and S. I. Vasin and V. M. Starov}. {Mathematical modeling of the hydrodynamic permeability of a membrane built up from porous particles with a permeable shell}. Colloids  Surfaces(2006) 272-278.

\bibitem{Flow around nano spheres and nano cylinders}
{\sc M. T. Matthews and J.M. Hil}. {Flow around nano spheres and nano cylinders}. Quart J Mech Appl Math 59(2) (2006) 191-210.

\bibitem{Slow viscous flow through a membrane built up from porous cylindrical particles with an impermeable core}
{\sc S. Deo and P.K. Yadav and A. Tiwari}. {Slow viscous flow through a membrane built up from porous cylindrical particles with an impermeable core}. Appl. Math. Model. 34 (2010) 1329–1343.

\bibitem{Hydrodynamic permeability of aggregates of porous particles with an impermeable core}
{\sc S. Deo and A. N. Filippov and A. Tiwari and S. I. VasinI and V. M. Starov}. {Hydrodynamic permeability of aggregates of porous particles with an impermeable core}. Adv. Colloid Interface Sci 164 (2011) 21-27.
\end{thebibliography}

\title{A Fully Coupled Multiphase Flow and Geomechanics Solver for Highly Heterogeneous Porous Media.}
\tocauthor{Daegil  Yang} \author{} \institute{}
\maketitle
\begin{center}
{\large \underline{Daegil  Yang}}\\
Texas A and M University\\
{\tt daegil.yang@pe.tamu.edu}
\\ \vspace{4mm}
{\large George  Moridis}\\
Lawrence Berkeley National Laboratory \\
{\tt gjmoridis@lbl.gov}
\\ \vspace{4mm}
{\large Thomas  Blasingame}\\
Texas A and M University\\
{\tt t-blasingame@pe.tamu.edu}
\end{center}

\section*{Abstract}
This paper introduces a fully coupled multiphase flow and geomechanics solver that can be applied to modeling highly heterogeneous porous media. Multiphase flow in deformable porous media is a multiphysics problem that considers the flow physics and rock physics simultaneously. To model this problem, the multiphase flow equations and equilibrium equation must be tightly coupled. Conventional finite element modeling of coupled flow and geomechanics does not conserve mass locally since it uses continuous basis functions. Mixed finite element discretization that satisfies local mass conservation of the flow equation can be a good solution for this problem. In addition, the stabilized finite element method for discretizing the saturation equation minimizes numerical diffusion and provides better resolution of saturation solution. 
In this work, we developed a coupled multiphase flow and geomechanics solver that solves fully coupled governing equations, namely pressure, velocity, saturation, and displacement equations. The solver can deal with full tensor permeability and elastic moduli for modeling a highly heterogeneous reservoir system. 
The results of the numerical experiments are very encouraging.  We used the solver to simulate a reservoir system that has very heterogeneous permeability and elastic moduli fields and found that the numerical solution captures the complex multiphysics of the system.  In addition, we obtained higher resolution of saturation solution than with the conventional finite volume discretization. This would help us make better estimate of numerical solution of complex multiphysics problems. 


\bibliographystyle{plain}
\begin{thebibliography}{10}
\bibitem{From Single-Phase to Compositional Flow Applicability of Mixed Finite Elements}
{\sc Z. Chen and R.E. Ewing}. {From Single-Phase to Compositional Flow: Applicability of Mixed Finite Elements}. Transport in Porous Media 27 (1997) 225-242.

\bibitem{Multi-level adaptive simulation of transient two-phase flow in heterogeneous porous media}
{\sc C. Chueh and M. Secanell and W. Bangerth and N. Djilali}. {Multi-level adaptive simulation of transient two-phase flow in heterogeneous porous media}. Computers and Fluids 39 (2010) 1585-1596.

\bibitem{Stabilized finite element methods for coupled geomechanics and multiphase flow}
{\sc J. Wan}. {Stabilized finite element methods for coupled geomechanics and multiphase flow}. PhD thesis (2002) Stanford University.

\bibitem{A locally conservative finite element framework for the simulation of coupled flow and reservoir geomechanics}
{\sc B. Jha and R. Juans}. {A locally conservative finite element framework for the simulation of coupled flow and reservoir geomechanics}. Acta Geotech. 2 (2007) 139-153.
\end{thebibliography}

\title{Generalized Polynomial Chaos: Approximation Through Change of Measure}
\tocauthor{Xiu Yang} \author{} \institute{}
\maketitle
\begin{center}
{\large \underline{Xiu Yang}}\\
Brown University\\
{\tt xiu\_yang@brown.edu}
\\ \vspace{4mm}
{\large Xiaoliang Wan}\\
Louisiana State University\\
{\tt xlwan@math.lsu.edu}
\\ \vspace{4mm}
{\large George  Karniadakis}\\
Brown University\\
{\tt george\_karniadakis@brown.edu}
\end{center}

\section*{Abstract}
In the past decade, generalized polynomial chaos (gPC) has been successfully applied to uncertainty quantification (UQ) problems. The gPC basis functions are based on the uncertainties in the input, boundary conditions, parameters, etc. For many problems, this may not be a optimal choice, hence, extension methods like multi-element method and dynamically orthogonal basis are proposed to improve the efficiency by changing the basis functions adaptively. In our work, we consider the basic problem of approximating a nonlinear random function with gPC expansion. Due to the nonlinearity, the random function may need to be approximated by a high order polynomial expansion for a certain accuracy. We explore the possibility to reduce the number of basis modes by projecting the target function onto a set of orthogonal polynomials with respect to a different probability measure. This new probability measure is represented by finite element and is obtained by solving an optimization problem. The expansion of the target function with the new polynomials is more economical. We apply this method to solve stochastic partial differential equations (SPDE) and we obtain accurate results with fewer terms in the gPC expansion, hence it is effective especially when the SPDE solver is costly.

\bibliographystyle{plain}
\begin{thebibliography}{10}
\bibitem{The Wiener-Askey polynomial chaos for stochastic differential equations}
{\sc D. Xiu and G. Karniadakis}. {The Wiener-Askey polynomial chaos for stochastic differential equations}. SIAM J. Sci. Comput..

\bibitem{An adaptive multi-element generalized polynomial chaos method for stochastic differential equations}
{\sc X. Wan and G. Karniadakis}. {An adaptive multi-element generalized polynomial chaos method for stochastic differential equations}. J. Comput. Phys.
\end{thebibliography}

\title{Anisotropic Simplex Mesh Adaptation by Riemannian Metric Optimization: Application to Parametrized Equations}
\tocauthor{Masayuki  Yano} \author{} \institute{}
\maketitle
\begin{center}
{\large Masayuki  Yano}\\
Massachusetts Institute of Technology\\
{\tt myano@mit.edu}
\\ \vspace{4mm}
{\large David Darmofal}\\
Massachusetts Institute of Technology\\
{\tt darmofal@mit.edu}
\end{center}

\section*{Abstract}
We present an application of a versatile anisotropic simplex mesh adaptation framework to parametrized partial differential equations (PDEs). Combined with the high-order discontinuous Galerkin discretization and the dual-weighted residual (DWR) error estimate, our adaptive framework iterates toward a mesh that minimizes the output error for a given number of degrees of freedom by considering a continuous
optimization problem of the Riemannian metric field representing the anisotropic simplex mesh [4] (cf. [2]). The adaptation procedure consists of three key steps: sampling of the anisotropic error behavior using element-wise local solves --- a simplex extension of the tensor-element sampling strategy [1]; synthesis of the local errors based on an affine-invariant representation [3] of the metric-error relation; and optimization of the  metric-error model to generate the optimal mesh. The versatile adaptation framework handles any discretization order, naturally incorporates both 
the primal and adjoint solution behaviors, and robustly treats irregular features. We demonstrate the effectiveness of the method for single-design-point simulations.
\\
We then apply the versatile mesh adaptation framework above in the context of
parametrized PDEs. In particular, we control the 
discretization error associated with the polynomial chaos expansion or
reduced basis approximation of parametrized PDEs. To this end, we construct spatial error indicators that account for
discretization errors for a range of parametric configurations, and iterates toward a single universal mesh that works well over the
parameter range. We apply the method to the two-dimensional Reynolds-averaged Navier-Stokes equations. The numerical results demonstrate the ability of --- as well as limitations of --- the single universal mesh to effectively represent the
relevant features of the solution field over a range of parameters. A detailed comparison of the solutions obtained on the universal mesh
and single-design-point optimized meshes are made.

\bibliographystyle{plain}
\begin{thebibliography}{10}
\bibitem{Adaptivity   and a posteriori error estimation for DG methods on anisotropic   meshes}
{\sc P. Houston and E. H. Georgoulis and E. Hall}. {Adaptivity   and a posteriori error estimation for {DG} methods on anisotropic   meshes}. International Conference on Boundary and Interior Layers,   2006.

\bibitem{Continuous mesh model and   well-posed continuous interpolation error estimation}
{\sc  A. Loseille and F. Alauzet}. {Continuous mesh model and   well-posed continuous interpolation error estimation}. INRIA,   RR-6846, 2009..

\bibitem{A Riemannian   framework for tensor computing}
{\sc X. Pennec and P. Fill and N. Ayache}. {A Riemannian   framework for tensor computing}. Int. J. Comput. Vision, 66, 41--66, 2006..

\bibitem{An optimization framework   for anisotropic simplex mesh adaptation}
{\sc  M. Yano and D. L. Darmofal}. {An optimization framework   for anisotropic simplex mesh adaptation}. J. Comp. Phys., 231,   7626--7649, 2012.
\end{thebibliography}

\title{Hybrid Power Analysis Attack in Frequency Domain for Security Modules}
\tocauthor{Masaya Yoshikawa} \author{} \institute{}
\maketitle
\begin{center}
{\large Masaya Yoshikawa}\\
Meijo university\\
{\tt dpa\_cpa@yahoo.co.jp}
\end{center}

\section*{Abstract}
Recently, the number of electronic devices handling confidential information has increased. In these devices, encryption is applied to protect the confidential information. In the encryption, encryption standards, such as the data encryption standard (DES) and advanced encryption standard (AES), are widely used. It is mathematically proven that brute force attack is the only successful attack method against these encryption standards. Therefore, these encryption standards are secured by calculation time.
However, when these encryption standards are incorporated into electronic devices as large scale integration (LSI), it is possible to estimate confidential information, such as cipher keys, by analyzing the circuit's operation using information, such as power consumption or electromagnetic waves that are leaked during the LSI's operation. Such attacks are generally called side-channel attacks. Especially, power analysis attack of side-channel attacks is very risky because it cracks security codes by statistically processing the difference in power consumption and can easily be attacked. Therefore, it is important to verify  the efficiency of the resistance evaluation of cryptographic LSI from a view point of security.
In the present study, to improve the efficiency of the resistance evaluation of cryptographic LSI, a new power analysis attack method was proposed against the AES. The proposed method has the ability to shorten processing time and to improve attack accuracy by attacking only a specified frequency band in the frequency domain. 
Since the power spectrum is the frequency axis, the phase component is not necessary to consider. Therefore, even though the power analysis is difficult to perform in the conventional time domain due to a position aberration error on the time axis, the proposed algorithm can be performed in the frequency domain.
Experimental results proved the validity of the proposed method.

\bibliographystyle{plain}
\begin{thebibliography}{10}
\bibitem{Multi-rounds masking method against DPA attacks}
{\sc M.Yoshikawa  and M.Sugiyam }. {Multi-rounds masking method against DPA attacks}. Proc. of IEEE International Conference on Information Reuse and Integration, pp.100-103, 2011.

\bibitem{Evaluation technique for cryptography circuits with measures against power analysis attacks}
{\sc M.Yoshikawa  and T.Asai }. {Evaluation technique for cryptography circuits with measures against power analysis attacks}. Lecture Notes in Information Engineering, vol.25, pp.76-81, 2012.

\bibitem{Early Feedback on Side-Channel Risks with Accelerated Toggle-Counting}
{\sc Z.Chen  and P.Schaumont }. {Early Feedback on Side-Channel Risks with Accelerated Toggle-Counting}. Proc. of IEEE Workshop on Hardware Oriented Security and Trust, pp.90-95, 2009.

\bibitem{Improved algebraic side-channel attack on AES}
{\sc M.S.E.Mohamed  and et al.}. {Improved algebraic side-channel attack on AES}. Proc. of IEEE Workshop on Hardware Oriented Security and Trust, pp.146-151, 2012.
\end{thebibliography}

\title{A Penalty Method for Coupling Fluid-structure Interactions}
\tocauthor{Yue Yu} \author{} \institute{}
\maketitle
\begin{center}
{\large \underline{Yue Yu}}\\
Brown University\\
{\tt yue\_yu\_1@brown.edu}
\\ \vspace{4mm}
{\large Johnny Guzman}\\
Brown University\\
{\tt Johnny\_Guzman@Brown.edu}
\\ \vspace{4mm}
{\large George Karniadakis}\\
Brown University\\
{\tt george\_karniadakis@brown.edu}
\end{center}

\section*{Abstract}
In [1], a new stabilized explicit coupling partitioned scheme has been proposed based on Nitsche's method, for the fluid-structure interaction problem. In this work, we further extend it to the case where the pressure and velocity are decoupled, i.e. the fluid part is solved with projection methods. Specifically, proper penalty terms are applied on the displacement, pressure and velocity solvers separately, to control the variations at the interface. Using energy stability analysis, it can be shown that the scheme is stable independent of the fluid-structure density ratio. All the implementations are done with the spectral element method as explained in [2]. Numerical examples are provided to show that although the penalty terms will degrade the time accuracy of the scheme by half order, optimal accuracy can be recovered by performing defect-correction subiterations.

\bibliographystyle{plain}
\begin{thebibliography}{10}
\bibitem{Stabilization of explicit coupling in fluidstructure interaction involving fluid incompressibility}
{\sc E. Burman and M. A. Fernandez}. {Stabilization of explicit coupling in fluid–structure interaction involving fluid incompressibility}. Computer Methods in Applied Mechanics and Engineering, 198(2009) 766-784.

\bibitem{A convergence study of a new partitioned fluidstructure interaction algorithm based on fictitious mass and damping}
{\sc H. Baek and G. E. Karniadakis}. {A convergence study of a new partitioned fluid–structure interaction algorithm based on fictitious mass and damping}. Journal of Computational Physics, 231 (2011) 629-652.
\end{thebibliography}

\title{Fractional Spectral Element Method}
\tocauthor{Mohsen Zayernouri} \author{} \institute{}
\maketitle
\begin{center}
{\large \underline{Mohsen Zayernouri}}\\
Division of Applied Mathematics, Brown University\\
{\tt mohsen\_zayernouri@brown.edu}
\\ \vspace{4mm}
{\large George  Em Karniadakis}\\
Division of Applied Mathematics, Brown University\\
{\tt george\_karniadakis@brown.edu}
\end{center}

\section*{Abstract}
We develop an efficient spectral element method for time- and space-fractional partial differential equations (FPDEs). To this end, we develop a new theory for proper basis functions to utilize in approximating the fractional differential operators. Particularly, a novel discontinuous Galerkin \textit{hp}-finite element method is developed for general linear FPDEs. We examine our scheme for fractional advection and fractional diffusion equations. The fractional spectral element method yields the theoretical exponential convergence in the $p$-refinement in addition to the algebraic convergence in the $h$-refinement. 

\bibliographystyle{plain}
\begin{thebibliography}{10}
\bibitem{High-order finite element methods for time-fractional partial differential equations}
{\sc Yingjun Jiang and Jingtang Ma}. {High-order finite element methods for time-fractional partial differential equations}. Journal of Computational and Applied Mathematics, Volume 235, Issue 11, 1 April 2011, Pages 3285–3290.

\bibitem{Finite Element Solutions for the Space Fractional Diffusion Equation with a Nonlinear Source Term}
{\sc Y. J. Choi and S. K. Chung}. {Finite Element Solutions for the Space Fractional Diffusion Equation with a Nonlinear Source Term}. Hindawi Publishing Corporation Abstract and Applied Analysis, Volume 2012, Article ID 596184, 25 pages doi:10.1155/2012/596184.
\end{thebibliography}

\title{Parallel Solvers for Large Scale Sparse Linear Systems}
\tocauthor{Wu Zhang} \author{} \institute{}
\maketitle
\begin{center}
{\large Wu Zhang}\\
Shanghai University\\
{\tt wzhang@shu.edu.cn}
\end{center}

\section*{Abstract}
The distributed shared memory system has become the development trend of High Performance Computers, which share memory on each node and distributed memory among nodes. In order to take full advantage of this multi-level architecture, we first design a multi-granularity parallel algorithm (MPI/OpenMP) to solve large scale banded linear systems. The experimental results show that: as the band width and nodes increase, MPI/OpenMP hybrid parallel model has a better speedup and scalability. Secondly, we propose a hierarchical parallel algorithm based on multi-granularity MPI/OpenMP hybrid programming model because global communication plays an important role in the impact of algorithm scalability in parallel computing. The algorithm changes global communication into local communication, eliminates the bottleneck problem of global communication and improves the scalability of solving large banded linear system. A higher efficient parallel GaBP algorithms based on GPU architecture to solve symmetric diagonally dominant sparse matrix is developed finally in this paper.

\bibliographystyle{plain}
\begin{thebibliography}{10}
\bibitem{Preconditioning Techniques for Large Linear Systems A Survey}
{\sc Benzi  and M }. {Preconditioning Techniques for Large Linear Systems: A Survey}. J. Comput Phys. 182(2): 418-477.
\end{thebibliography}

\title{First-Principles Electronic Structure Calculations Based on Finite Element Discretizations}
\tocauthor{Aihui Zhou} \author{} \institute{}
\maketitle
\begin{center}
{\large Aihui Zhou}\\
Institute of Computational Mathematics and Scientific/Engineering Computing, Academy of Mathematics and Systems Science, Chinese Academy of Sciences\\
{\tt azhou@lsec.cc.ac.cn}
\end{center}

\section*{Abstract}
In this presentation, we will talk about
electronic structure calculations based on parallel adaptive and local finite
element discretizations. We will report several numerical
experiments in quantum chemistry and materials science, which show that our adaptive and
local finite element approaches are efficient. This presentation is
based on some joint works with X. Dai, J. Fang, X. Gao, X. Gong, L. Shen, Z. Yang,
D. Zhang, and J. Zhu.

\bibliographystyle{plain}
\begin{thebibliography}{10}
\bibitem{Finite volume discretizations for eigenvalue problems with applications to electronic structure calculations}
{\sc X. Dai and X. Gong and Z. Yang and D. Zhang and and A. Zhou}. {Finite volume discretizations for eigenvalue problems with applications to electronic structure calculations}. Multiscale Model. Simul. 9(2011) 208-240.
\end{thebibliography}

\newpage
    \noindent
    {\bf M. Abdollahzadeh}\\
    University of Beira Interior, Department of Electromechanical Engineering, Center for Aerospace Sciences and Technology, Portugal, Covilhã\\
        \noindent
    {\bf P. J. Oliveira}\\
    University of Beira Interior, Department of Electromechanical Engineering, Portugal, Covilhã\\
        \noindent
    {\bf J. Páscoa}\\
    University of Beira Interior, Department of Electromechanical Engineering, Center for Aerospace Sciences and Technology, Portugal, Covilhã\\
        \noindent
    {\bf Gabriel Aguilera}\\
    Universidad de M\'alaga\\
        \noindent
    {\bf Jos\'e Carlos Campos}\\
    Universidad de M\'alaga\\
        \noindent
    {\bf Jos\'e Luis Gal\'an}\\
    Universidad de M\'alaga\\
        \noindent
    {\bf Pedro Rodr\'{\i}guez}\\
    Universidad de M\'alaga\\
        \noindent
    {\bf Nissrine Akkari}\\
    University of La Rochelle, LaSIE (Laboratoire des Sciences de l'Ingénieur pour l'Environnement)\\
        \noindent
    {\bf Aziz Hamdouni}\\
    University of La Rochelle, LaSIE (Laboratoire des Sciences de l'Ingénieur pour l'environnement)\\
        \noindent
    {\bf Mustapha Jazar}\\
    University of Lebanon, LaMA (Laboratoire de Mathématiques et Applications)\\
        \noindent
    {\bf Erwan Liberge}\\
    University of La Rochelle, LaSIE (Laboratoire des Sciences de l'Ingénieur pour l'environnement)\\
        \noindent
    {\bf Khaled Alhussan}\\
    king Abdulaziz city for science and technology\\
        \noindent
    {\bf Ibraheem Alolyan}\\
    King Saud University\\
        \noindent
    {\bf Franck Assous}\\
    Ariel University Center, Israel\\
        \noindent
    {\bf Joel Chaskalovic}\\
    d'Alembert,  University Pierre and Marie Curie, Paris, France\\
        \noindent
    {\bf Melike Aydogan}\\
    Isik University\\
        \noindent
    {\bf Melike Aydogan}\\
    Isik University\\
        \noindent
    {\bf Yasar Polatoglu}\\
    Istanbul Kultur University\\
        \noindent
    {\bf Christopher Basting}\\
    University Erlangen-Nuremberg\\
        \noindent
    {\bf Dmitri Kuzmin}\\
    University Erlangen-Nuremberg\\
        \noindent
    {\bf Steffen Basting}\\
    University Erlangen-Nuremberg\\
        \noindent
    {\bf Melanie Bittl}\\
    University Erlangen-Nuremberg\\
        \noindent
    {\bf Dmitri Kuzmin}\\
    University Erlangen-Nuremberg\\
        \noindent
    {\bf Markus Blatt}\\
    Dr. Markus Blatt - HPC-Simulation-Software \& Services\\
        \noindent
    {\bf Paola Brambilla}\\
    Politecnico di Milano\\
        \noindent
    {\bf Andrea Cristina}\\
    Politecnico di Milano\\
        \noindent
    {\bf Alberto Guardone}\\
    Politecnico di Milano\\
        \noindent
    {\bf Moritz Braun}\\
    University of South Africa\\
        \noindent
    {\bf Jesse Chan}\\
    Institute for Computational Engineering and Sciences at UT Austin\\
        \noindent
    {\bf Leszek Demkowicz}\\
    Institute for Computational Engineering and Sciences at UT Austin\\
        \noindent
    {\bf Shivkumar Chandrasekaran}\\
    University of California, Santa Barbara\\
        \noindent
    {\bf Hrushikesh Mhaskar}\\
    Claremont Graduate University\\
        \noindent
    {\bf Franck Assous}\\
    Ariel University Center\\
        \noindent
    {\bf Joel Chaskalovic}\\
    D'Alembert, University Pierre and Marie Curie\\
        \noindent
    {\bf Heyrim Cho}\\
    Brown university\\
        \noindent
    {\bf George Karniadakis}\\
    Brown university\\
        \noindent
    {\bf Daniele Venturi}\\
    Brown university\\
        \noindent
    {\bf Minseok Choi}\\
    Brown University\\
        \noindent
    {\bf George Karniadakis}\\
    Brown University\\
        \noindent
    {\bf Themistoklis Sapsis}\\
    New York University\\
        \noindent
    {\bf Eric Chung}\\
    The Chinese University of Hong Kong\\
        \noindent
    {\bf Radian Belu}\\
    Drexel University\\
        \noindent
    {\bf Irina Ciobanescu Husanu}\\
    Drexel University\\
        \noindent
    {\bf Radian Belu}\\
    Drexel University\\
        \noindent
    {\bf Irina Ciobanescu Husanu}\\
    Drexel University\\
        \noindent
    {\bf Jonathan Cohen}\\
    NVIDIA\\
        \noindent
    {\bf A. Quarati, A. Clematis, E. Danovaro, A. Galizia, D. D'Agostino}\\
    CNR-IMATI\\
        \noindent
    {\bf Emanuele Danovaro, Alfonso Quarati}\\
    CNR-IMATI\\
        \noindent
    {\bf Andrea Clematis, Antonella Galizia, Daniele D'Agostino}\\
    CNR-IMATI\\
        \noindent
    {\bf Daniele D'Agostino}\\
    CNR-IMATI\\
        \noindent
    {\bf A. Galizia, D. D'Agostino A. Clematis}\\
    CNR-IMATI\\
        \noindent
    {\bf Rida Assaf}\\
    Western Michigan University\\
        \noindent
    {\bf Elise Helene J. de Doncker}\\
    Western Michigan University\\
        \noindent
    {\bf Carlo de Falco}\\
    MOX - Department of Mathematics - Politecnico di Milano\\
        \noindent
    {\bf Ivo Dolezel}\\
    Czech Technical University in Prague, Faculty of Electrical Engineering\\
        \noindent
    {\bf Vaclav Kotlan}\\
    University of West Bohemia, Faculty of Electrical Engineering\\
        \noindent
    {\bf Bohus Ulrych}\\
    University of West Bohemia, Faculty of Electrical Engineering\\
        \noindent
    {\bf Jerzy Barglik}\\
    Silesian University of Technology, Faculty of Material Science and Metallurgy, Katowice, Poland\\
        \noindent
    {\bf Ivo Dolezel}\\
    Czech Technical University, Faculty of Electrical Engineering, Praha, Czech Republic\\
        \noindent
    {\bf Roman Przylucki}\\
    Silesian University of Technology, Faculty of Material Science and Metallurgy, Katowice, Poland\\
        \noindent
    {\bf Albert Smalcerz}\\
    Silesian University of Technology, Faculty of Material Science and Metallurgy, Katowice, Poland\\
        \noindent
    {\bf Ivo Dolezel}\\
    Czech Technical University in Prague\\
        \noindent
    {\bf Jiri Doubek}\\
    Czech Technical University in Prague\\
        \noindent
    {\bf Jan Kyncl}\\
    Czech Technical University in Prague\\
        \noindent
    {\bf Lubomir Musalek}\\
    Czech Technical University in Prague\\
        \noindent
    {\bf Ladislav Musil}\\
    Czech Technical University in Prague\\
        \noindent
    {\bf Suchuan Dong}\\
    Purdue University\\
        \noindent
    {\bf Mahmoud El-Borai}\\
    Professor of mathematics Faculty of Science Alexandria University\\
        \noindent
    {\bf Khairia El-Nadi}\\
    Professor of mathematics Faculty of Science Alexandria University\\
        \noindent
    {\bf Mahmoud El-Borai}\\
    Professor of Mathematics Faculty of Science Alexandria University Egypt Egypt\\
        \noindent
    {\bf Khairia El-Nadi}\\
    Professor of Mathematics Faculty of Science Alexandria University  Egypt\\
        \noindent
    {\bf Omid Reza Esmaeili Motlagh}\\
    University Teknikal Malaysia Melaka\\
        \noindent
    {\bf Seyed Mahdi Homayouni}\\
    Lenjan Branch,  Islamic Azad University of Iran\\
        \noindent
    {\bf Sai Hong Tang}\\
    University Putra Malaysia\\
        \noindent
    {\bf Charbel Farhat}\\
    Department of Aeronautics and Astronautics Department of Mechanical Engineering Institute for Computational and Mathematical Engineering Stanford University, USA\\
        \noindent
    {\bf Alex Main}\\
    Institute for Computational and Mathematical Engineering Stanford University, USA\\
        \noindent
    {\bf Hadi Fekrmandi}\\
    Department of Mechanical and Materials Engineering, Florida International University\\
        \noindent
    {\bf Cesar Levy}\\
    Department of Mechanical and Materials Engineering, Florida International University\\
        \noindent
    {\bf Qin Ma}\\
    Edward F. Cross School of Engineering, Walla Walla University\\
        \noindent
    {\bf Mordechai Perl}\\
    Department of Mechanical Engineering, Ben Gurion University of the Negev\\
        \noindent
    {\bf Mike Fowler}\\
    Clarkson University\\
        \noindent
    {\bf Jos\'e Luis Gal\'an--Garc\'{\i}a}\\
    Applied Mathematics Dept., Universidad de M\'alaga, Spain\\
        \noindent
    {\bf Alberto Garc\'{\i}a--\'Alvarez}\\
    Deputy Director for Renfe (Spanish Railways) Passengers Services, Spain\\
        \noindent
    {\bf Luis Mesa}\\
    Spanish Railways Foundation, Spain\\
        \noindent
    {\bf Eugenio Roanes--Lozano}\\
    Algebra Dept., Universidad Complutense de Madrid, Spain\\
        \noindent
    {\bf Mariangel Garcia}\\
    Computational Science Research Center - San Diego State University\\
        \noindent
    {\bf Lawrence Bush}\\
    University of Wyoming\\
        \noindent
    {\bf Victor Ginting}\\
    University of Wyoming\\
        \noindent
    {\bf Yuliya Gorb}\\
    University of Houston\\
        \noindent
    {\bf nicolin govender}\\
    CSIR, University of Pretoria\\
        \noindent
    {\bf Ouahiba Guerri}\\
    LaSIE. Université de La Rohelle\\
        \noindent
    {\bf Aziz Hamdouni}\\
    LaSIE. Université de La Rohelle\\
        \noindent
    {\bf Erwan Liberge}\\
    LaSIE. Université de La Rohelle\\
        \noindent
    {\bf Mingyan He}\\
    Department of Mathematics,Tongji University\\
        \noindent
    {\bf Ziping Huang}\\
    Department of Mathematics, Tongji University\\
        \noindent
    {\bf Pengtao Sun}\\
    Department of Mathematical Sciences,University of Nevada Las Vegas\\
        \noindent
    {\bf Yuzhou Sun}\\
    Department of Mathematical Sciences,University of Nevada Las Vegas\\
        \noindent
    {\bf Cheng Wang}\\
    Department of Mathematics,Tongji University\\
        \noindent
    {\bf Jeffrey Heys}\\
    Montana State University\\
        \noindent
    {\bf Seyed Mahdi Homayouni}\\
    Department of Industrial Engineering, Lenjan Branch, Islamic Azad University, Zarrinshahr, Isfahan, Iran.\\
        \noindent
    {\bf O. Motlagh}\\
    Faculty of Manufacturing Engineering, Universiti Teknikal Malaysia Melaka (UTeM), 76100 Melaka, Malaysia\\
        \noindent
    {\bf Sai Hong Tang}\\
    Mechanical and Manufacturing Engineering Department, Universiti Putra Malaysia (UPM), Selangor, Malaysia\\
        \noindent
    {\bf Neihad Hussen Al-Khalidy}\\
    SLR Consulting, 2 Lincoln Street, Lane Cove, NSW 2066, Australia\\
        \noindent
    {\bf RAKESH KUMAR JHA}\\
    SVNIT SURAT\\
        \noindent
    {\bf RAKESH KUMAR JHA}\\
    SVNIT SURAT\\
        \noindent
    {\bf August Johansson}\\
    Department of Mathematics, UC Berkeley\\
        \noindent
    {\bf Suwan Juntiwasarakij}\\
    Computer Science, Faculty of Science, King Mongkut's Institute of Technology Ladkrabang, Bangkok, Thailand\\
        \noindent
    {\bf Bora Kalpakli}\\
    ROKETSAN Missile Industry\\
        \noindent
    {\bf Yusuf Ozyoruk}\\
    Middle East Technical University\\
        \noindent
    {\bf Hakan I. Tarman}\\
    Middle East Technical University\\
        \noindent
    {\bf Sandeep Koranne}\\
    Mentor Graphics Corporation\\
        \noindent
    {\bf Pavel Karban}\\
    Faculty of Electrical Engineering, University of West Bohemia\\
        \noindent
    {\bf Lukáš Korous}\\
    Faculty of Electrical Engineering, University of West Bohemia\\
        \noindent
    {\bf Pavel Kůs}\\
    Faculty of Electrical Engineering, University of West Bohemia\\
        \noindent
    {\bf František Mach}\\
    Faculty of Electrical Engineering, University of West Bohemia\\
        \noindent
    {\bf Pavel Šolín}\\
    University of Nevada, Reno\\
        \noindent
    {\bf Pavel Karban, Jindřich Jansa, David Pánek}\\
    University of West Bohemia\\
        \noindent
    {\bf Lukáš Koudela}\\
    University of West Bohemia\\
        \noindent
    {\bf Oldřich Tureček, Ladislav Zuzjak, Martin Schlosser, Jan Altman}\\
    University of West Bohemia\\
        \noindent
    {\bf Fritz Kretzschmar}\\
    Graduate School of Computational Engineering,  Technische Universitaet Darmstadt\\
        \noindent
    {\bf Roman Hamar}\\
    University of West Bohemia, Pilsen, Czech Republic\\
        \noindent
    {\bf Petr Krop\'{i}k}\\
    University of West Bohemia, Pilsen, Czech Republic\\
        \noindent
    {\bf Lenka \v{S}roubov\'{a}}\\
    University of West Bohemia, Pilsen, Czech Republic\\
        \noindent
    {\bf Miroslav Hromádka}\\
    University of West Bohemia\\
        \noindent
    {\bf Pavel Karban}\\
    University of West Bohemia\\
        \noindent
    {\bf Zdeněk Kubík}\\
    University of West Bohemia\\
        \noindent
    {\bf Denys Nikolayev}\\
    University of West Bohemia\\
        \noindent
    {\bf Jiří Skála}\\
    University of West Bohemia\\
        \noindent
    {\bf Michal Kuraz}\\
    Czech University of Life Sciences Prague, Faculty of Environmental Sciences, Department of Water Resources and Environmental Modeling, Czech Republic\\
        \noindent
    {\bf Petr Mayer}\\
    Czech Technical University in Prague, Faculty of Civil Engineering, Department of Mathematics, Czech Republic\\
        \noindent
    {\bf Pavel Karban}\\
    Department of Theory of Electrical Engineering, University of West Bohemia, Czech Republic\\
        \noindent
    {\bf Lukáš Korous}\\
    Department of Theory of Electrical Engineering, University of West Bohemia, Czech Republic\\
        \noindent
    {\bf Pavel Kůs}\\
    Department of Theory of Electrical Engineering, University of West Bohemia, Czech Republic\\
        \noindent
    {\bf František Mach}\\
    Department of Theory of Electrical Engineering, University of West Bohemia, Czech Republic\\
        \noindent
    {\bf Dmitri Kuzmin}\\
    University Erlangen-Nuremberg\\
        \noindent
    {\bf Walid Larbi}\\
    Conservatoire National des Arts et M\'{e}tiers, Paris, France\\
        \noindent
    {\bf Mats G. Larson}\\
    Mathematics, Umea University, Sweden\\
        \noindent
    {\bf Tor Troeng}\\
    Mathematics, Umea University, Sweden\\
        \noindent
    {\bf Clément Durochat}\\
    Inria Sophia Antipolis-Méditerranée\\
        \noindent
    {\bf Stéphane Lanteri}\\
    Inria Sophia Antipolis-Méditerranée\\
        \noindent
    {\bf Raphaël Léger}\\
    Inria Sophia Antipolis-Méditerranée\\
        \noindent
    {\bf Claire Scheid}\\
    University of Nice-Sophia Antipolis and Inria Sophia Antipolis-Méditerranée\\
        \noindent
    {\bf Jonathan Viquerat}\\
    Inria Sophia Antipolis-Méditerranée\\
        \noindent
    {\bf Jichun Li}\\
    University of Nevada Las Vegas\\
        \noindent
    {\bf Shengtai Li}\\
    Los Alamos National Laboratory\\
        \noindent
    {\bf Wenyuan Liao}\\
    University of Calgary\\
        \noindent
    {\bf Aaron Luttman}\\
    National Security Technologies, LLC\\
        \noindent
    {\bf Micha\l\, Odyniec}\\
    National Security Technologies, LLC\\
        \noindent
    {\bf Ivo Dole\v{z}el}\\
    University of West Bohemia\\
        \noindent
    {\bf Pavel K\r{u}s}\\
    University of West Bohemia\\
        \noindent
    {\bf Pavel Karban}\\
    University of West Bohemia\\
        \noindent
    {\bf Franti\v{s}ek Mach}\\
    University of West Bohemia\\
        \noindent
    {\bf Eric Machorro}\\
    National Security Technologies, LLC\\
        \noindent
    {\bf Hadi Manap}\\
    University of Malaysia Pahang (UMP)\\
        \noindent
    {\bf Lawrence Bush}\\
    University of Wyoming\\
        \noindent
    {\bf Victor Ginting}\\
    University of Wyoming\\
        \noindent
    {\bf Bradley McCaskill}\\
    University of Wyoming\\
        \noindent
    {\bf Reza Abedi}\\
    The University of Tennessee, Space Institute\\
        \noindent
    {\bf Scott Miller}\\
    Applied Research Lab, Penn State University\\
        \noindent
    {\bf Misun Min}\\
    Argonne National Laboratory\\
        \noindent
    {\bf William Mitchell}\\
    National Institute of Standards and Technology\\
        \noindent
    {\bf Muhammad Ikram Mohd Rashid}\\
    University of Malaysia Pahang\\
        \noindent
    {\bf Noraini Mohd Razali}\\
    Universiti Malaysia Pahang\\
        \noindent
    {\bf Kamel Nafa}\\
    Sultan Qaboos University\\
        \noindent
    {\bf Abigail Bowers}\\
    Clemson University, USA\\
        \noindent
    {\bf Eliot Fried}\\
    McGill University, Canada\\
        \noindent
    {\bf Tae-Yeon Kim}\\
    McGill University, Canada\\
        \noindent
    {\bf Monika Neda}\\
    University of Nevada Las Vegas\\
        \noindent
    {\bf Leo Rebholz}\\
    Clemson University, USA\\
        \noindent
    {\bf Marco Nehmeier}\\
    University of W\"urzburg\\
        \noindent
    {\bf John Pryce}\\
    Cardiff University\\
        \noindent
    {\bf J\"urgen Wolff von Gudenberg}\\
    University of W\"urzburg\\
        \noindent
    {\bf Ueli Koch}\\
    Lab for Electromagnetic Fields and Microwave Electronics (IFH), ETH Zurich, Switzerland\\
        \noindent
    {\bf Jens Niegemann}\\
    Lab for Electromagnetic Fields and Microwave Electronics (IFH), ETH Zurich, Switzerland\\
        \noindent
    {\bf Alberto Guardone}\\
    Politecnico di Milano\\
        \noindent
    {\bf Valentina Motta}\\
    Politecnico di Milano\\
        \noindent
    {\bf Michele Nini}\\
    Politecnico di Milano\\
        \noindent
    {\bf Jan Nov\'{a}k}\\
    Brno University of Technology, Faculty of Civil Engineering, Institute of Structural Mechanics, Veve\v{r}\'{i} 331/95, Brno, Czech Republic\\CTU in Prague, Faculty of Civil Engineering, Department of Mechnics, Th\'akurova 7, Prague, Czech Republic\\
        \noindent
    {\bf Jay Gopalakrishnan}\\
    Department of Mathematics and Statistics, Portland State University\\
        \noindent
    {\bf Ignacio Muga}\\
    Instituto de Matemáticas, Pontificia Universidad Católica de Valparaíso\\
        \noindent
    {\bf Nicole Olivares}\\
    Department of Mathematics and Statistics, Portland State University\\
        \noindent
    {\bf Hengguang Li}\\
    Wayne State University\\
        \noindent
    {\bf Jeffrey Ovall}\\
    University of Kentucky\\
        \noindent
    {\bf Vittorio Cecchetto}\\
    Roma Tre University\\
        \noindent
    {\bf Antonio DiCarlo}\\
    Roma Tre University\\
        \noindent
    {\bf Francesco Furiani}\\
    Roma Tre University\\
        \noindent
    {\bf Enrico Marino}\\
    Roma Tre University\\
        \noindent
    {\bf Alberto Paoluzzi}\\
    Roma Tre University\\
        \noindent
    {\bf Federico Spini}\\
    Roma Tre University\\
        \noindent
    {\bf Costas Papadimitriou}\\
    Department of Mechanical Engineering, University of Thessaly, Volos, Greece\\
        \noindent
    {\bf Dimitrios Papadimitriou}\\
    Department of Mechanical Engineering, University of Thessaly, Volos, Greece\\
        \noindent
    {\bf Jarom\'ir Hor\'a\v{c}ek}\\
    Institute of Thermomechanics Academy of Sciences, Dolej\v{s}kova 5, Prague 8, Czech Republic\\
        \noindent
    {\bf Karel Kozel}\\
    Institute of Thermomechanics Academy of Sciences, Dolej\v{s}kova 5, Prague 8, Czech Republic\\
        \noindent
    {\bf Petra Po\v{r}\'izkov\'a}\\
    Czech Technical University in Prague, Karlovo n\'am\v{e}st\'i 13, 121 35, Prague 2, Czech Republic\\
        \noindent
    {\bf Minvydas Ragulskis}\\
    Kaunas University of Technology\\
        \noindent
    {\bf Saeed Rahman}\\
    Northwestern Polytecnical University\\
        \noindent
    {\bf Roberto Armellin}\\
    Politecnico di Milano\\
        \noindent
    {\bf Alberto Guardone}\\
    Politecnico di Milano\\
        \noindent
    {\bf Nawin Ryan Nannan}\\
    Anton de Kom Universiteit van Suriname\\
        \noindent
    {\bf Barbara Re}\\
    Politecnico di Milano\\
        \noindent
    {\bf Jesse Chan}\\
    University of Texas at Austin\\
        \noindent
    {\bf Leszek Demkowicz}\\
    University of Texas at Austin\\
        \noindent
    {\bf Robert Moser}\\
    University of Texas at Austin\\
        \noindent
    {\bf Nathan V. Roberts}\\
    University of Texas at Austin\\
        \noindent
    {\bf Andrew Christlieb}\\
    Michigan State University\\
        \noindent
    {\bf James Rossmanith}\\
    Iowa State University\\
        \noindent
    {\bf David Seal}\\
    Michigan State University\\
        \noindent
    {\bf Florian Rudolf}\\
    Institute for Microelectronics, Technische Universität Wien\\
        \noindent
    {\bf Karl Rupp}\\
    MCS Division, Argonne National Laboratory\\
        \noindent
    {\bf Siegfried Selberherr}\\
    Institute for Microelectronics, Technische Universität Wien\\
        \noindent
    {\bf Marek Bergander}\\
    AGH University of Science and Technology\\
        \noindent
    {\bf Jacek Cieślik}\\
    AGH University of Science and Technology\\
        \noindent
    {\bf Rafał Rumin}\\
    AGH University of Science and Technology\\
        \noindent
    {\bf Karl Rupp}\\
    Argonne National Laboratory\\
        \noindent
    {\bf Barry Smith}\\
    Argonne National Laboratory\\
        \noindent
    {\bf Andrea Sacconi}\\
    Department of Mathematics - Imperial College London, UK\\
        \noindent
    {\bf Jose Castillo}\\
    San Diego State University\\
        \noindent
    {\bf Christopher Paolini}\\
    San Diego State University\\
        \noindent
    {\bf Eduardo Sanchez}\\
    San Diego State University\\
        \noindent
    {\bf Jose Castillo}\\
    San Diego State University\\
        \noindent
    {\bf Christopher Paolini}\\
    San Diego State University\\
        \noindent
    {\bf Eduardo Sanchez}\\
    San Diego State University\\
        \noindent
    {\bf Arkadiusz Miaskowski}\\
    University of Life Sciences in Lublin\\
        \noindent
    {\bf Bartosz Sawicki}\\
    Warsaw University of Technology\\
        \noindent
    {\bf Sascha M Schnepp}\\
    Laboratory for Electromagnetic Fields and Microwave Electronics (IFH), ETH Zurich, Switzerland\\
        \noindent
    {\bf Shusil Maurya}\\
    Department of Earth and Planetray Sciences, University of Allahabad, India\\
        \noindent
    {\bf Pitam Singh}\\
    Department of Mathematics, Motilal nehru National Institute of Technology, Allahabad, India\\
        \noindent
    {\bf Priyamvada Singh}\\
    Department of  Earth and Planetary Sciences, Universty of Allahabad,  India\\
        \noindent
    {\bf Priyamvada Singh}\\
    Uinversity of Allahabad\\
        \noindent
    {\bf J. N. Tripathi}\\
    Uinversity of Allahabad\\
        \noindent
    {\bf Pavel Solin}\\
    University of Nevada, Reno\\
        \noindent
    {\bf Uaday Singh}\\
    Indian Institute of Technology Roorkee, Roorkee, Uttarakhand, 247667\\
        \noindent
    {\bf Shailesh Kumar Srivastava}\\
    Indian Institute of Technology Roorkee, Roorkee, Uttarakhand, 247667\\
        \noindent
    {\bf Adisak Sukul}\\
    Department of Computer Science, Faculty of Science, King Mongkut's Institute of Technology Ladkrabang, Thailand.\\
        \noindent
    {\bf Pengtao Sun}\\
    Department of Mathematical Sciences, University of Nevada Las Vegas\\
        \noindent
    {\bf Mingyan He}\\
    Tongji University\\
        \noindent
    {\bf Pengtao Sun}\\
    University of Nevada, Las Vegas\\
        \noindent
    {\bf Yuzhou Sun}\\
    University of Nevada, Las Vegas\\
        \noindent
    {\bf Petr Louda}\\
    Czech Technical University in Prague, Faculty of Mechanical Engineering, Department of Technical Mathematics\\
        \noindent
    {\bf Petr Sváček}\\
    Czech Technical University in Prague, Faculty of Mechanical Engineering, Department of Technical Mathematics\\
        \noindent
    {\bf Petr Sváček}\\
    Czech Technical University in Prague, Faculty of Mechanical Engineering, Department of Technical Mathematics\\
        \noindent
    {\bf GHANSHYAM THAKUR}\\
    MANIT,Bhopal\\
        \noindent
    {\bf Jose Castillo}\\
    San Diego State University\\
        \noindent
    {\bf Mary Thomas}\\
    San Diego State University\\
        \noindent
    {\bf Gretar Tryggvason}\\
    University of Notre Dame\\
        \noindent
    {\bf Alberto Guardone}\\
    Politecnico di Milano, Dipartimento di Scienze e Tecnologie Aerospaziali\\
        \noindent
    {\bf Federica Vignati}\\
    Politecnico di Milano, Dipartimento di Scienze e Tecnologie Aerospaziali\\
        \noindent
    {\bf William Peter Boshoff}\\
    Stellenbosch University\\
        \noindent
    {\bf Jan Vorel}\\
    Czech Technical University in Prague\\
        \noindent
    {\bf Byunghyun Jang}\\
    The University of Mississippi\\
        \noindent
    {\bf Yafei Jia}\\
    The University of Mississippi\\
        \noindent
    {\bf Zhangping Wei}\\
    The University of Mississippi\\
        \noindent
    {\bf Karl Rupp}\\
    MCS Division, Argonne National Laboratory\\
        \noindent
    {\bf Siegfried Selberherr}\\
    Institute for Microelectronics, Technische Universität Wien\\
        \noindent
    {\bf Josef Weinbub}\\
    Institute for Microelectronics, Technische Universität Wien\\
        \noindent
    {\bf Troy Butler}\\
    Colorado State University\\
        \noindent
    {\bf Tim Wildey}\\
    Sandia National Labs\\
        \noindent
    {\bf Antony Farrington}\\
    Imperial College London\\
        \noindent
    {\bf Peter Vincent}\\
    Imperial College London\\
        \noindent
    {\bf Freddie Witherden}\\
    Imperial College London\\
        \noindent
    {\bf Kwai Wong, Andrew Kail, Xiaopeng Zhao}\\
    University of Tennessee\\
        \noindent
    {\bf Edurado D'Azevedo}\\
    Oak Ridge National Laboratory\\
        \noindent
    {\bf Zhiang Hu}\\
    Chinese University of Hong Kong\\
        \noindent
    {\bf Kwai Wong, Shiquan Su}\\
    kwong@utk.edu\\
        \noindent
    {\bf Guenter Hofstetter}\\
    innsbruck university\\
        \noindent
    {\bf Zhenzhong Shen}\\
    hohai University\\
        \noindent
    {\bf Juncai Xu}\\
    hohai University\\
        \noindent
    {\bf Manoj Kumar Yadav}\\
    NIT, Patna, India\\
        \noindent
    {\bf Pramod Kumar Yadav}\\
    Motilal Nehru National Institute of Technology Allahabad, India\\
        \noindent
    {\bf Thomas Blasingame}\\
    Texas A and M University\\
        \noindent
    {\bf George Moridis}\\
    Lawrence Berkeley National Laboratory\\
        \noindent
    {\bf Daegil Yang}\\
    Texas A and M University\\
        \noindent
    {\bf George Karniadakis}\\
    Brown University\\
        \noindent
    {\bf Xiaoliang Wan}\\
    Louisiana State University\\
        \noindent
    {\bf Xiu Yang}\\
    Brown University\\
        \noindent
    {\bf David Darmofal}\\
    Massachusetts Institute of Technology\\
        \noindent
    {\bf Masayuki Yano}\\
    Massachusetts Institute of Technology\\
        \noindent
    {\bf Masaya Yoshikawa}\\
    Meijo university\\
        \noindent
    {\bf Johnny Guzman}\\
    Brown University\\
        \noindent
    {\bf George Karniadakis}\\
    Brown University\\
        \noindent
    {\bf Yue Yu}\\
    Brown University\\
        \noindent
    {\bf George Em Karniadakis}\\
    Division of Applied Mathematics, Brown University\\
        \noindent
    {\bf Mohsen Zayernouri}\\
    Division of Applied Mathematics, Brown University\\
        \noindent
    {\bf Wu Zhang}\\
    Shanghai University\\
        \noindent
    {\bf Aihui Zhou}\\
    Institute of Computational Mathematics and Scientific/Engineering Computing, Academy of Mathematics and Systems Science, Chinese Academy of Sciences\\
    \end{document}