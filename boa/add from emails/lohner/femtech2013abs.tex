
\input usualformat
\input abbrevs
\input epsf

\bf
\cl{A General-Purpose Fortran to GPU Translator}
\ms \noi
\cl{Rainald L\"ohner}
\cl{CFD Center, George Mason University}
\cl{Fairfax, VA 22030-4444}
\rm
\bs \noi
Even though many of those that code in Java, C++, CUDA and
MATLAB/Mathematica may find this hard to believe, the fact
remains that a large number of scientific computing codes
are written either entirely or partially in Fortran.
\par \noi
The real value of these so-called
legacy codes does not reside so much in the basic numerical
techniques, but in the many physics models and engineering
options that have been incorporated, debugged and validated
over the years (sometimes over decades). Given that these
options represent hundreds of man-years of dedicated and
labour-intensive work, it appears unlikely that these codes
will be rewritten in coming years.
\par \noi
It is therefore not surprising that every time a new generation
of computer hardware appears, the imminent demise of these
legacy codes is predicted. The observation made time and again
is that after a few years, these same legacy codes are modified
so that they run efficiently on the new hardware.
\par \noi
The appearance of graphical processing units (GPU) is yet
another technological leap. As further increases in clockcycles
are unlikely, GPUs may become a key component of scientific
computing in the coming decades. The question then becomes how
to best port scientific computing codes written in Fortran to
GPUs in a way that is transparent, effective and error-free.
\par \noi
This has been the goal of ASI Parallel Fortran, a general-purpose
Fortran to GPU translator.
\par \noi
While other automatic approaches often compromise on performance,
negating the ultimate purpose of porting legacy codes to GPUs,
ASI Parallel Fortran ensures that the unique performance requirements
of GPUs are fulfilled.
In summary, ASI Parallel Fortran transparently:
\par \noi
\item{-} Implements Fortran loops as high-performance GPU kernels;
\item{-} Represents multi-dimensional Fortran arrays using
a memory layout appropriate for GPU performance requirements;
\item{-} Minimises data transfer by tracking array shapes and enforcing
  consistency across subroutine calls;
\item{-} Handles GPU array I/O (via read/write/print) and memory
transfer;
\item{-} Integrates with MPI for hybrid parallelism;
\item{-} Maintains compatibility with pure-CPU code; and 
\item{-} Interfaces with Thrust and other GPU libraries.
\par \noi
The talk will show in detail what ASI Parallel Fortran does, 
how it is used and implemented, and demonstrate the use of 
ASI Parallel Fortran for general purpose CFD/CEM/Thermal Finite
Element and Finite Difference codes. The codes (automatically)
ported to date include FEFLO, a FEM CFD code with more than a million
lines, FEHEAT, a nonlineat thermodynamics code, and FDSOL, a FDM
CFD code for near-incompressible flows. The smaller codes were
ported to the GPU environment in just a few hours, demonstrating
the ease of use and maturity of the translator.

\bs \noi
\ub{Example: Blast in A Room}
\ms \noi
This example, run with FEFLO, considers the flow and particle transport 
resulting from a blast in a room where dilute material has been deposited. 
The flow interacts with a powder-like material that is modeled via particles.
The geometry, together with the solution, can be discerned from
Figure~1.

\vbox{
\vskip 18pt
\centerline{
\epsfxsize=8.0cm
\epsffile{blast_in_room_part_velo.eps}
}
\vskip 5pt
\centerline{Figure 1~~Blast in Room With Dilute Material}
%\vskip 18pt
}

\ms \noi
Table 1 shows timings studies carried out with the following set of
parameters: compressible Euler, ideal gas EOS, explicit FEM-FCT,
initialization from a 1-D file, 4.0~Mels, 93.5~Kparts, 60 timesteps.
One can see that the automatically produced GPU code runs efficiently.

$$\vbox{
\ms
\leftline {{\bf Table 1} \quad Blast in Room With Dilute Material}
\bigskip
\hrule
\settabs
\+& nelem       \quad 
  & CPU/GPU     \quad \quad \quad
  & mvecl       \quad 
  & Time (sec)  \cr
                                                     %SAMPLE LINE
\smallskip
\+& nelem & CPU/GPU & mvecl & Time (sec) \cr
\medskip
\hrule
\medskip
\+& 4.0 M & Xeon E5530 (1) & 32 & 305 \cr
\+& 4.0 M & Xeon E5530 (2) & 32 & 178 \cr
\+& 4.0 M & Xeon E5530 (4) & 32 & 113 \cr
\+& 4.0 M & Xeon E5530 (6) & 32 &  79 \cr
\+& 4.0 M & Xeon E5530 (8) & 32 &  68 \cr
\+& 4.0 M & Tesla C2070    & 51200 &  49 \cr
\medskip
\hrule }$$

\vfill \eject
\end
